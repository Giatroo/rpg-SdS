%!TeX root=../Rules.tex
\chapter{Raças}%
\label{cha:racas}
\begin{multicols}{2}

As raças continuam exatamente com as mesmas características que o Livro do
Jogador. Aqui detalharei melhor um pouco sobre como essas raças se dispõem na
\textit{lore} do mundo para que o jogador possa fazer um melhor \textit{role
play} e escolher uma raça que tenha uma história que o agrade mais.

As raças continuam exatamente com as mesmas características que o Livro do
Jogador. A única exceção é o \textit{tiefling} que não existe e o
\textit{meio-orc} que foi modificado. Além disso, há uma nova raça chama de
\textit{automato}.

Para detalhes de \textit{lore} de cada uma das raças, acesse o
\href{https://www.notion.so/giatro/Ra-as-bbe65832d376458898bacc23741f00d3?pvs=4}
{notion}.

\section{Meio-Orc}%

O Meio-Orc é uma raça focada em força e resistência. O principal aspecto da raça
é receber +2 em Força e +1 em Constituição. A única modificação feita aqui é em
questões de role play. Ao invés dessa raça se chamar Meio-Orc, ela se chama
\textit{Raças de Cafrilar} e pode ser algum dos tipos abaixo:

\begin{itemize}
    \item \textit{Meio-Orc};
    \item \textit{Meio-Ogro};
    \item \textit{Meio-Troll};
    \item \textit{Meio-Goblin};
    \item \textit{Meio-Gnoll};
    \item \textit{Meio-Bugbear};
    \item \textit{Humano Selvagem};
    \item \textit{Elfo Selvagem};
\end{itemize}

No caso de ser um Humano ou Elfo Selvagem, esses nomes são apenas para
diferenciar de um humano ou elfo comum. Mas em questão de role play, o jogador
se apresenta como humano ou elfo.

\section{Automatos}%

Automatos são um tipo de raça criada pelos gnomos para servir as suas
necessidades básicas como serviçais. Entretanto, com o passar do tempo, esses
seres foram ganhando consciência própria e se tornando uma raça quase
independente.

Automatos são uma classe focada em inteligência e força, o que os torna bastante
únicos.

\paragraph{Aumento no Valor de Habilidade}
Seu valo de Inteligência aumenta em 2 e seu valor de Força aumenta em 1 (ou o
contrário, à escolha do jogador).

\paragraph{Idade}
Automatos são criados e não nascem. Eles são criados com a aparência de um
humano de 20 anos e não envelhecem. Entretanto, eles podem ser destruídos e
precisam ser consertados.

\paragraph{Tendência}
Automatos são leais e bons em sua essência. Entretanto, eles podem ser
programados para serem maus e caóticos.

\paragraph{Tamanho}
Automatos são geralmente do tamanho de um gnomo, mas podem ser maiores ou
menores dependendo do seu propósito.

\paragraph{Deslocamento}
Seu deslocamento base de caminhada é de 9 metros.

\paragraph{Módulos Adicionais}
Automatos podem ser modificados com módulos adicionais que lhes dão
habilidades especiais. Esses módulos podem ser fabricados pelos próprios
automatos dados que eles tenham os materiais necessários (metal).

\begin{itemize}
    \item \textbf{Módulo de Visão no Escuro}: Visão no escuro de 18 metros.
        Custo de 50 pp, 10 po, 5 pc;
    \item \textbf{Módulo de Idioma Adicional}: Aprende um idioma adicional
        (dentre os básicos falados por criaturas humanoides). Custo de 30 pp, 5
        po, 10 pc;
    \item \textbf{Módulo de Sopro Elemental}: Pode usar uma ação para soltar um
        sopro elemental do seu elemento. Todas as criaturas em uma área de 4,5
        metros de raio devem fazer um teste de resistência de Constituição. A CD
        é igual a 8 + seu modificador de Constituição + seu bônus de
        proficiência. Uma criatura sofre 2d6 de dano do tipo escolhido em um
        teste falho, ou metade desse dano em um teste bem sucedido. O dano
        aumenta para 3d6 no 6º nível, 4d6 no 11º nível e 5d6 no 16º nível. Você
        pode usar esse sopro uma vez por descanso longo. Custo de 100 pp, 20 po,
        10 pc;
    \item \textbf{Módulo de Arma}: Você pode sacar uma arma em uma de suas
        mãos utilizando uma ação. Essa arma é uma arma simples corpo-a-corpo que
        causa 1d6 de dano do tipo perfurante. Você é proficiente com essa arma.
        Custo de 20 pp, 5 po, 5 pc;
\end{itemize}



\end{multicols}

