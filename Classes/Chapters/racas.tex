%!TeX root=../Rules.tex
\chapter{Raças}%
\label{cha:racas}
\begin{multicols}{2}

As raças continuam exatamente com as mesmas características que o Livro do
Jogador. Aqui detalharei melhor um pouco sobre como essas raças se dispõem na
\textit{lore} do mundo para que o jogador possa fazer um melhor \textit{role
play} e escolher uma raça que tenha uma história que o agrade mais.

\section{Humanos}%

Os humanos, raça mais expansionista de Jǒna, estão por toda parte: desde as
pequenas Cidades Amigas no Sul de Estália até as tribos isoladas em Cafrilar.
Entretanto, o grande ponto de concentração dos humanos é na famosa Península do
Goto, onde ficam as cidades de Calífora, Nize e Argan.

Essas cidades surgiram das guerras e unificações entres antigas tribos (ou
autodenominadas guildas) que habitavam os ricos da região, princialmente o
Kaitû e o Jote. Há muitos anos, uma série de guildas habitavam o rio Kaitû, até
que as águas dele começaram a ficar venenosas e intragáveis. Os Archon e os
Grégios, duas dessas guildas, entraram em guerra culpando uma à outra pelo
ocorrido. Outras guildas como os Taêno migraram da região para o norte, onde
encontraram um novo rio, que hoje é conhecido como Rio Taêno, e lá montaram suas
residências. Já os Saicro habitavam o rio Jote, e forneciam água para os que
habitavam o Kaitû, cobrando caro por isso e impedindo que essas guildas
migrassem para o Jote.

Após vários eventos, dois humanos conhecidos como Aaron e Victor Wilton chegaram
até a região e uniram essas guildas revelando serem magos Saicro que estavam o
tempo todo envenenando a água só para vender água potável para essas guildas.
Então as guildas do Kaitû se juntaram e travaram uma guerra contra os Saicro e
outras guildas do norte, expulsando-os mais e mais para o norte e tomando o Jote
para si.

As guildas humanas no Norte fundaram a cidade de Nize, que fica na costa Norte
da península e tem sua divisão praticamente marcada pelo Rio Nize. Já as guildas
do Sul se uniram e fundaram Calífora, a maior potência economia e militar de
toda Estália até os dias de hoje.


\section{Draconianos}%

Os Draconianos são uma das raças mais incomuns em Estália. Eles são oriundos do
longínquo Império de Kenia, conhecido no continente do Sul como o Império dos
Dragões.

Kenia é um império antiquíssimo, que já existia na época de Aaron I, o fundador
de Calífora, e que muitos dizem ter nascido de uma cidade mais antiga ainda
conhecida como Retuc, que hoje é a atual capital do império.

O continente de Cafrilar, onde fica Kenia, é conhecido por ter um clima árido e
desértico, e ser de difícil acesso. Todas as outras raças de Jǒna pouco se
adaptam a esse ambiente, mas para os draconianos, esse não poderia ser um lar
melhor.

Poucos são aqueles de Estália que já foram até Kenia, e a maioria deles são
mercadores que navegam no Entreterras e chegam ao Porto de Das-Galul, talvez um
dos maiores e mais belos portos do mundo. O porto fica na cidade de Galul-Gadev.
Ela fica ao sul de Retuc e é a principal exportadora dos caríssimos produtos
kenianos.

Os povos do Sul adoram e pagam muito caro pelos produtos advindos de Retuc.
Kenia é bastante conhecida por seus palácios majestosos e aqueles realmente
ricos no Sul tentam copiar essa cultura comprando diversos tipos de produtos:
desde a culinária, com especiarias e farinhas das mais diversas; passando pelos
produtos de beleza, como pó de maquiagem, bijuterias de ouro e prata dos mais
puros, vestidos cravejados de gemas preciosas ou até itens mágicos como cajados
formidáveis e tomos de capa de couro de criaturas do deserto; e também materiais
de decoração como enormes tapetes, vasos de cerâmica, castiçais e candelabros,
quadros e esculturas, todos feitos à mão pelos renomados artistas draconianos da
capital de Kenia.

Apesar de Galul-Gadev ser uma cidade riquíssima, devido ao valioso comércio com
os povos de Estália, a capital do Império dos Dragões é Retuc, sem dúvida a
maior cidade de Cafrilar e, segundo alguns, a maior cidade de todo o mundo
(muito mais bonita que Calífora, confirme dizem).

Na cidade de Retuc se encontram os mais ricos e poderoso draconianos de todos.
Ela é uma cidade que fica mais ao norte de Cafrilar, a vários quilômetros de
Galul-Gadev, na beira de um enorme lago no meio do deserto, fazendo dela um
local perfeito para se habitar. Ali são pouquíssimos os não-draconianos e esses
não são tratados com muita empatia.

Da mesma forma que os sulistas adoram fazer negócios com os kenianos, eles
também os temem veementemente. Kenia é um império escravagista e extremamente
militarizado. Toda a vida em Retuc é uma vida militar. Os draconianos homens são
retirados de suas mães logo cedo e enfiados em barracas no meio de campos
militares onde estão destinados a treinar para se tornarem soldados ou falhar
tentando. Os que se tornam soldados ficam conhecidos como Salgum e são o cargo
mais honroso possível. Já os que falham acabam se tornando Gadum, que são cargos
secundários como comerciantes, construtores, etc.

Os Saldum por sua vez possuem uma hierarquia entre si, que começa nos
adolescentes recém formados pelo processo seletivo e chega nos grandes generais
dos exércitos de cada um dos elementos draconianos, dos quais alguns fazem parte
do conselho do império que aconselha o grande imperador, ou Dracon-Gastol como
falam os draconianos.

Já as mulheres, muitas vezes, ficam com um papel secundário no império, mas que
dão frutos a grande parte da cultura de lindos objetos de decoração e
embelezamento. Algumas delas, entretanto, se tornam Senhoras do Destino, ou
Sas-Ladnun. Elas são uma espécie de ordem de draconianas feiticeiras cujas mais
velhas são extremamente poderosas. Elas não vão ao campo de batalha como os
homens, mas exercem seu poder secretamente. Nem sequer o próprio Dracon-Gastol
sabe ao certo como a ordem realmente opera. Este é um segredo apenas das
próprias Sas-Ladnun.

\section{Anões}%

Uma das raças menos sociáveis de Estália, os anões são conhecidos por sua
arquitetura impecável e suas moradias dentro de montanhas, algo que qualquer
outra raça fica incrédula ao ver.

A sociedade anã não é muito grande. Eles se contentam com suas moradias em
montanhas e só saem de lá sejam impelidos por alguma força externa como ocorreu
na cordilheira do Kaitû anos atrás.

Os anões, em geral, contam os anos desde a Diáspora de Malor, como é conhecida a
antiga metrópole anã. Malor era uma cidade gigantesca que, segundo dizem, foi
criada por guildas anãs antes mesmo das guildas humanas se estabelecerem nas
margens do Jote dentro das montanhas da cordilheira de onde nasce o rio Kaitû,
ao sul da Península do Goto. A cidade era enorme e prosperou por muitos anos. Os
anões eram talvez uma das raças mais ricas do mundo, minerando metais e
materiais precisos que eram vendidos principalmente paras os elfos das florestas
ao sul de Estália e para os humanos do norte da Península do Goto, onde surgia a
cidade de Nize.

Os anões jamais deixariam seu querido lar se não fosse por um desastre até hoje
não completamente explicado. Muitas teorias existem, mas o fato é que algo
aconteceu de forma que os Espíritos do Mal começaram a invadir e brotar dentro
de Malor. Isso chegou a um ponto que ficou insustentável: várias outras
monstruosidades começaram a se aproveitar da fraqueza anã para invadir e se
apossar gradualmente dos salões enormes de Malor.

Dessa forma, a riqueza anã começou a ser toda gasta em uma espécie de guerra
contra inimigos poderosos e numerosos demais até um ponto que eles começaram a
minguar e não ter outra escolha a não ser fugir para outros pontos de Estália.
Hoje ainda existem lendas de anões morando em Malor e lutando bravamente contra
criaturas malignas que se apossaram da metrópole, mas os homens de Calífora.

Atualmente, os anões podem ser encontrados em várias das grandes cidades de
Estália, como Calífora, Nize, Argan, Vandelya... Mas ainda existem duas grandes
cidades verdadeiramente anãs que tentam ser o que Malor foi outrora. A primeira
delas é, na verdade, um conjunto de pequenas cidades muito próximas nas
montanhas de Galalio conhecidas como Minas Galalio. Elas são consideradas
território de Nize e os anões pagam tributos (em minérios e metais) para os
Saicro, mas fora isso, são praticamente independentes. Já a segunda cidade anã
fica mais ao sul, na cordilheira que divide a Península do Goto com o resto do
continente de Estália. Essa cidade é conhecida como Nova Malor e se tornou a
nova capital anã realmente independente. Entretanto, ela não é nem um décimo do
que fora outrora.

Nova Malor hoje também sobrevive principalmente exportando minerais precioso e
metais várias civilizações, mas grande parte de sua riqueza surgiu provendo
pedras para a construção de Calífora durante os últimos trezentos anos. Hoje,
entretanto, seu principal comércio é com os elfos de Valoar, na Floresta de
Argan, e também com os gnomos da Gadia, que hoje utilizam muito metal para suas
engenhocas.

\section{Gnomos}%

Os gnomos são criaturinhas minúsculas extremamente inquietas de corpo e
espírito. Eles são nativos da ilha da Gadia, no meio do Entreterras e poucos são
vistos fora de lá.

A verdade é que os gnomos são criaturas bastante misteriosas e incompreendidas
pelo resto das civilizações. Segundo eles próprios, os gnomos são numerosos e
habitam uma dimensão chamada de Bandos, de onde vem todo seu conhecimento
milenar sobre eletricidade e mecânica. Contudo, em Jǒna, poucos realmente
acreditam nessas histórias de erva-de-cheiro.

Entretanto, o que realmente todos sabem é que os gnomos são engenheiros e
cientistas natos. Sua inquietude de fato muitas vezes acaba os matando em
experimentos malucos envolvendo explosões ou qualquer outro tipo de coisa
perigosa, mas também acaba revelando tecnologias almejadas ou temidas por todos
os povos da Estália, como os balões e barcos voadores ou então os gigantes de
ferro que transformam miúdos gnomos em seres maiores que dois elfos.

A sorte dos outros povos é que os gnomos estão mais interessados em aventuras,
explorações e descobertas do que em guerra e, portanto, nunca utilizaram essas
tecnologias contra qualquer um dos povos livres.

Além dessas, outra tecnologia que chama bastante atenção dos não-gnomos é a
capacidade deles de criar seres mecânicos completamente conscientes. É o que
eles denominam automatos. Os automatos são praticamente uma raça por si só,
criada pelos gnomos, mas completamente livre de seus criadores. A opinião dos
demais sobre esses seres robóticos é segmentada entre aqueles que condenam e até
destruiriam esses robôs se não fosse pelo medo de uma rechaça gnoma, como os
elfos e halfings; e aqueles que acham fascinante e gostariam muito de saber
criar coisa parecida, como os humanos e anões. E não só com relação aos
automatos, mas os elfos vêm muitas das criações gnômicas como não-naturais e que
violam a ordem natural das coisas, enquanto que os humanos e anões, inventores
em menor escala, gostariam muito de descobrir os segredos dos gnomos.

Na Gadia, os gnomos não possuem realmente uma cidade ou algo parecido. Eles se
organizam em espécies de vilas com casas enormes (para padrões dos gnomos)
subterrâneas, mas, diferente dos anões, saem constantemente à luz do sol para
explorar a natureza ou fazer seus experimentos ao ar livre. Além disso, apesar
de serem parecidos com os haflings, que fazem as casas subterrâneas, todas as
outras construções dos gnomos são construídas para cima, como fazem os humanos e
elfos: mercados, oficinas, comércios, templos...

E por falar em templo, os gnomos são os detentores de um dos maiores templos de
toda Estália, senão o maior, conhecido como o Templo da Gadia, onde muitas
pessoas de outras raças vão em busca de conhecimento e elevação espiritual.
Todos aqueles que seguem a religião dos cinco deuses regentes possuem um apreço
por esse templo, pois seria onde supostamente os cinco deuses derrotaram Berkas,
o deus caído. Esse templo, então, é um território neutro e cuidado por
sacerdotes de várias raças diferentes. Para lá devem ir aqueles que querem se
tornar Goddion, a ordem religiosa que cultua os cinco deuses e espalha sua fé
por todo o continente.

\section{Automatos}%

\section{Elfos}%

\section{Halfings}%

    \end{multicols}

