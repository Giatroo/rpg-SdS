%!TeX root=../Rules.tex
\chapter{Regras Extras}%
\label{cha:regras_extras}
\begin{multicols}{2}

\section{Dinheiro}%
\label{sec:dinheiro}

As civilizações de Estália e Cafrilar dão mais valor à prata do que ao ouro. Por
conta disso, \textbf{peças de prata} são mais valiosas que \textbf{peças de
ouro}. Na prática, toda vez que você vir um preço em peças de ouro (po) em D\&D,
troque por peças de prata e vice-versa. Por exemplo, se um item custa $1$ pp em
D\&D, então ele custa $1$ po em SdS.

Qualquer item descrito nesse livro ou em outros recursos de SdS já estão nessa
nova convenção.

\section{dai'Nemai}%
\label{sec:dai'Nemai}

Do Taêno, \textit{Nemai} significa \textit{espírito} e o prefixo \textit{dai}
dá a caracterização de \textit{maligno}, portanto, esse termo pode ser bem
traduzido como \textit{Espírito do Mau}.

Essas criaturas malignas estão por toda parte, aparecendo em regiões sem
luminosidade e prontas para devorar seres humanos vivos e os levarem para o
Submundo, de onde elas veem. Desde os primórdios da humanidade, os humanos estão
acostumados a dormir com velas que iluminam seus quartos para não correrem o
risco dessas criaturas aparecerem. Ninguém sabe ao certo como elas funcionam, de
onde elas vieram e o que elas realmente querem. A única coisa que se sabe é que
eles aparecem em qualquer lugar que esteja escuro o suficiente e têm sede de
morte a qualquer ser humano.

Os seres humanos acreditam que essas criaturas são a última criação de Berkas, o
Deus das Sombras, antes de ser aprisionado dentro dos seres humanos. Os
Espíritos do Mau perseguem e querem matar todos os humanos para libertar Berkas.

\subsection{Aparecimento e Desaparecimento}%
\label{sub:aparecimento_e_desaparecimento}

No Mundo Natural, os Espíritos do Mau podem aparecer em qualquer região que seja
de \textbf{escuridão total} e consegue permanecer no mundo natural em qualquer
região de \textbf{escuridão total} ou \textbf{penumbra}, mas se entrar em
contato com \textbf{luz plena}, emitida por qualquer tipo de fonte de luz
(mágica ou não mágica), o Espírito desaparece.

\subsection{Fontes Extras de Luz}%
\label{sub:fontes_extras_de_luz}

Os itens que emitem luz presentes em D\&D são alterados e alguns outros existem:

\begin{center}
\begin{tabular}{ccc}
\textbf{Item} & \textbf{Custo} & \textbf{Peso} \\ \hline
Lâmpada & 1 pp & 0,5 kg \\
Lanterna coberta & 10 pp  & 1 kg \\
Lanterna furta-fogo & 20 pp & 1 kg \\
Lanterna tática & 50 pp & 1 kg \\
Óleo (frasco) & 1 pp & 0,5 kg \\
Óleo de emergência (frasco) & 5 po & 0,1 kg \\
Tocha & 2 pc & 0,5 kg \\
Vela & 2 pc & - \\
Poste de luz tático & 100 pp & 15 kg \\
\end{tabular}
\end{center}

\paragraph{Lâmpada.}%
\label{par:lampada}

Uma lâmpada lança luz plena em um raio de 3 metros e penumbra por mais 4,5
metros. Uma vez acesa, a lâmpada queima por 6 horas usando um frasco de óleo
(500 ml).

\paragraph{Lâmpada Coberta.}%
\label{par:lampada_coberta}

Uma lanterna coberta lança luz plena em um raio de 6 metros e penumbra por mais
6 metros. Uma vez acesa, ela queima por 6 horas usando um frasco de óleo (500
ml). Usando uma ação, você pode abaixar a cobertura, reduzindo a claridade para
penumbra em um raio de 1,5 metro.

\paragraph{Lanterna Furta-Fogo.}%
\label{par:lanterna_furta_fogo}

Uma lanterna furta-fogo lança luz plena em um cone de 12 metros e penumbra por
mais 12 metros. Uma vez acesa, ela queima por 6 horas usando um frasco de óleo
(500 ml).

\paragraph{Lânterna Tática.}%
\label{par:lanterna_tatica}

Uma lanterna tática lança luz plena em um raio de 9 metros e penumbra por mais 9
metros. Uma vez acesa, ela queima por 6 horas usando um frasco de óleo (500 ml).
Usando uma ação, mudar seu formato para emitir luz plena em um cone de 18 metros
e penumbra por mais 18 metros. Uma lânterna tática possui dois compartimentos
para frascos de óleo de emergência, podendo acender a lânterna com eles
utilizando uma ação bônus.

\paragraph{Óleo de Emergência.}%
\label{par:oleo_de_emergencia}

Um óleo de emergência é um óleo impuro, que queima três vezes mais rápido que
um frasco de óleo normal. Além disso, ele não consegue ser utilizado em balhata
como um óleo comum (pois não produz chamas em contato com dano de fogo).

\paragraph{Poste de Luz Tático.}%
\label{par:poste_de_luz_tatico}

Um poste de luz tático pode ser levado em operações e fixado no chão para
iluminar uma área em volta. Esse poste pode ser dobrado e reduzido à altura de
apenas 1m, para ser carregado. E pode ser desdobrado e fixado ao chão me uma
região de 1,5m por 1,5m, tornando essa região área de deslocamento difícil.
Depois de fixado, o poste pode ser levantado até 5 metros de altura. A 3 metros
de altura, ele lança luz plena em um raio de 9 metros e penumbra por mais 9
metros, e a 5 metros, ele lança luz plena em um raio de 6 metros e penumbra por
mais 15 metros. Para fixá-lo no chão, é preciso estar em duas pessoas e leva 10
minutos. Para levantar até 3 metros, leva mais 2 minutos e para levantar até 5,
mais 1 minuto. Uma vez aceso, o poste queima por 6 horas usando um frasco de
óleo (500 ml).

\section{Equipamentos Orientais}%
\label{sec:equipamentos_orientais}

Para efeitos de \textit{role playing}, abaixo segue uma tabela que converte as
armas convencionais de D\&D em armas orientais. Essas armas podem ser
utilizadas, por exemplo, por Monges (ver \ref{cha:monge}) ou qualquer classe.

\begin{center}
\begin{tabular}{c | c}
\textbf{Arma original} & \textbf{Arma oriental} \\ \hline
Foice Curta & Kama ou Leque de Guerra \\
Lança & Naginata \\
Machadinha & Kunai \\
Porrete & Bastão \\
Dardo & Shuriken \\
Chicote & Kusarigama ou Nunchaku \\
Espada Curta & Katana \\
\end{tabular}
\end{center}

\end{multicols}
