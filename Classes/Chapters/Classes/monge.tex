%!TeX root=../../rules.tex
\chapter{Monge}%
\label{cha:monge}
\begin{multicols}{2}

\section*{Características de Classe}%

Como um monge, você adquire as seguintes características de classe.

\subsubsection{Pontos de Vida}%

\noindent\textbf{Dado de Vida}: 1d8 por nível de monge \nl
\textbf{Pontos de Vida no 1º Nível:} 8 + seu modificador de Constituição. \nl
\textbf{Pontos de Vida nos Níveis Seguintes:} 1d8 (ou 5) + seu modificador de
Constituição por nível de monge após o 1º.

\subsubsection{Proficiências}%

\noindent\textbf{Armaduras:} Nenhuma \nl
\textbf{Armas:} Armas simples, espadas curtas \nl
\textbf{Ferramentas:} Escolha um tipo de ferramenta de artesão ou um instrumento
musical \jump
\textbf{Testes de Resistência}: Força, Destreza \nl
\textbf{Pericias:} Escolha dois dentre Acrobacia, Atletismo, Furtividade,
História, Intuição e Religião

\subsubsection{Equipamento}%

Você começa com o seguinte equipamento, além do equipamento concedido pelo seu
antecedente.
\begin{itemize}
    \item (a) uma espada curta ou (b) qualquer arma simples
    \item (a) um pacote de explorador ou (b) um pacote de aventureiro
    \item 10 dardos
\end{itemize}

\section*{Conjuração}%

A tabela a seguir mostra em quais níveis o monge tem acesso às magias de cada
ciclo:

\begin{center}
\begin{tabular}{|||c||c|||}
    \hline
    \textbf{Nível} & \textbf{Ciclo} \\
    \hline
    1 & 1º \\
    \hline
    3 & 2º \\
    \hline
    5 & 3º \\
    \hline
    7 & 4º \\
    \hline
    9 & 5º \\
    \hline
    11 & 6º \\
    \hline
    13 & 7º \\
    \hline
    15 & 8º \\
    \hline
    17 & 9º \\
    \hline
\end{tabular}
\end{center}

\subsubsection*{Habilidade de Conjuração}%

O maior entre sua Destreza e Sabedoria é a sua habilidade para você conjurar
suas magias de monge, pois os monges são um conjunto de velocidade, impacto e
paciência. Você usa o máximo entre Destreza e Sabedoria sempre que alguma magia
se referir a sua habilidade de conjurar magias. Além disso, você usa o máximo do
modificador de Destreza e Sabedoria para definir a CD dos testes de resistência
para as magias de monge que você conjura e quando você realiza uma jogada de
ataque com uma magia.

\begin{center}
\textbf{CD para suas magias} = 8 + bônus de proficiência + seu modificador de
Destreza ou Sabedoria (o maior dentro eles) \nl

\textbf{Modificador de ataque de magia} = seu bônus de proficiência + seu
modificador de Destreza ou Sabedoria (o maior dentro eles)
\end{center}

\section*{Defesa sem Armadura}%

A partir do 1º nível, quando você não estiver vestindo nenhuma armadura nem
empunhando um escudo, sua Classe de Armadura será 10 + seu modificador de
Destreza + seu modificador de Sabedoria.

\section*{Artes Marciais}%

No 1º nível, sua prática nas artes marciais concede a você maestria nos estilos
de combate que utilizam golpes desarmados e armas de monge, que são as espadas
curtas e quaisquer armas simples corpo-a-corpo que não tenham a propriedade duas
mãos ou pesada.

Você ganha os seguintes benefícios enquanto estiver desarmado ou empunhando uma
arma de monge e não estiver vestindo nenhuma armadura ou empunhando um escudo:

\begin{itemize}
    \item Você pode usar Destreza ao invés de Força para as jogadas de ataque e
        dano dos seus golpes desarmados e de suas armas de monge.
    \item Você pode rolar um d4 no lugar do dano normal dos seus golpes
        desarmados e armas de monge. Esse dado muda à medida que você adquire
        níveis de monge, tornando-se um d6 no 6º nível, um d8 no 11º nível e um
        d10 no 17º nível.
    \item Quando você usa a ação de Ataque com um golpe desarmado ou uma arma de
        monge no seu turno, você pode realizar um golpe desarmado com uma ação
        bônus. Por exemplo, se você realizar a ação de Ataque com um bastão,
        você também poderá realizar um golpe desarmado com uma ação bônus,
        assumindo que você ainda não realizou uma ação bônus nesse turno.
\end{itemize}

Determinados monastérios usam formas especializadas de armas de monge. Abaixo
segue uma tabela com recomendações de adaptações das armas originais para armas
orientais.

\begin{center}
\begin{tabular}{c | c}
\textbf{Arma original} & \textbf{Arma oriental} \\ \hline
Foice Curta & Kama ou Leque de Guerra \\
Lança & Naginata \\
Machadinha & Kunai \\
Porrete & Bastão \\
Dardo & Shuriken \\
Chicote & Kusarigama ou Nunchaku \\
Espada Curta & Katana \\
\end{tabular}
\end{center}

\paragraph{Observação.}%

Outras classes podem também fazer uso dessas adaptações, se desejarem.

\section*{Conexão com o Plano Astral}%
\label{sec:conexao_com_o_plano_astral}

A partir do 2º nível, seu treinamento permitiu que você controlasse a energia
mística de seu plano astral. Seu acesso a esse plano é representado por um
número de pontos de chi. Seu nível de monge determina o número de pontos que
você tem, como mostrado na tabela O Monge.

Você pode gastar esses pontos para abastecer várias características de chi. Você
começa conhecendo três dessas características: Rajada de Golpes, Defesa Paciente
e Passo do Vento. Você aprende mais características de chi à medida que adquire
níveis nessa classe.

Quando você gasta um ponto de chi, ele se torna indisponível até você terminar
um descanso curto ou longo, no fim deste, todos os pontos de chi gastos volta
para você. Você deve gastar, pelo menos, 30 minutos do descanso meditando para
recuperar seus pontos de chi.

Algumas das características de chi requerem que seu alvo realize um teste de
resistência para resistir ao efeito da característica. A CD do teste de
resistência é calculada a segui:

\begin{center}
\textbf{CD de resistência de Chi} = 8 + bônus de proficiência +
seu modificador de Sabedoria
\end{center}

\subsubsection*{Rajada de Golpes}%
\label{ssub:rajada_de_golpes}

Imediatamente após você realizar a ação de Ataque no seu turno, você pode gastar
1 ponto de chi para realizar dois golpes desarmados com uma ação bônus.

\subsubsection{Defesa Paciente}%
\label{ssub:defesa_paciente}

Você pode gastar 1 ponto de chi para realizar a ação de Esquivar, com uma ação
bônus, no seu turno.

\subsubsection*{Passo do Vento}%
\label{ssub:passo_do_vento}

Você pode gastar 1 ponto de chi para realizar a Ação de Desengajar ou Disparada,
com uma ação bônus, no seu turno, e sua distância de salto é dobrada nesse
turno.

\section*{Movimento sem Armadura}%
\label{sec:movimento_sem_armadura}

A partir do 2° nível, seu deslocamento aumenta em 3 metros enquanto você não
estiver usando armadura nem empunhando um escudo. Esse bônus aumenta quando você
alcança determinados níveis, como mostrado na tabela O Monge.

No 9° nível, você ganha a habilidade de se mover através de superfícies
verticais e sobre líquidos, no seu turno, sem cair durante o movimento.

\section*{Tradição Monástica}%
\label{sec:tradicao_monastica}

Quando você alcança o 3° nível, você ingressa numa tradição monástica: o Caminho
da Mão Aberta, o Caminho Sombrio e o Caminho dos Quatro Elementos, todas
detalhadas no final da descrição dessa classe. Sua tradição concede a você
características no 3° nível e novamente no 6°, 11° e 17° nível.

\section*{Defletir Projéteis}%
\label{sec:defletir_projeteis}

A partir do 3° nível, você pode usar sua reação para defletir ou apanhar o
projétil quando você é atingido por um ataque de arma à distância. Quando o
fizer, o dano que você sofrer do ataque é reduzido em 1d10 + seu modificador de
Destreza + seu nível de monge.

Se o dano for reduzido a 0, você pode apanhar o projétil se ele for pequeno o
suficiente para ser segurando em uma mão e você tenha, pelo menos, uma mão
livre. Se você apanhar um projétil dessa forma, você pode gastar 1 ponto de chi
para realizar uma ataque à distância com a arma ou munição que você acabou de
pegar, como parte da mesma reação. Você realiza esse ataque com proficiência,
independentemente das armas em que você é proficiente, e o projétil conta como
uma arma de monge para o ataque. A distância do ataque do monge é de 6/18
metros.

\section*{Incremento em Habilidade}%
\label{sec:incremento_em_habilidade}

Quando você atinge o 4° nível e novamente no 8°, 12°, 16° e 19° nível, você pode
aumentar um valor de habilidade, à sua escolha, em 2 ou você pode aumentar dois
valores de habilidade, à sua escolha, em 1. Como padrão, você não pode elevar um
valor de habilidade acima de 20 com essa característica.

\section*{Queda Lenta}%
\label{sec:queda_lenta}

Começando no 4° nível, você pode usar sua reação, quando você cai, para reduzir
o dano de queda sofrido por um valor igual a cinco vezes seu nível de monge.

\section*{Ataque Extra}%
\label{sec:ataque_extra}

A partir do 5° nível, você pode atacar duas vezes, ao invés de uma, sempre que
você realizar a ação de Ataque no seu turno.

\section*{Ataque Atordoante}%
\label{sec:ataque_atordoante}

A partir do 5° nível, você pode interferir no fluxo de chi do corpo de um
oponente. Quando você atingir outra criatura com um ataque corpo-a-corpo com
arma, você pode gastar 1 ponto de chi para tentar um ataque atordoante. O alvo
deve ser bem sucedido num teste de resistência de Constituição ou ficará
atordoado até o final do seu próximo turno.

\section*{Golpes de Chi}%
\label{sec:golpes_de_chi}

A partir do 6° nível, seus golpes desarmados contam como armas mágicas com o
propósito de ultrapassar a resistência ou imunidade a ataque e danos
não-mágicos.

\section*{Evasão}%
\label{sec:evasao}

A partir do 7º nível, você pode esquivar-se agilmente de certos efeitos em área,
como o sopro elétrico de um dragão azul ou uma magia \textit{bola de fogo}.
Quando você for alvo de um efeito que exige um teste de resistência de Destreza
para sofrer metade do dano, você não sofre dano algum se passar, e somente
metade do dano se falhar.

\section*{Mente Tranquila}%
\label{sec:mente_tranquila}

A partir do 7° nível, você pode usar sua ação para terminar um efeito em si
mesmo, que esteja lhe enfeitiçando ou amedrontando.

\section*{Pureza Corporal}%
\label{sec:pureza_corporal}

No 10° nível, sua maestria do chi flui através de você, tornando-o imune a
doenças e venenos.

\section*{Idiomas do Sol e da Lua}%
\label{sec:idiomas_do_sol_e_da_lua}

A partir do 13° nível, você aprende a tocar o chi de outras mentes fazendo com
que você compreenda todos os idiomas falados. Além do mais, qualquer criatura
que possa entender um idioma poderá entender o que você fala.

\section*{Alma de Diamante}%
\label{sec:alma_de_diamante}

A partir do 14° nível, sua maestria do chi concede a você proficiência em todos
os testes de resistência. Além disso, toda vez que você realizar um teste de
resistência e falha, você pode gastar 1 ponto de chi para jogar novamente e
ficar com o segundo resultado.

\section*{Corpo Atemporal}%
\label{sec:corpo_atemporal}

No 15° nível, seu chi sustenta você tanto que você não sofre os efeitos da
velhice e não pode envelhecer magicamente. Você ainda morrerá de velhice, no
entanto. Além disso, você não precisa mais de comida ou água.

\section*{Corpo Vazio}%
\label{sec:corpo_vazio}

A partir do 18° nível, você pode usar sua ação para gastar 4 pontos de chi e
ficar invisível por 1 minuto. Durante esse tempo, você também adquire
resistência a todos os danos, exceto dano de energia.

Além disso, você pode gastar 8 pontos de chi para conjurar a magia
\textit{projeção astral}, sem precisar de componentes materiais. Quando o fizer,
você não pode levar qualquer outra criatura com você.

\section*{Auto Aperfeiçoamento}%
\label{sec:auto_aperfeicoamento}

No 20° nível, quando você rolar iniciativa e não tiver nenhum ponto de chi
restante, você recupera 4 pontos de chi.

\section*{Tradições Monásticas}%
\label{sec:tradicoes_monasticas}

Tradições de busca monástica são comuns nos monastérios espalhados pelo
multiverso. A maioria dos monastérios pratica exclusivamente uma tradição, mas
alguns poucos honram as todas tradições e instruem cada monge, de acordo com
suas aptidões e interesses. Todas as tradições compartilha as mesmas técnicas
básicas, divergindo à medida que o estudante se torna mais adepto. Portanto, um
monge precisa escolher uma tradição apenas quando alcançar o 3° nível.

\subsection*{Caminho da Mão Aberta}%
\label{sub:caminho_da_mao_aberta}

Monges do Caminho da Mão Aberta são os mestres supremos das artes de combate
marcial, tanto armado quanto desarmado. Eles aprendem técnicas para empurrar e
derrubar seus oponentes, manipulando o chi para curar os ferimentos dos seus
corpos e praticando uma meditação avançada que os protege de mazelas.

\subsubsection*{Técnica da Mão Aberta}%
\label{ssub:tecnica_da_mao_aberta}

Começando quando você escolhe essa tradição, no 3° nível, você pode manipular o
chi do seu inimigo quando você controla o seu. Toda vez que você atingir uma
criatura com um dos seus ataques garantidos por sua Rajada de Golpes, você pode
impor um dos seguintes efeitos no alvo:

\begin{itemize}
    \item Ele deve ser bem sucedido num teste de resistência de Destreza ou
        cairá no chão.
    \item Ele deve realizar um teste de resistência de Força. Se falhar, você
        pode empurrá-lo 4,5 metros para longe de você.
    \item Ele não pode realizar reações até o final do seu próximo turno.
\end{itemize}

\subsubsection*{Integridade Corporal}%
\label{ssub:integridade_corporal}

No 6° nível, você ganha a habilidade de se curar. Com uma ação, você recupera
uma quantidade de pontos de vida igual a três vezes seu nível de monge. Você
deve terminar um descanso longo antes de poder usar essa característica
novamente.

\subsubsection*{Tranquilidade}%
\label{ssub:tranquilidade}

A partir do 11° nível, você pode entrar num estado especial de meditação que
rodeia você com uma aura pacifica. No final de um descanso longo, você ganha o
efeito da magia santuário que dura até o começo do seu próximo descanso longo (a
magia pode terminar prematuramente, como normal). A CD do teste de resistência
para a magia é 8 + seu modificador de Sabedoria + seu bônus de proficiência.

\subsubsection*{Palma Vibrante}%
\label{ssub:palma_vibrante}

No 17° nível, você ganha a habilidade de criar vibrações letais no corpo de
alguém. Quando você atingir uma criatura com um golpe desarmado, você pode
gastar 3 pontos de chi para começar essas vibrações imperceptíveis, que duram
por um número de dias igual ao seu nível de monge. As vibrações são inofensivas,
a não ser que você use sua ação para terminá-las. Para tanto, você e o alvo
devem estar no mesmo plano de existência. Quando você usa essa ação, a criatura
deve realizar um teste de resistência de Constituição. Se ela falhar, ela será
reduzida a 0 pontos de vida. Se ela passar, ela sofrerá 10d10 de dano
necrótico.

Você pode ter apenas uma criatura sob o efeito dessa característica por vez.
Você pode escolher terminar as vibrações inofensivamente, sem usar uma ação.

\subsection*{Caminho da Sombra}%
\label{sub:caminho_da_sombra}

Monges do Caminho da Sombra seguem uma tradição que valoriza furtividade e
subterfugio. Esses monges devem ser chamados de ninjas ou dançarinos das sombras
e eles servem como espiões e assassinos. Às vezes, os membros de um monastério
ninja são membros da mesma família, formando um clã que jurou sigilo sobre suas
artes e missões. Outros monastérios parecem mais com guildas de ladrões,
oferecendo seus serviços a nobres, mercadores ricos ou qualquer um que possa
pagar suas taxas. Independente dos seus métodos, os líderes desses monastérios
esperam obediência inquestionável de seus estudantes.

\subsubsection*{Artes Sombrias}%
\label{ssub:artes_sombrias}

Começando quando você escolhe essa tradição, no 3° nível, você pode usar seu chi
para simular o efeito de certas magias. Com uma ação, você pode gastar 2 pontos
de chi para conjurar escuridão, visão no escuro, passos sem pegadas ou silêncio,
sem precisar de componentes materiais. Além disso, você ganha o truque ilusão
menor, se você ainda não o conhecia.

\subsubsection*{Passo das Sombras}%
\label{ssub:passo_das_sombras}

No 6° nível, você ganha a habilidade de entrar em uma sombra e sair em outra.
Quando você estiver sob penumbra ou na escuridão, com uma ação bônus, você pode
se teletransportar a até 18 metros para um espaço desocupado que você possa ver
que também esteja sob penumbra ou escuridão. Você, então, terá vantagem no
primeiro ataque corpo-a-corpo que você fizer antes do final do seu turno.

\subsubsection*{Manto de Sombras}%
\label{ssub:manto_de_sombras}

No 11° nível, você aprendeu a se tornar uno com as sombras. Quando você estiver
em uma área de penumbra ou escuridão, você pode usar sua ação para se tornar
invisível. Você permanece invisível até realizar um ataque, conjurar uma magia
ou se for para uma área de bem iluminada.

\subsubsection*{Oportunista}%
\label{ssub:oportunista}

No 17° nível, você pode explorar um momento de distração de uma criatura quando
ela é atingida por um ataque. Toda vez que uma criatura a até 1,5 metro de você
for atingida por um ataque realizar por outra criatura diferente de você, você
pode usar sua reação para realizar um ataque corpo-a-corpo contra essa
criatura.

\subsection*{Estilo do Mestre Bêbado}%
\label{sub:estilo_do_mestre_bebado}

O Estilo do Mestre Bêbado ensina seus alunos a se moverem ridiculamente, com
movimentos imprevisíveis de um bêbado. Um mestre bêbado balança, cambaleante em
seus passos desequilibrados, apresentando o que parece ser um incompetente
combatente que está insatisfeito em lutar. Os movimentos erráticos do mestre
bêbado escondem uma dança cuidadosamente executada repleta de bloqueios,
esquivas, avanços, ataques e recuos.

Um mestre bêbado muitas vezes gosta de bancar o idiota para trazer alegria aos
desanimados ou demonstrar humildade para o arrogante, mas quando entra em
batalha, o mestre bêbado pode ser um inimigo magistral.

\subsubsection*{Proficiências Bônus}%
\label{ssub:proficiencias_bonus}

Quando você escolhe essa tradição no 3º nível, você recebe proficiência na
perícia atuação se você não tiver. Suas técnicas de artes marciais misturam
treinamento em combate com a precisão de um dançarino e a excentricidade de um
palhaço, Você também ganha proficiência para usar suprimento do cervejeiro se
você não tiver.

\subsubsection*{Técnica Bêbada}%
\label{ssub:tecnica_bebada}

No 3º nível você aprende como girar e desviar rapidamente como parte da sua
Rajada de Golpes. Quando usar sua Rajada de Golpes, você ganha o beneficio da
ação Desengajar e seu deslocamento é aumentado em 3 metros até o final do turno.

\subsubsection*{Balanço Bêbado}%
\label{ssub:balanco_bebado}

No 6º nível você pode, repentinamente, mover-se de forma cambaleante. Você ganha
os seguintes benefícios.

\paragraph{Levantar do Chão.}%
Quando você estiver caído poderá levantar-se usando 3 metros de seu
deslocamento, no lugar de usar metade do deslocamento.

\paragraph{Redirecionar Ataque.}%
Quando uma criatura errar um ataque corpo-a-corpo contra você, você poderá
gastar 1 ponto de chi como reação para fazer com que esse ataque acerte outra
criatura à sua escolha, desde que você possa ver essa criatura e ela esteja a
1,5 metros de você.

\subsubsection*{Sorte do Bêbado}%
\label{ssub:sorte_do_bebado}

A partir do 11º nível, você parece ter uma maré de sorte no momento certo.
Quando fizer um teste de habilidade, um ataque ou um teste de resistência e
tiver desvantagem nas jogadas, você pode usar 2 pontos de chi para cancelar a
desvantagem na jogada.

\subsubsection*{Frenesi do Bêbado}%
\label{ssub:frenesi_do_bebado}

No 17º nível você ganha a habilidade de fazer um número de ataques esmagadores
contra um grupo de inimigos. Quando usar sua Rajada de Golpes, você pode fazer
até três ataques adicionais (totalizando cinco ataques da Rajada de Golpes),
desde que cada ataque da Rajada de Golpes atinja uma criatura diferente nesse
turno.

\subsection*{Estilo do Kensei}%
\label{sub:estilo_do_kensei}

Monges do Estilo do Kensei treinam incansavelmente com suas armas, ao ponto em
que a arma se tora uma extensão do corpo. Estabelecida em dominar lutas com
espadas, a tradição se expandiu, incluindo diferentes armas.

Um Kensei vê uma arma da mesma forma que um caligrafo ou pintor aprecia a caneta
ou o pincel. Não importa a arma, o Kensei a vê como uma ferramenta usada para
expressar a beleza e precisão das artes marciais. Esse nível de domínio faz de
um Kensei um inigualável guerreiro, sendo esse o resultado de uma intensa
devoção, prática e estudo.

\subsubsection*{Caminho do Kensei}%
\label{ssub:caminho_do_kensei}

Quando você escolhe essa tradição no 3º nível, seu treinamento em artes marciais
o leva a dominar o uso de certo tipo de armas. Esse caminho também inclui
instruções na fina arte da caligrafia ou pintura. Você ganha os seguintes
benefícios.

\paragraph{Armas Kensei.}%
Escolha dois tipos de armas para serem suas armas Kensei: Uma arma corpo-a-corpo
e outro arma a distância. As armas podem ser armas simples ou armas marciais que
não tenham as propriedades pesada e especial. O arco longo é uma escolha válida.
Você ganha proficiência com essas armas caso não tenha. As armas escolhidas são
armas de monge para você. Muitas características dessa tradição funcionam
somente com suas armas Kensei. Quando alcançar os níveis 6º, 11º e 17º na
classe de monge, você poderá escolher outro tipo de arma, a distância ou
corpo-a-corpo, para ser uma arma Kensei para você, seguindo os critérios acima.

\paragraph{Esquiva Rápida.}%
Se você fizer um ataque desarmado, como parte da sua ação de ataque, e estiver
segurando uma Arma Kensei, você pode usá-la para se defender (desde que seja uma
arma corpo-a-corpo). Enquanto a arma estiver em sua mão e você não estiver
incapacitado, você ganha um bônus de +2 na CA até o começo do próximo turno.

\paragraph{Tiro do Kensei.}%
Você pode usar sua ação bônus, no seu turno, para fazer seus ataques à distância
com uma Arma Kensei mais mortais. Quando você usa essa habilidade, qualquer alvo
que você atingir usando uma Arma Kensei a distância aplica dano adicional de 1d4
do tipo da arma escolhida. Você mantém esse beneficio até o final do seu turno
atual.

\paragraph{Caminho do Pincel.}%
Você ganha proficiência ou com Suprimentos de Caligrafia ou com Ferramentas de
Pintor, escolha um.

\subsubsection*{Uno com a Lâmina}%
\label{ssub:uno_com_a_lamina}

No 6º Nível você imbui seu chi para suas Armas Kensei, garantindo os seguintes
benefícios.

\paragraph{Armas Kensei Mágicas.}%
Seus ataques com Armas Kensei contam como armas mágicas com o propósito de
ultrapassar a resistência ou imunidade a ataques e danos não-mágicos.

\paragraph{Golpe Habilidoso}%
Quando atingir um alvo com uma Arma Kensei, você pode gastar 1 ponto de chi para
que a Arma causa dano extra igual ao seu dado de Artes Marciais. Você pode usar
essa característica somente uma vez em cada um dos seus turnos.

\subsubsection*{Aguçar a Lâmina}%
\label{ssub:agucar_a_lamina}

No 11º nível, você ganha a habilidade de melhorar suas armas utilizando seu chi.
Com uma ação bônus você pode gastar até 3 pontos de chi para dar a uma Arma
Kensei que você toque um bônus nas rolagens de ataque e dano. O bônus é igual
ao número de pontos de chi que você gastou. Esse bônus dura 1 minuto ou até você
usar essa característica novamente. Essa característica não tem efeito em Armas
Mágicas que já tenham bônus nos testes de ataque e dano.

\subsubsection*{Precisão Infalível}%
\label{ssub:precisao_infalivel}

No 17º nível, seu domínio com armas garante a você uma extraordinária precisão.
Quando errar um ataque usando uma arma de monge no seu turno, você pode atacar
novamente. Essa característica só poderá ser usada uma vez em cada um dos seus
turnos.

\subsection*{Estilo da Alma Solar}%
\label{sub:estilo_da_alma_solar}

Monges do Estilo da Alma Solar aprendem a canalizar sua energia vital em
calcinantes raios solares. Apenas monges da Ordem da Luz conseguem trilhar esse
caminho.

\subsubsection{Raio Solar Radiante}%
\label{ssub:raio_solar_radiante}

Quando escolher essa tradição no 3º nível, você pode lançar raios mágicos
radiantes.

Você ganha uma nova opção de ataque que pode ser usada junto com a ação de
Ataque. Esse ataque especial é um ataque à distância com alcance de 9 metros.
Você é proficiente com ele e você adiciona seu modificador de Destreza nos
testes de ataque e dano. O dano é do tipo radiante e o dado de dano é 1d4. Esse
dado se altera conforme você ganha níveis de monge, de acordo com a coluna de
Artes Marciais da tabela de monge.

Quando usar a ação de Ataque no seu turno e utilizar esse ataque especial como
parte dele, você pode gastar 1 ponto de chi para fazer esse ataque especial duas
vezes, com uma ação bônus.

Quando ganhar a habilidade de ataque extra, esse Ataque Especial pode ser usado
em qualquer um dos ataques que você fizer como parte da sua ação de Ataque.

\subsubsection*{Golpe do Arco Abrasador}%
\label{ssub:golpe_do_arco_abrasador}

No 6º nível, você ganha a habilidade de canalizar seu chi em abrasadoras ondas
de energia. Imediatamente após ter feito a ação de ataque no seu turno, você
pode gastar 2 pontos de chi para conjurar a magia mãos flamejantes como ação
bônus. Cada ponto adicional de chi gasto aumenta o nível da magia mãos
flamejantes em 1. O número máximo de pontos de chi (2 mais quaisquer pontos
adicionais) que você pode gastar na magia é igual a metade do seu nível de
monge.

\subsubsection{Explosão Calcinante}%
\label{ssub:explosao_calcinante}

No 11º nível, você ganha a habilidade de criar um orbe de luz que entra em
erupção causando uma explosão devastadora. Com uma ação, você magicamente cria
um orbe e o lança em um ponto, a sua escolha dentro de 45 metros, que rompe em
uma esfera de luz radiante por um breve, mas mortal momento.

Cada criatura em uma esfera de 8 metros de raio deve ser bem sucedida num teste
de resistência em Constituição ou sofrerá 2d6 de dano radiante. Uma criatura não
precisa fazer o teste caso esteja em cobertura total que seja opaca. Você pode
aumentar o dano da esfera gastando pontos de chi. Cada ponto gasto, até o máximo
de 3, aumenta o dano em 2d6.

\subsubsection*{Escudo Solar}%
\label{ssub:escudo_solar}

No 17º nível, você é rodeado por uma aura mágica luminosa. Você espalha luz
plena em um raio de 9 metros e penumbra por mais 9 metros. Você pode extinguir
ou restaurar a luz como uma ação bônus.

Se uma criatura o atingir quando sua luz brilhar, você pode usar sua reação para
causar dano radiante a criatura, o dano radiante é igual a 5 mais seu
modificador de Sabedoria.

\subsection*{Caminho da Misericórdia}%
\label{sub:caminho_da_misericordia}

Monges do Caminho da Misericórdia aprendem a manipular a força vital dos outros
para auxiliar aqueles em necessidade. Eles são médicos ambulantes para os pobres
e feridos. Contudo, para aqueles além de sua ajuda, eles trazem um fim rápido,
como um ato de misericórdia.

Aqueles que seguem o Caminho da Misericórdia podem ser membros de uma ordem
religiosa, administrada a favor dos necessitados e fazendo escolhas amargas
baseadas na realidade em vez de no idealismo. Alguns podem ser curadores de fala
mansa, amados por seus companheiros, enquanto outros podem ser mascarados
portadores de uma macabra misericórdia.

Os caminhantes desta senda normalmente vestem mantos com capuzes pesados, e
frequentemente escondem seus rostos com uma máscara, apresentando a si mesmos
como portadores sem face da vida e da morte.

\subsubsection*{Implementos da Misericórdia}%
\label{ssub:implementos_da_misericordia}

No 3º nível, você adquire proficiência nas perícias de Intuição e Medicina, além
de ganhar proficiência com o kit de herbalismo. Você também recebe uma máscara
especial, que você frequentemente veste quando está utilizando as
características dessa subclasse. Você determina sua aparência, ou a gera
aleatoriamente jogando na tabela de Máscara Misericordiosa. Exemplos de máscara
são de corvo, preta e branca, rosto chorando, rosto gargalhando, caveira,
borboleta.

\subsubsection{Mãos Curativas}%
\label{ssub:maos_curativas}

No 3º nível, seu toque místico pode curar ferimentos. Como uma ação, você pode
gastar 1 ponto de chi para tocar uma criatura e restaurar uma quantia de pontos
de vida igual ao resultado da jogada de um dado de Artes Marciais + seu
modificador de Sabedoria. Quando você utilizar sua torrente de golpes, você pode
substituir um dos ataques desarmados por um uso dessa característica sem gastar
um ponto de chi pela cura.

\subsubsection*{Mãos da Injúria}%
\label{ssub:maos_da_injuria}

No 3º nível, você pode usar o seu chi para causar ferimentos. Quando você acerta
uma criatura com um ataque desarmado, você pode gastar 1 ponto de chi para
causar dano necrótico extra igual a uma jogada do seu dado de Artes Marciais +
seu modificador de Sabedoria. Você pode utilizar essa característica apenas uma
vez por turno.

\subsubsection*{Toque do Curandeiro}%
\label{ssub:toque_do_curandeiro}

No 6º nível, você pode administrar até mesmo grandes curas com seu toque e, se
achar necessário, você pode usar seu conhecimento para causar dano.

Quando você utilizar suas Mãos curativas em uma criatura, você também pode
remover uma doença ou uma das seguintes condições que esteja afetando a
criatura: atordoado, cego, ensurdecido, envenenado ou paralisado.

Quando você utilizar suas Mãos da injúria em uma criatura, você pode sujeitar o
alvo à condição de envenenado até o final do seu próximo turno.

\subsubsection*{Torrente de Cura e Dor}%
\label{ssub:torrente_de_cura_e_dor}

No 11º nível, você agora pode propagar uma torrente de cura e dor. Quando você
usar sua Torrente de golpes, você agora pode substituir cada um de seus ataques
desarmados por um uso de suas mãos curativas sem gastar pontos de chi para
curar.

Adicionalmente, quando você realizar um ataque desarmado com a Torrente de
golpes, você pode utilizar a característica Mãos da injúria com esse golpes sem
gastar o ponto de chi por ela. Você ainda pode utilizar a Mãos da injúria apenas
uma vez por turno.

\subsubsection*{Mão da Misericórdia Final}%
\label{ssub:mao_da_misericordia_final}

No 17º nível, seu domínio sobre a energia da vital abriu a porta para a
misericórdia final. Como uma ação, você pode tocar o corpo de uma criatura que
morreu dentro das últimas 24 horas e gastar 5 pontos de chi. A criatura então
retorna à vida, recuperando uma quantia de pontos de vida igual a 4d10 + seu
modificador de Sabedoria. Se a criatura morreu enquanto sob o efeito de uma
dessas condições, ela a terá removida ao ser trazida de volta: cega,
ensurdecida, paralisada, envenenada ou atordoada.

Uma vez que use essa característica, você não poderá fazê-lo novamente até que
termine um descanso longo.

\subsubsection*{Caminho da Forma Astral}%
\label{ssub:caminho_da_forma_astral}

Um monge que segue o Caminho da Forma Astral acredita que seu corpo é uma
ilusão. Eles enxergam seu chi como a representação de sua forma verdadeira, sua
forma astral. Essa forma astral possui a capacidade de ser uma força da ordem ou
da desordem, com alguns monastérios treinando estudantes para usarem seus
poderes para proteger os fracos e outros instruindo aspirantes em como
manifestar seu eu verdadeiro em serviço do poder.

\subsubsection*{Braços da Forma Astral}%
\label{ssub:bracos_da_forma_astral}

Seu domínio sobre o chi permite que você manifeste uma parte de sua forma
astral. Como uma ação bônus, você pode gastar 1 ponto de chi para invocar os
braços de sua forma astral. Ao fazê-lo, cada criatura a sua escolha que você
possa ver a até 3m de você deve ser bem-sucedida em uma salvaguarda de Destreza
ou sofrerá dano de energia equivalentes à rolagem de dois dados de Artes
Marciais.

Durante 10 minutos, esses braços espectrais pairam perto de seus ombros ou
cercam seus braços (conforme desejar). Você determina a aparência desses braços,
e eles desaparecem assim que você for incapacitado ou morto.

Enquanto os braços espectrais estiverem presentes, você ganha os seguintes
benefícios:

\begin{itemize}
    \item Você pode utilizar o seu modificador de Sabedoria em vez do
        modificador de força para testes de habilidade e salvaguardas de Força.
    \item Você pode utilizar os braços espectrais para realizar ataques
        desarmados.
    \item Quando realizar um ataque desarmado com esses braços em seu turno, seu
        alcance para eles é superior em 1,5m ao seu alcance normal.
    \item Os ataques desarmados que você faz com os braços espectrais podem
        utilizar seu modificador de Sabedoria em vez do modificador de Força ou
        Destreza para as rolagens de ataque e de dano, e seu tipo de dano é de
        energia.
\end{itemize}

\subsubsection*{Semblante da Forma Astral}%
\label{ssub:semblante_da_forma_astral}

No 6º nível, você pode invocar o semblante de sua forma astral. Como uma ação
bônus, ou como parte da ação bônus utilizada para ativar os Braços da forma
Astral, você pode gastar 1 ponto de chi para invocar essa aparência por 10
minutos. Ela desaparece assim que você for incapacitado ou morto. Enquanto esse
semblante espectral estiver presente, você adquire os seguintes benefícios.

\paragraph{Visão Astral.}%
Você pode enxergar normalmente na escuridão, tanto mágica quanto não-mágica, a
uma distância de até 36m.

\paragraph{Sabedoria Astral.}%
Você possui vantagem nos testes de Sabedoria (Intuição) e Carisma (Intimidação).

\paragraph{Palavra Astral.}%
Quando você fala, você pode direcionar as suas palavras a uma criatura a sua
escolha que você possa ver a até 18m de você, fazendo com que apenas essa
criatura possa ouvi-lo. Alternativamente, você pode amplificar sua voz de forma
que todas as criaturas a até 180m possam escutá-lo.

\subsubsection*{Corpo Astral}%
\label{ssub:corpo_astral}

No 11º nível, quando você tiver tanto seus braços quanto seu semblante da forma
astral ativos, você pode fazer com que o corpo de sua forma astral apareça
(nenhuma ação é necessária). Esse corpo espectral recobre sua forma física como
uma armadura, conectando seus braços e seu semblante. Você determina sua
aparência. Enquanto o corpo astral está ativo, você ganha os seguintes
benefícios.

\paragraph{Defletir energia.}%
Quando você sofre dano ácido, gélido, ígneo, elétrico, trovejante ou de energia,
você pode usar sua reação para defletir isso. Ao fazê-lo, o dano que você
sofreria é reduzido em 1d10 + seu modificador de Sabedoria (com a redução mínima
de 1 ponto).

\paragraph{Braços aprimorados.}%
Uma vez em cada um dos seus turnos, quando você acertar um alvo com os Braços da
Forma Astral, você pode provocar dano extra ao alvo equivalente ao valor de seu
dado de Artes Marciais.

\subsubsection*{Forma Astral Desperta}%
\label{ssub:forma_astral_desperta}

No 17º nível, sua conexão com sua forma astral está completa, permitindo que
você libere todo o seu potencial. Como uma ação bônus, você pode gastar 5 pontos
de chi para invocar os braços, o semblante e o corpo de sua forma astral e
despertá-las por 10 minutos. Essa forma desaparece se você for incapacitado ou
morto.
Enquanto sua forma astral estiver desperta, você ganha os seguintes benefícios:

\paragraph{Armadura espiritual.}%
Você ganha +2 de bônus em sua CA.

\paragraph{Barreira astral.}%
Sempre que usar a característica de ataque extra para atacar duas vezes, você
poderá atacar uma terceira vez se todos os seus ataques forem realizados com os
braços da forma astral.

\end{multicols}
