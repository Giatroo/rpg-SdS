%!TeX root=../../rules.tex
\chapter{Monge}%
\label{cha:monge}
\begin{multicols}{2}

\section*{Características de Classe}%

Como um monge, você adquire as seguintes características de classe.

\subsubsection{Pontos de Vida}%

\noindent\textbf{Dado de Vida}: 1d8 por nível de monge \nl
\textbf{Pontos de Vida no 1º Nível:} 8 + seu modificador de Constituição. \nl
\textbf{Pontos de Vida nos Níveis Seguintes:} 1d8 (ou 5) + seu modificador de
Constituição por nível de monge após o 1º.

\subsubsection{Proficiências}%

\noindent\textbf{Armaduras:} Nenhuma \nl
\textbf{Armas:} Armas simples, espadas curtas \nl
\textbf{Ferramentas:} Escolha um tipo de ferramenta de artesão ou um instrumento
musical \jump
\textbf{Testes de Resistência}: Força, Destreza \nl
\textbf{Pericias:} Escolha dois dentre Acrobacia, Atletismo, Furtividade,
História, Intuição e Religião

\subsubsection{Equipamento}%

Você começa com o seguinte equipamento, além do equipamento concedido pelo seu
antecedente.
\begin{itemize}
    \item (a) uma espada curta ou (b) qualquer arma simples
    \item (a) um pacote de explorador ou (b) um pacote de aventureiro
    \item 10 dardos
\end{itemize}

\section*{Conjuração}%

A tabela a seguir mostra em quais níveis o monge tem acesso às magias de cada
ciclo:

\begin{center}
\begin{tabular}{|||c||c|||}
    \hline
    \textbf{Nível} & \textbf{Ciclo} \\
    \hline
    1 & 1º \\
    \hline
    3 & 2º \\
    \hline
    5 & 3º \\
    \hline
    7 & 4º \\
    \hline
    9 & 5º \\
    \hline
    11 & 6º \\
    \hline
    13 & 7º \\
    \hline
    15 & 8º \\
    \hline
    17 & 9º \\
    \hline
\end{tabular}
\end{center}

\subsubsection*{Habilidade de Conjuração}%

O maior entre sua Destreza e Sabedoria é a sua habilidade para você conjurar
suas magias de monge, pois os monges são um conjunto de velocidade, impacto e
paciência. Você usa o máximo entre Destreza e Sabedoria sempre que alguma magia
se referir a sua habilidade de conjurar magias. Além disso, você usa o máximo do
modificador de Destreza e Sabedoria para definir a CD dos testes de resistência
para as magias de monge que você conjura e quando você realiza uma jogada de
ataque com uma magia.

\begin{center}
\textbf{CD para suas magias} = 8 + bônus de proficiência + seu modificador de
Destreza ou Sabedoria (o maior dentro eles) \nl

\textbf{Modificador de ataque de magia} = seu bônus de proficiência + seu
modificador de Destreza ou Sabedoria (o maior dentro eles)
\end{center}

\section*{Defesa sem Armadura}%

A partir do 1º nível, quando você não estiver vestindo nenhuma armadura nem
empunhando um escudo, sua Classe de Armadura será 10 + seu modificador de
Destreza + seu modificador de Sabedoria.

\section*{Artes Marciais}%

No 1º nível, sua prática nas artes marciais concede a você maestria nos estilos
de combate que utilizam golpes desarmados e armas de monge, que são as espadas
curtas e quaisquer armas simples corpo-a-corpo que não tenham a propriedade duas
mãos ou pesada.

\end{multicols}
