%!TeX root=../../rules.tex
\chapter{Mago}%
\label{cha:mago}
\begin{multicols}{2}

\section*{Características de Classe}%

Como um mago, você adquire as seguintes características de classe.

\subsubsection{Pontos de Vida}%

\noindent\textbf{Dado de Vida}: 1d6 por nível de mago \nl
\textbf{Pontos de Vida no 1º Nível:} 6 + seu modificador de Constituição. \nl
\textbf{Pontos de Vida nos Níveis Seguintes:} 1d6 (ou 4) + seu modificador de
Constituição por nível de mago após o 1º.

\subsubsection{Proficiências}%

\noindent\textbf{Armaduras:} Nenhuma \nl
\textbf{Armas:} Armas simples \nl
\textbf{Ferramentas:} Nenhuma \jump
\textbf{Testes de Resistência}: Inteligência, Sabedoria \nl
\textbf{Pericias:} Escolha duas entre Arcanismo, Atuação, História, Intuição,
Investigação, Persuasão.

\subsubsection{Equipamento}%

Você começa com o seguinte equipamento, além do equipamento concedido pelo seu
antecedente.
\begin{itemize}
    \item (a) um bordão ou (b) uma adaga ou (c) um arco curto;
    \item (a) uma bolsa de componentes ou (b) um foco arcano;
    \item (a) um pacote de estudioso ou (b) um pacote de explorador;
\end{itemize}

\section*{Conjuração}%

A tabela a seguir mostra em quais níveis o mago tem acesso às magias de cada
ciclo:

\begin{center}
\begin{tabular}{|||c||c|||}
    \hline
    \textbf{Nível} & \textbf{Ciclo} \\
    \hline
    1 & 1º \\
    \hline
    3 & 2º \\
    \hline
    5 & 3º \\
    \hline
    7 & 4º \\
    \hline
    9 & 5º \\
    \hline
    11 & 6º \\
    \hline
    13 & 7º \\
    \hline
    15 & 8º \\
    \hline
    17 & 9º \\
    \hline
\end{tabular}
\end{center}

\subsubsection*{Habilidade de Conjuração}%

Inteligência é a sua habilidade para você conjurar suas magias de mago, pois os
magos aprendem novas magias através de estudo e memorização. Você usa sua
Inteligência sempre que alguma magia se referir a sua habilidade de conjurar
magias. Além disso, você usa o seu modificador de Inteligência para definir a CD
dos testes de resistência para as magias de mago que você conjura e quando você
realiza uma jogada de ataque com uma magia.

\begin{center}
\textbf{CD para suas magias} = 8 + bônus de proficiência + seu modificador de
Inteligência. \nl

\textbf{Modificador de ataque de magia} = seu bônus de proficiência + seu
modificador de Inteligência
\end{center}

\end{multicols}
\begin{center}
\begin{tabular}{
        | b{8mm}<{\centering}
        b{23mm}<{\centering}
        p{70mm}<{\raggedright\arraybackslash} |
}
    \hline

    \multicolumn{3}{|l|}{\textbf{\Large O Mago}} \\

    % Header
    \textbf{Nível} & \textbf{Bônus de Proficiência} & \textbf{Características}
    \\
    \hline \hline

    \textbf{1º} & +2 & Plano Astral \\
    \hline
    \textbf{2º} & +2 & Metamagia \\
    \hline
    \textbf{3º} & +2 & Foco na Batalha \\
    \hline
    \textbf{4º} & +2 & Incremento em Habilidade \\
    \hline
    \textbf{5º} & +3 & Conjuração Extra \\
    \hline
    \textbf{6º} & +3 & - \\
    \hline
    \textbf{7º} & +3 & Conjurador Nato \\
    \hline
    \textbf{8º} & +3 & Incremento em Habilidade \\
    \hline
    \textbf{9º} & +4 & - \\
    \hline
    \textbf{10º} & +4 & Metamagia \\
    \hline
    \textbf{11º} & +4 & Dominar Magia, Conjuração Extra \\
    \hline
    \textbf{12º} & +4 & Incremento em Habilidade \\
    \hline
    \textbf{13º} & +5 & - \\
    \hline
    \textbf{14º} & +5 & - \\
    \hline
    \textbf{15º} & +5 & Dominar Magia \\
    \hline
    \textbf{16º} & +5 & Incremento em Habilidade \\
    \hline
    \textbf{17º} & +6 & Metamagia \\
    \hline
    \textbf{18º} & +6 & - \\
    \hline
    \textbf{19º} & +6 & Incremento em Habilidade, Dominar Magia \\
    \hline
    \textbf{20º} & +6 & Conjuração Extra \\
    \hline
\end{tabular}
\end{center}
\begin{multicols}{2}

\section*{Plano Astral}%

Todos os conjuradores possuem um plano astral. Nele, o conjurador guarda sua
mana e pode canalizá-la para conjurar maravilhas no Mundo Natural.

\subsection*{Plano expandido}%

Com muito tempo de estudo e dedicação, os magos conseguem expandir seu plano
astral, tornando-os especialistas em conjuração de magias que outras classes
nunca conseguiram conjurar.

Os estudos e prática de um mago fazem com que seu plano se fortaleça e expanda
como um músculo. Você possui o dobro de mana armazenado em seu plano astral do
que qualquer outra classe conseguiria.

Além disso, ao invés de começar a ficar exausto com 20\% da mana, um mago
começa a ficar exausto com 15\%. E portanto, o teste é 10 + (15\% - porcentagem
atual de mana restante).

\subsection*{Recuperação Arcana}%

Você aprendeu como recuperar um pouco de sua mana em seus estudos como mago. Uma
vez por dia, quando você terminar um descanso curto, você pode recuperar $10\%$
de sua mana total.

\section*{Metamagia}%

No 2º nível, você adquire a habilidade de distorcer suas magias para se
adequarem às suas necessidades. Você ganha duas das seguintes opções de
Metamagia, à sua escolha. Você adquire outra no 10º e 17º nível.

Você pode usar apenas uma opção de Metamagia em uma magia quando a conjura, a
não ser que esteja descrito o contrário.

\subsubsection*{Magia Acelerada}%

Quando você conjurar uma magia que tenha um tempo de conjuração de 1 ação, você
gasta $50\%$ a mais da mana descrita na magia para conjurá-la como uma ação
bônus.

\subsubsection*{Magia Aumentada}%

Quando você conjura uma magia que obriga uma criatura a realizar um teste de
resistência contra o seu efeito, você pode gastar $75\%$ a mais da mana descrita
na magia para dar desvantagem a um alvo da magia no primeiro teste de
resistência feito contra ela.

\subsubsection*{Magia Cuidadosa}%

Quando você conjurar uma magia que obriga outras criaturas a realizarem um teste
de resistência, você pode proteger algumas dessas criaturas da força total da
magia. Para tanto, você gasta $25\%$ a mais da mana descrita na magia e escolhe
um número dessas criaturas até o seu modificador de Carisma (mínimo de uma
criatura). Uma criatura escolhida passa automaticamente no teste de resistência
contra a magia.

\subsubsection*{Magia Distante}%

Quando você conjurar uma magia que tenha distância de $1,5$ metro ou maior, você
pode gastar $25\%$ a mais da mana descrita na magia para dobrar o alcance dela.

Quando você conjurar uma magia com alcance de toque, você pode gastar $75\%$ da
mana descrita na magia para mudar o alcance para $4,5$ metros.*

\subsubsection*{Magia Duplicada}%

Quando você conjurar uma magia que seja incapaz de ter mais de uma criatura como
alvo no nível atual dela e não possua alcance pessoal, você pode gastar uma
quantidade de mana igual a $25\%$ da mana descrita na magia vezes o nível da
magia para ter uma segunda criatura, no alcance da magia, como alvo (se a magia
for um truque, então o gasto é de $25\%$ da mana descrita na magia).

\subsubsection{Magia Estendida}%

Quando você conjurar uma magia que tenha duração de $1$ minuto ou maior, você
pode gastar $50\%$ da mana usada para conjurar a magia para reduzir pela metade
o custo adicional de mana por tempo descrito na magia e dobrar o tempo total de
duração da magia (máximo de $24$ horas).

\subsubsection*{Magia Perseguidora}%

Se você realizar uma jogada de ataque para uma magia e errar, você pode gastar
$50\%$ da mana usada para conjurar a magia e jogar o d20 novamente, ficando com
o novo resultado.

Você pode usar a Magia Perseguidora mesmo se já tiver usado outra opção
Metamagia durante a conjuração da magia.

\subsubsection*{Magia Sutil}%

Quando você conjurar uma magia, você pode gastar $25\%$ do custo descrito na
magia para fazê-lo sem qualquer componente somático ou verbal.

\section*{Foco na batalha}%

A partir do 3º nível, o mago possui uma experiência em combate grande o
suficiente para conjurar magias sem se distrair com os barulhos e luzes da
batalha ao seu redor.

Quando você rolar o dano de uma magia e rolar o maior valor possível em qualquer
dado, escolha um desses dados, role ele novamente e adicione o valor rolado ao
dano.

Você só pode fazer isso uma vez por rodada.

\section*{Incremento em Habilidade}%

Quando você atinge o 4º nível e novamente no 8º, 12º, 16º e 19º nível, você pode
ganha dois pontos de habilidades para distribuir entre suas habilidades. Por
padrão, você não pode elevar um valor de habilidade acima de 20 com essa
característica.

Opcionalmente, você pode escolher um talento seguindo as regras de talentos.

\section*{Conjuração Extra}%

A partir do 5º nível, você pode conjurar duas vezes, ao invés de uma, quando
usar a ação Conjuração durante o seu turno.

Esse número aumenta para 3 quando você alcançar o 11º nível de mago e para 4
quando alcançar o 20º nível de mago.

\section*{Conjurador Nato}%
\label{sec:conjurador_nato}

A partir do 7º nível, o mago mostra ser um conjurador mais eficiente que
qualquer outro tipo, executando as mesmas magias que outros conjuradores, mas de
forma muito mais eficiente.

A partir de agora, sempre que uma magia exigir uma rolagem de dados, você joga o
dobro da quantidade de dados indicada e pega a metade de dados maiores. Por
exemplo, se uma magia possui 4d6 de dano, então você deve rolar 8d6 e pegar os
4 maiores valores.

Para fazer isso, entretanto, o mago precisa estar utilizando seu item de
conjuração e ter pelo menos uma mão livre para fazer meio \textit{símbolo de
poder}.

\section*{Dominar Magia}%

A partir do 11º nível, você treinou e utilizou tanto certas magias em combate
que já sabe conjurá-las sem qualquer esforço. Escolha duas magias de 1º nível e
uma magia de 2º nível para conjurar sem utilizar mana. No nível 15º, você pode
escolher mais uma magia de 1º nível e uma de 3º nível. No nível 19º, você pode
escolher mais uma magia de 2º nível e uma magia de 4º nível.

Você utiliza as magias sem gastar mana em seu nível mínimo. Se você quiser
melhorar a magia, então precisará gastar a mana adicional que você gastaria.

\end{multicols}

%%%%%%%%%%%%%%%%%%%%%%%%%%%%%%%%%%%%%%%%%%%%%%%%%%%%%%%%%%%%%%%%%%%%%%%%%%%%%%%%
