%!TeX root=../../rules.tex
\chapter{Bardo}%
\label{cha:bardo}
\begin{multicols}{2}

\section*{Características de Classe}%

Como um bardo, você adquire as seguintes características de classe.

\subsubsection{Pontos de Vida}%

\noindent\textbf{Dado de Vida}: 1d8 por nível de bardo \nl
\textbf{Pontos de Vida no 1º Nível:} 8 + seu modificador de Constituição. \nl
\textbf{Pontos de Vida nos Níveis Seguintes:} 1d8 (ou 5) + seu modificador de
Constituição por nível de bardo após o 1º.

\subsubsection{Proficiências}%

\noindent\textbf{Armaduras:} Armaduras leves \nl
\textbf{Armas:} Armas simples, bestas de mão, espadas longas, rapieiras, espadas curtas \nl
\textbf{Ferramentas:} Três instrumentos musicais, à sua escolha \jump
\textbf{Testes de Resistência}: Destreza, Carisma \nl
\textbf{Pericias:} Escolha três quaisquer

\subsubsection{Equipamento}%

Você começa com o seguinte equipamento, além do equipamento concedido pelo seu
antecedente.
\begin{itemize}
    \item (a) uma rapieira, (b) uma espada longa ou (c) qualquer arma simples
    \item (a) um pacote de diplomata ou (b) um pacote de artista
    \item um instrumento musical a sua escolha
    \item uma armadura de couro e uma adaga
\end{itemize}

\section*{Conjuração}%

A tabela a seguir mostra em quais níveis o bardo tem acesso às magias de cada
ciclo:

\begin{center}
\begin{tabular}{|||c||c|||}
    \hline
    \textbf{Nível} & \textbf{Ciclo} \\
    \hline
    1 & 1º \\
    \hline
    3 & 2º \\
    \hline
    5 & 3º \\
    \hline
    7 & 4º \\
    \hline
    9 & 5º \\
    \hline
    11 & 6º \\
    \hline
    13 & 7º \\
    \hline
    15 & 8º \\
    \hline
    17 & 9º \\
    \hline
\end{tabular}
\end{center}

\subsubsection*{Habilidade de Conjuração}%

Carisma é a sua habilidade para você conjurar suas magias de bardo, pois os
bardos encantam seus aliados e fazem seus inimigos temerem sua beleza.
Você usa seu Carisma sempre que alguma magia se referir a sua habilidade de
conjurar magias. Além disso, você usa o seu modificador de Carisma para definir
a CD dos testes de resistência para as magias de bardo que você conjura e
quando você realiza uma jogada de ataque com uma magia.

\begin{center}
\textbf{CD para suas magias} = 8 + bônus de proficiência + seu modificador de
Carisma. \nl

\textbf{Modificador de ataque de magia} = seu bônus de proficiência + seu
modificador de Carisma
\end{center}

\end{multicols}
\begin{center}
\begin{tabular}{
        | b{8mm}<{\centering}
        b{23mm}<{\centering}
        b{20mm}<{\centering}
        b{20mm}<{\centering}
        p{70mm}<{\raggedright\arraybackslash} |
}
    \hline

    \multicolumn{5}{|l|}{\textbf{\Large O Bardo}} \\

    % Header
    \textbf{Nível} & \textbf{Bônus de Proficiência} & \textbf{Inspiração de
    Bardo} & \textbf{Canção de Descanso} & \textbf{Características} \\
    \hline \hline

    \textbf{1º} & +2 & 1d6 & 1d6 & Plano Astral, Inspiração de Bardo \\
    \hline
    \textbf{2º} & +2 & 1d6 & 1d6 & Versatilidade, Canção de Descanso \\
    \hline
    \textbf{3º} & +2 & 1d6 & 1d6 & Colégio de Bardo, Aptidão \\
    \hline
    \textbf{4º} & +2 & 1d6 & 1d6 & Incremento em Habilidade \\
    \hline
    \textbf{5º} & +3 & 1d8 & 1d6 & Fonte de Inspiração \\
    \hline
    \textbf{6º} & +3 & 1d8 & 1d6 & Canção de Proteção, Colégio de Bardo \\
    \hline
    \textbf{7º} & +3 & 1d8 & 1d6 &  - \\
    \hline
    \textbf{8º} & +3 & 1d8 & 1d6 & Incremento em Habilidade \\
    \hline
    \textbf{9º} & +4 & 1d8 & 1d8 & - \\
    \hline
    \textbf{10º} & +4 & 1d10 & 1d8 & Segredos Mágicos, Aptidão \\
    \hline
    \textbf{11º} & +4 & 1d10 & 1d8 & - \\
    \hline
    \textbf{12º} & +4 & 1d10 & 1d8 & Incremento em Habilidade \\
    \hline
    \textbf{13º} & +5 & 1d10 & 1d10 & - \\
    \hline
    \textbf{14º} & +5 & 1d10 & 1d10 & Colégio de Bardo, Segredos Mágicos \\
    \hline
    \textbf{15º} & +5 & 1d12 & 1d10 & - \\
    \hline
    \textbf{16º} & +5 & 1d12 & 1d10 & Incremento em Habilidade \\
    \hline
    \textbf{17º} & +6 & 1d12 & 1d12 & - \\
    \hline
    \textbf{18º} & +6 & 1d12 & 1d12 & Segredos Mágicos \\
    \hline
    \textbf{19º} & +6 & 1d12 & 1d12 & Incremento em Habilidade \\
    \hline
    \textbf{20º} & +6 & 1d12 & 1d12 & Inspiração Superior \\
    \hline
\end{tabular}
\end{center}
\begin{multicols}{2}

\section*{Plano Astral}%

Todos os conjuradores possuem um plano astral. Nele, o conjurador guarda sua
mana e pode canalizá-la para conjurar maravilhas no Mundo Natural.

\subsection*{Plano expandido}%

Com muito tempo de estudo e dedicação, os bardos conseguem expandir seu plano
astral, tornando-os especialistas em conjuração de magias que outras classes
nunca conseguiram conjurar.

Os estudos e prática de um bardo fazem com que seu plano se fortaleça e expanda
como um músculo. Você possui o dobro de mana armazenado em seu plano astral do
descrido como mana base.

Além disso, ao invés de começar a ficar exausto com 20\% da mana, um bardo
começa a ficar exausto com 15\%. E portanto, o teste é 10 + (15\% - porcentagem
atual de mana restante).

\subsection*{Talento Musical}%

Para conseguirem suas proezas, os bardos precisam se focar apenas na música, seu
taleto é todo proveniente dela. Eles obrigatoriamente conjuram magias da escola
musical quando começam no 1º nível. Mas podem adquirir outras escolas ao longo
do tempo.

Além disso, você deve usar um instrumento musical como foco de conjuração.

\section*{Inspiração de Bardo}

Você pode inspirar os outros através de palavras animadoras ou música. Para
tanto, você usa uma ação bônus no seu turno para escolher uma outra criatura,
que não seja você mesmo, a até 18 metros de você que possa ouvi-lo. Essa
criatura ganha um dado de Inspiração de Bardo, um d6.

Uma vez, nos próximos 10 minutos, a criatura poderá rolar o dado e adicionar o
valor rolado a um teste de habilidade, jogada de ataque ou teste de resistência
que ela fizer. A criatura pode esperar até rolar o d20 antes de decidir usar o
dado de Inspiração de Bardo, mas deve decidir antes do Mestre dizer se a rolagem
foi bem ou mal sucedida. Quando o dado de Inspiração de Bardo for rolado, ele é
gasto. Uma criatura pode ter apenas um dado de Inspiração de Bardo por vez.

Você pode usar essa característica um número de vezes igual ao seu modificador
de Carisma (no mínimo uma vez). Você recupera todos os usos quando termina um
descanso longo.

Seu dado de Inspiração de Bardo muda quando você atinge certos níveis na classe.
O dado se torna um d8 no 5° nível, um d10 no 10° nível e um d12 no 15° nível.

\section*{Versatilidade}%

A partir do 2° nível, você pode adicionar metade do seu bônus de proficiência,
arredondado para baixo, em qualquer teste de habilidade que você fizer que ainda
não possua seu bônus de proficiência.

\section*{Canção de Descanso}%

A partir do 2° nível, você pode usar música ou oração calmantes para ajudar a
revitalizar seus aliados feridos durante um descanso curto. Se você ou qualquer
criatura amigável que puder ouvir sua atuação recuperar pontos de vida no fim do
descanso curto ao gastar um ou mais Dados de Vida, cada uma dessas criaturas
recupera 1d6 pontos de vida adicionais.

Além disso, você pode escolher uma dessas criaturas que gastaram Dados de Vida
para recuperar 5\% da mana total do seu personagem.
Os pontos de vida adicionais aumentam quando você alcança determinados níveis na
classe: para 1d8 no 9° nível, para 1d10 no 13° nível e para 1d12 no 17°
nível.

A quantidade de criaturas que podem recuperar mana também aumenta quando você
alcança determinados níveis na classe: para duas no 5º nível, para três no
9º nível, para quatro no 13º e para cinco no 17º.

\section*{Colégio de Bardo}%

No 3° nível, você investiga as técnicas avançadas de um colégio de bardo, à sua
escolha: o Colégio do Conhecimento ou o Colégio da Bravura, ambos detalhados no
final da descrição da classe. Sua escolha lhe concede características no 3°
nível e novamente no 6° e 14° nível.

\section*{Aptidão}%

No 3° nível, escolha duas das perícias em que você é proficiente. Seu bônus de
proficiência é dobrado em qualquer teste de habilidade que você fizer que
utilize qualquer das perícias escolhidas.

No 10° nível, você escolhe mais duas perícias em que é proficiente para ganhar
esse benefício.

\section*{Incremento em Habilidade}%

Quando você atinge o 4° nível e novamente no 8°, 12°, 16° e 19° nível, você pode
aumentar um valor de habilidade, à sua escolha, em 2 ou você pode aumentar dois
valores de habilidade, à sua escolha, em 1. Como padrão, você não pode elevar um
valor de habilidade acima de 20 com essa característica.

\section*{Fonte de Inspiração}%

Quando você atinge o 5° nível, você passa a recuperar todos as utilizações
gastas da sua Inspiração de Bardo quando você termina um descanso curto ou
longo.

\section*{Canção de Proteção}%

No 6° nível, você adquire a habilidade de usar notas musicais ou palavras de
poder para interromper efeito de influência mental. Com uma ação, você pode
começar uma atuação que dura até o fim do seu próximo turno. Durante esse tempo,
você e qualquer criatura amigável a até 9 metros de você terá vantagem em testes
de resistência para não ser amedrontado ou enfeitiçado. Uma criatura deve ser
capaz de ouvir você para receber esse benefício.

A atuação termina prematuramente se você for incapacitado ou silenciado ou se
você terminá-la voluntariamente (não requer ação).

\section*{Segredos Mágicos}%

No 10° nível, você usurpou conhecimento mágico de um vasto espectro de
disciplinas. Escolha duas magias de qualquer classe e ordem. A magia que você
escolher deve ser de um nível que você possa conjurar, ou um truque.

As magias escolhidas contam como magias de bardo para você.

Você aprende duas magias adicionais de qualquer classe no 14° nível e novamente
no 18° nível.

\section*{Inspiração Superior}%

No 20° nível, quando você rolar iniciativa e não tiver nenhum uso restante de
Inspiração de Bardo, você recupera um uso.

\section*{Colégios de Bardo}%

O caminho de um bardo é gregário. Bardos buscam uns aos outros para trocar
canções e histórias, gabando-se de suas realizações e partilhando seus
conhecimentos.

Bardos formam associações esporádicas, que eles chamam de colégios, para
facilitar sua coleta e preservar suas tradições.

\subsection*{Colégio do Conhecimento}%

Bardos do Colégio do Conhecimento conhecem algo sobre a maioria das coisas,
coletando pedaços de conhecimento de fontes tão diversas quanto tomos eruditos
ou contos de camponeses. Quer seja cantando baladas populares em taverna, quer
seja elaborando composições para cortes reais, esses bardos usam seus dons para
manter a audiência enfeitiçada. Quando os aplausos acabam, os membros da
audiência vão estar se questionando se tudo que eles creem é verdade, desde sua
crença no sacerdócio do templo local até sua lealdade ao rei.

A fidelidade desses bardos reside na busca pela beleza e verdade, não na
lealdade a um monarca ou em seguir os dogmas de uma divindade. Um nobre que
mantém um bardo desses como seu arauto ou conselheiro, sabe que o bardo prefere
ser honesto que político.

Os membros do colégio se reúnem em bibliotecas e, as vezes, em faculdades de
verdade, completas com salas de aula e dormitórios, para partilhar seu
conhecimento uns com os outros. Eles também se encontram em festivais ou em
assuntos de estado, onde eles podem expor corrupção, desvendar mentiras e zombar
da superestima de figuras de autoridade.

\subsubsection*{Proficiência Adicional}%

Quando você se junta ao Colégio do Conhecimento no 3° nível, você ganha
proficiência em três perícias, à sua escolha.

\subsubsection*{Palavras de Interrupção}%

Também no 3° nível, você aprende como usar sua perspicácia para distrair,
confundir e, de outras formas, atrapalhar a confiança e competência de outros.

Quando uma criatura que você pode ver a até 18 metros de você realizar uma
jogada de ataque, um teste de habilidade ou uma jogada de dano, você pode usar
sua reação para gastar um uso de Inspiração de Bardo, rolando o dado de
Inspiração de Bardo e subtraindo o número rolado da rolagem da criatura. Você
escolhe usar essa característica depois da criatura fazer a rolagem, mas antes
do Mestre determinar se a jogada de ataque ou teste de habilidade foi bem ou mal
sucedido, ou antes da criatura causar dano. A criatura será imune se não puder
ouvir ou se não puder ser enfeitiçada.

\subsubsection*{Segredos Mágicos Adicionais}%

No 6° nível, você aprende duas magias, à sua escolha, de qualquer classe e
ordem. As magias que você escolher devem ser de um nível que você possa conjurar
ou um truque. As magias escolhidas contam como magias de bardo pra você.

\subsubsection*{Perícia Inigualável}%

A partir do 14° nível, quando você fizer um teste de habilidade, você pode
gastar um uso de Inspiração de Bardo. Role o dado de Inspiração de Bardo e
adicione o número rolado ao seu teste de habilidade. Você pode escolher fazer
isso depois de rolar o dado do teste de habilidade, mas antes do Mestre dizer se
foi bem ou mal sucedido.

\subsection*{Colégio da Bravura}%

Os bardos do Colégio da Bravura são escaldos destemidos de quem os contos mantêm
viva a memória dos grandes heróis do passado, dessa forma inspirando uma nova
geração de heróis. Esses bardos se reúnem em salões de hidromel ou ao redor de
fogueiras para cantar os feitos dos grandiosos, tanto do passado quanto do
presente. Eles viajam pelos lugares para testemunhar grandes eventos em primeira
mão e para garantir que a memória desses eventos não se perca nesse mundo. Com
suas canções, eles inspiram outros a alcançar o mesmo patamar de realizações dos
antigos heróis.

\subsubsection{Proficiência Adicional}%

Quando você se junta ao Colégio da Bravura no 3° nível, você adquire
proficiência com armadura médias, escudos e armas marciais.

\subsubsection{Inspiração em Combate}%

Também no 3° nível, você aprende a inspirar os outros em batalha. Uma criatura
que possuir um dado de Inspiração de Bardo seu, pode rolar esse dado e adicionar
o número rolado a uma jogada de dano que ele tenha acabado de fazer.

Alternativamente, quando uma jogada de ataque for realizada contra essa
criatura, ela pode usar sua reação para rolar o dado de Inspiração de Bardo e
adicionar o número rolado a sua CA contra esse ataque, depois da rolagem ser
feita, mas antes de saber se errou ou acertou.

\subsubsection{Conjuração Extra}%

A partir do 6° nível, você pode conjurar duas vezes, ao invés de uma, sempre que
você realizar a ação de Conjuração no seu turno.

Esse número aumenta para 3 quanto você atingir o 12º nível de bardo e para 4
quando você alcançar o 20º nível de bardo.

\subsubsection{Dominar Magia}%

No 14° nível, você dominou a arte de tecer a conjuração e usar armas em um ato
harmonioso. Escolha duas magias de 1º nível e uma magia de 2º nível para
conjurar sem utilizar mana. No nível 20º, você pode escolher mais duas magias de
1º nível e uma de 3º nível.

\subsection*{Colégio do Glamour}%

O Colégio do Glamour é a casa dos bardos que dominaram seus oficias no vibrante
reino de Agrestia das Fadas ou sob a tutela de alguém que morava lá.  Treinados
por sátiros, eladrins e outros seres feéricos, esses bardos aprendem a usar sua
mágica para deleitar e cativar outros.

Os bardos desse colégio são dotados de uma mistura de admiração e medo. Suas
apresentações são materiais para lendas. Esses bardos são tão eloquentes que um
discurso ou uma música que executem pode fazer com que seus sequestradores o
libertem ileso e pode acalmar um dragão furioso. A mesma mágica que lhes
permitem dominar bestas, também pode dobrar a mentes. Bardos vilões desse
colégio podem aproveitar-se de uma comunidade por semanas, abusando de sua
magia para transformar seus habitantes em escravos. Os bardos heróis, em vez
disso, usam esse poder para alegrar os oprimidos e prejudicar os opressores.

\subsubsection{Manto da Inspiração}%

Quando se unir ao colégio do Glamour no 3° nível, você ganha a habilidade de
entonar uma canção de magia feérica que imerge seus aliados com vigor e
velocidade.

Com uma ação bônus, você pode gastar uma de suas Inspirações de Bardo para se
conceder uma aparência maravilhosa. Quando fizer isso, escolha um número de
criaturas que você possa ver e que possam vê-lo a uma distãncia de 18 metros de
você, até um número de escolhidos igual ao seu modificador de Carisma (mínimo de
um). Cada um ganha 5 pontos de vida temporários. Quando uma criatura ganha
esses pontos de vida temporários, ela pode usar imediatamente sua reação se
deslocar até seu limite sem provocar ataques de oportunidade.

O número de pontos de vida temporários aumenta quando se atinge determinados
níveis nesta classe, aumentando para 8 no 5° nível, 11 no 10° nível e 14 no 15°
nível.

\subsubsection{Performance de Lumbrante}%

A partir do 3º nível, você pode carregar sua performance com magia sedutora e
feérica.
Se você fizer isto, por pelo menos 1 minuto, você pode tentar inspirar admiração
em sua plateia, cantando, recitando um poema ou dançando. No final de sua
atuação, escolha um número de humanoides a 18 metros de você que assistiram e
escutaram tudo, até um número igual ao seu modificador de Carisma (mínimo de
um).

Cada alvo deve ser bem-sucedido em um teste de resistência de Sabedoria contra
a CD de conjuração de magia ou ficará encantado por você. Enquanto encantado
dessa maneira, o alvo o idolatrará, dirá maravilhas ao seu respeito a qualquer
um que fale com ele sobre, e impede qualquer um se oponha a você, evitando
violência, a menos que sejam inclinados para lutas em seus interesses. O efeito
no alvo se encerra após 1 hora, se ele tomar qualquer dano, ou se testemunhar
seus ataques ou danos causados a qualquer um de seus aliados.

Se o alvo obtiver sucesso no teste de resistência, ele não terá noção que tentou
encanta-lo.

Depois de usar essa característica, não pode usá-la novamente até terminar um
descanso curto ou longo.

\subsubsection{Manto da Majestade}%

No 6° nível, você ganha a habilidade de se cobrir com uma magia feérica que faz
com que os outros queiram servi-lo. Como uma ação bônus, você conjura
\textit{comando} sem gastar mana e adquire uma aparência de beleza sobrenatural
por 1 minuto ou até que sua concentração acabe (como se estivesse concentrando
em uma magia).

Durante este tempo, você pode conjurar \textit{comando} como uma ação bônus em
qualquer um de seus turnos, sem gastar mana. Qualquer criatura encantada por uma
conjuração utilizando essa característica automaticamente falha em testes de
resistências contra a magia \textit{comando}. Uma vez usado esta característica,
só poderá usá-la novamente após terminar um descanso longo.

\subsubsection{Majestade Inquebrável}%

No 14º nível, sua aparência ganha traços permanentes que faz com que um aspecto
de outro mundo, parecendo mais adorável e feroz.

Além disso, como uma ação de bônus, você pode assumir uma presença magicamente
majestosa por 1 minuto ou até que esteja incapacitado. Durante a duração, sempre
que qualquer criatura tenta atacá-lo pela primeira vez em um turno, o atacante
deve fazer um teste de resistência de Carisma contra o CD de resistência à
magia. Se falhar, não poderá ataca-lo neste turno e deverá escolher um novo alvo
ou o ataque é cancelado. Em caso de um teste bem sucedido, a criatura poderá
ataca-lo neste turno, mas terá desvantagem em qualquer teste de resistência
contra suas magias no próximo turno.

Uma vez que assuma essa presença majestosa, não poderá fazê-lo novamente até
terminar um descanso longo ou curto.

\subsection*{Colégio das Espadas}%

Bardos do colégio das espadas são chamados Lâminas e eles entretêm através de
façanhas ousadas com a proeza nas armas. Os Lâminas executam acrobacias como
engolir espadas, atirar facas e fazer malabarismo e combates simulados. Embora
eles usem suas armas para entreter, eles também são altamente treinados e
habilidosos guerreiros natos.

Seus talentos com armas inspiram muitos Lâminas a levarem uma vida dupla. Um
Lâmina pode usar um trupe de circo como cobertura de ações nefastas, como
assassinato, roubo e chantagem. Outros Lâminas atacam os ímpios, trazendo
justiça contra os cruéis e poderosos. As maiorias dos grupos ficam felizes em
aceitar o talento de um Lâmina para acrescentar emoção para as performances, mas
poucos artistas confiam plenamente neles.

Lâminas que abandonam suas vidas como artistas sempre se deparam com problemas
que tornam impossível com que mantenham suas vidas secretas. Um Lâmina pego
roubando ou se envolvendo com a justiça é uma responsabilidade muito grande para
a maioria das trupes. Com suas habilidades em armas e magias, esses Lâminas
trabalham como responsáveis por guildas de ladrões ou acabam por conta própria
como aventureiros.

\subsubsection{Proficiência Bônus}%

Quando você entra no Colégio das Espadas no 3° nível, você ganha proficiência
com armadura media e cimitarras.
Se você é proficiente com uma arma simples ou marcial, você pode usar como foco
de feitiço para os seus feitiços difíceis.

\subsubsection{Estilo de Luta}%

No 3° nível, você adota um estilo de luta como sua especialidade. Escolha uma
das seguintes opções. Você não pode escolher um estilo mais de uma vez mesmo que
algo no jogo lhe permita.

\textit{\textbf{Duelo.}} Quando você está empunhando uma arma corpo-a-corpo em
uma mão e nenhuma outra, você ganha um bônus de +2 nas rolagens de ataque com
essa arma.

\textit{\textbf{Combate com Duas Armas.}} Quando você se empenha em duas armas,
você pode adicionar o modificador de habilidade ao dano do segundo ataque.

\subsubsection{Floreio de Lâminas}%

No 3° nível, você aprende a realizar exibições impressionantes de proeza e
velocidade marcial.

Sempre que você toma a ação de Ataque no seu turno, sua velocidade de caminhada
aumenta 3 metros até o final do turno, e se um ataque de arma que você faz como
parte desta ação atinge uma criatura, você pode usar uma das seguintes opções de
Floreios de Lâmina de sua escolha. Você pode usar apenas uma opção por turno.

\textbf{\textit{Floreio Defensivo.}} Você pode gastar uma Inspiração de Bardo
para causar dano extra ao alvo que acerta. O dano é igual ao número que você
rola no dado de inspiração de bardo. Você também adiciona o número rolado a sua
CA até o início de seu próximo turno.

\textit{\textbf{Floreio Cortante.}} Você pode gastar uma de suas Inspirações de
Bardo para causar dano extra ao alvo que acerta e a qualquer outra criatura de
sua escolha que você possa a 1,5 metros de você. O dano é igual ao número que
você rola no dado de inspiração de bardo.

\textit{\textbf{Floreio Móvel.}} Você pode gastar uma de suas Inspirações de
Bardo para causar dano extra ao alvo que acerta. O dano é igual ao número que
você rola no dado de inspiração de bardo. Você também pode empurrar o alvo até
1,5 metros de distância de você. Você pode então usar instantaneamente sua
reação para se deslocar até um espaço desocupado dentro de 1,5 metros do alvo.

\subsubsection{Ataque Extra}%

No 6° nível, você pode atacar duas vezes ao invés de uma, sempre que você toma a
ação Atacar em seu turno.

\subsubsection{Floreio do Mestre}%

No 14° nível, sempre que você usa o Floreio de Lâminas, você pode rolar um d6 e
usar ao invés de gastar um dado de Inspiração do Bardo.
\end{multicols}


