%!TeX root=../../rules.tex
\chapter{Ladino}%
\label{cha:ladino}
\begin{multicols}{2}

\section*{Características de Classe}%

Como um ladino, você adquire as seguintes características de classe.

\subsubsection{Pontos de Vida}%

\noindent\textbf{Dado de Vida}: 1d8 por nível de ladino \nl
\textbf{Pontos de Vida no 1º Nível:} 8 + seu modificador de Constituição. \nl
\textbf{Pontos de Vida nos Níveis Seguintes:} 1d8 (ou 5) + seu modificador de
Constituição por nível de ladino após o 1º.

\subsubsection{Proficiências}%

\noindent\textbf{Armaduras:} Armaduras leves \nl
\textbf{Armas:} Armas simples, bestas de mão, espadas longas, rapieiras, espadas
curtas\nl
\textbf{Ferramentas:} Ferramentas de ladrão \jump
\textbf{Testes de Resistência}: Destreza, Inteligência \nl
\textbf{Pericias:} Escolha quatro entre Acrobacia, Atletismo, Atuação,
Enganação, Furtividade, Intimidação, Intuição, Investigação, Percepção,
Persuasão e Prestidigitação.

\subsubsection{Equipamento}%

Você começa com o seguinte equipamento, além do equipamento concedido pelo seu
antecedente.
\begin{itemize}
    \item (a) uma rapieira ou (b) uma espada longa;
    \item (a) um arco curto e uma aljava com 20 flechas ou (b) uma espada curta;
    \item (a) um pacote de assaltante ou (b) um pacote de aventureiro ou (c) um
        pacote de explorador;
    \item Armadura de couro, duas adagas e ferramentas de ladrão.
\end{itemize}

\section*{Conjuração}%

A tabela a seguir mostra em quais níveis o ladino tem acesso às magias de cada
ciclo:

\begin{center}
\begin{tabular}{|||c||c|||}
    \hline
    \textbf{Nível} & \textbf{Ciclo} \\
    \hline
    2 & 1º \\
    \hline
    5 & 2º \\
    \hline
    8 & 3º \\
    \hline
    11 & 4º \\
    \hline
    14 & 5º \\
    \hline
    17 & 6º \\
    \hline
    20 & 7º \\
    \hline
\end{tabular}
\end{center}


\subsubsection*{Habilidade de Conjuração}%

Destreza é a sua habilidade para você conjurar suas magias de ladino, pois
ladinos conjuram magias utilizando suas mãos rápidas e seu movimento sutil e
ágil como uma dança no meio do combate. Você usa sua Destreza sempre que alguma
magia se referir a sua habilidade de conjurar magias. Além disso, você usa o seu
modificador de Destreza para definir a CD dos testes de resistência para as
magias de ladino que você conjura e quando você realiza uma jogada de ataque com
uma magia.

\begin{center}
\textbf{CD para suas magias} = 8 + bônus de proficiência + seu modificador de
Destreza. \nl

\textbf{Modificador de ataque de magia} = seu bônus de proficiência + seu
modificador de Destreza
\end{center}

\end{multicols}
\begin{center}
\begin{tabular}{
        | b{8mm}<{\centering}
        b{23mm}<{\centering}
        b{15mm}<{\centering}
        p{70mm}<{\raggedright\arraybackslash} |
}
    \hline

    \multicolumn{4}{|l|}{\textbf{\Large O Ladino}} \\

    % Header
    \textbf{Nível} & \textbf{Bônus de Proficiência} & \textbf{Ataque Furtivo} &
    \textbf{Características} \\
    \hline \hline

    \textbf{1º} & +2 & 1d6 & Especialização, Ataque Furtivo, Gíria de Ladrão \\
    \hline
    \textbf{2º} & +2 & 1d6 & Ação Ardilosa \\
    \hline
    \textbf{3º} & +2 & 2d6 & Arquétipo de Ladino \\
    \hline
    \textbf{4º} & +2 & 2d6 & Incremento em Habilidade \\
    \hline
    \textbf{5º} & +3 & 3d6 & Esquiva Sobrenatural \\
    \hline
    \textbf{6º} & +3 & 3d6 & Especialização \\
    \hline
    \textbf{7º} & +3 & 4d6 & Evasão \\
    \hline
    \textbf{8º} & +3 & 4d6 & Incremento em Habilidade \\
    \hline
    \textbf{9º} & +4 & 5d6 & Arquétipo de Ladino \\
    \hline
    \textbf{10º} & +4 & 5d6 & Incremento em Habilidade \\
    \hline
    \textbf{11º} & +4 & 6d6 & Talento Confiável \\
    \hline
    \textbf{12º} & +4 & 6d6 & Incremento em Habilidade \\
    \hline
    \textbf{13º} & +5 & 7d6 & Arquétipo de Ladino \\
    \hline
    \textbf{14º} & +5 & 7d6 & Sentido Cego \\
    \hline
    \textbf{15º} & +5 & 8d6 & Mente Escorregadia \\
    \hline
    \textbf{16º} & +5 & 8d6 & Incremento em Habilidade \\
    \hline
    \textbf{17º} & +6 & 9d6 & Arquétipo de Ladino \\
    \hline
    \textbf{18º} & +6 & 9d6 & Elusivo \\
    \hline
    \textbf{19º} & +6 & 10d6 & Incremento em Habilidade \\
    \hline
    \textbf{20º} & +6 & 10d6 & Golpe de Sorte \\
    \hline
\end{tabular}
\end{center}
\begin{multicols}{2}

\section*{Especialização}%

No 1º nível, você escolhe duas de suas perícias que seja proficiente, ou uma
perícia que seja proficiente e ferramentas de ladrão. Seu bônus de proficiência
é dobrado em qualquer teste de habilidade que fizer com elas.

No 6º nível, você pode escolher outras duas de suas proficiências (em perícias
ou ferramentas de ladrão) para ganhar esse benefício.

\section*{Ataque Furtivo}%

A partir do 1º nível, você sabe como atacar sutilmente e explorar a distração de
seus inimigos. Uma vez por turno, você pode adicionar 1d6 nas jogadas de dano
contra qualquer criatura que acertar, desde que tenha vantagem nas jogadas de
ataque. O ataque deve ser com uma arma de acuidade ou à distância.

Você não precisa ter vantagem nas jogadas de ataque se outro inimigo do seu alvo
estiver a 1,5 metro de distância dele, desde que este inimigo não esteja
incapacitado e você não tenha desvantagem nas jogadas de ataque.

A quantidade de dano extra aumenta conforme você ganha níveis nessa classe. No
1º nível você começa com 1d6 e, a cada dois nível, você ganha um d6 adicional
(no nível 3, 5, 7, ...).

\section*{Gíria de Ladrão}%

Durante seu treinamento você aprendeu as gírias de ladrão, um misto de dialeto,
jargão e códigos secretos que permitem você passar mensagens secretas durante
uma conversa aparentemente normal. Somente outra criatura que conheça essas
gírias de ladrão entende as mensagens.

Leva-se quatro vezes mais tempo para transmitir essa mensagem do que falar a
mesma ideia claramente.

Além disso, você entende um conjunto de sinais secretos e símbolos usados para
transmitir mensagens curtas e simples, como saber se uma área é perigosa ou se é
território de uma guilda de ladrões, se o saque está próximo, se as pessoas na
área são alvos fáceis ou até mesmo indicar lugares seguros para ladinos se
esconderem.

\section*{Ação Ardilosa}%

A partir do 2º nível, seu pensamento rápido e agilidade faz você se mover e agir
rapidamente. Você pode usar uma ação bônus durante cada um de seus turnos em
combate. Esta ação pode ser usada somente para Disparada, Desengajar ou
Esconder.

\section*{Arquétipo de Ladino}%

No 3º nível, você escolhe um arquétipo que se esforçará para se equiparar
através de exercícios de suas habilidades de ladino. Sua escolha garante a você
características no 3º nível e de novo no 9º, 13º e 17º nível. As possíveis
escolhas estão detalhadas no final da descrição da classe.

\section*{Incremento em Habilidade}%

Quando você atinge o 4º nível e novamente no 8º, 10º, 12º, 16º e 19º nível, você
pode ganha dois pontos de habilidades para distribuir entre suas habilidades.
Por padrão, você não pode elevar um valor de habilidade acima de 20 com essa
característica.

Opcionalmente, você pode escolher um talento seguindo as regras de talentos.

\section*{Esquiva Sobrenatural}%

A partir do 5º nível, quando um inimigo que você possa ver o acerta com um
ataque, você pode usar sua reação para reduzir pela metade o dano sofrido.

\section*{Evasão}%

A partir do 7º nível, você pode esquivar-se agilmente de certos efeitos em área,
como o sopro flamejante de um dragão vermelho ou uma magia tempestade de gelo.

Quando você for alvo de um efeito que exige um teste de resistência de Destreza
para sofrer metade do dano, você não sofre dano algum se passar, e somente
metade do dano se falhar.

\section*{Talento Confiável}%

No 11º nível, você refinou suas perícias beirando à perfeição. Toda vez que você
fizer um teste de habilidade no qual possa adicionar seu bônus de proficiência,
você trata um resultado no d20 de 9 ou menor como um 10.

\section*{Sentido Cego}%

No 14º nível, se você for capaz de ouvir, você está ciente da localização de
qualquer criatura escondida ou invisível a até 3 metros de você

\section*{Mente Escorregadia}%

No 15º nível, você adquire uma grande força de vontade, adquirindo proficiência
nos testes de resistência de Sabedoria.

\section*{Elusivo}%

A partir do 18º nível, você se torna tão sagaz que raramente alguém encosta a
mão em você. Nenhuma jogada de ataque tem vantagem contra você, desde que você
não esteja incapacitado.

\section*{Golpe de Sorte}%

No 20º nível, você adquire um dom incrível para ter sucesso nos momentos em que
mais precisa. Se um ataque seu falhar contra um alvo ao seu alcance, você pode
transformar essa falha em um acerto. Ou se falhar em um teste qualquer, você
pode tratar a jogada desse mesmo teste como 20 natural.

Uma vez que você use essa característica, você não pode fazê-lo de novo até
terminar um descanso curto ou longo.

\section*{Arquétipos de Ladino}%

Ladinos possuem muitas características em comum, incluindo a ênfase no
aperfeiçoamento de suas perícias, na precisão e aproximação mortal em combate, e
nos seus reflexos cada vez mais rápidos. Mas, diferentes ladinos orientam seus
talentos em direções variadas, personificadas pelos vários arquétipos de ladino.
Seu arquétipo escolhido reflete o seu foco – não necessariamente a indicação de
sua profissão, mas a descrição de suas técnicas preferidas.

\subsection*{Assassino}%

Você focou seu treinamento na macabra arte da morte.

Aqueles que devotam-se a esse arquétipo são diversos: assassinos de aluguel,
espiões, caçadores de recompensa e, até mesmo, padres especialmente treinados em
exterminar os inimigos das suas divindades. Subterfúgio, veneno e disfarces
ajudam você a eliminar seus oponentes com eficiência mortífera.

\subsubsection{Proficiência Adicional}%

Quando você escolhe esse arquétipo, no 3º nível, você ganha proficiência com kit
de disfarce e kit de venenos.

\subsubsection{Assassinar}%

A partir do 3° nível, você fica mais mortal quando pega seus oponentes
desprevenidos. Você tem vantagem nas jogadas de ataque contra qualquer criatura
que ainda não tenha chegado ao turno dela no combate. Além disso, qualquer
ataque que você fizer contra essa criatura que está surpresa, será um ataque
crítico.

\subsubsection{Especialização em Infiltração}%

A partir do 9° nível, você pode infalivelmente, criar identidades falsas para si
mesmo. Você deve gastar sete dias e 25 po para estabelecer o histórico,
profissão e filiações para uma identidade. Você não pode estabelecer uma
identidade que pertença a alguém. Por exemplo, você deveria adquirir roupas
apropriadas, cartas de introdução e um certificado, aparentemente oficial, para
estabelecer-se como um membro da casa de comércio de uma cidade remota, assim,
você poderia introduzir-se na companhia de outros comerciantes abastados.

Posteriormente, se você adotar a nova identidade como disfarce, outras criaturas
acreditarão que você é aquela pessoa, até terem algum motivo obvio para pensarem
o contrário.

\subsubsection{Impostor}%

No 13° nível, você adquire a habilidade de imitar a fala, escrita e
comportamento de outra pessoa, infalivelmente.

Você deve gastar pelo menos três horas estudando esses três componentes do
comportamento de uma pessoa, ouvindo sua articulação, examinando sua escrita e
observando seus maneirismos.

Seu ardil é imperceptível para um observador casual. Se uma criatura desconfiada
suspeitar que algo está errado, você tem vantagem em qualquer teste de Carisma
(Enganação) que você fizer para evitar ser detectado.

\subsubsection{Golpe Letal}%

No 17° nível, você se torna um mestre da morte instantânea. Quando você atacar e
atingir uma criatura que esteja surpresa, ela deve realizar um teste de
resistência de Constituição (CD 8 + seu modificador de Destreza + seu bônus de
proficiência). Se ela falhar, dobre o dano do seu ataque contra a criatura.

\subsection*{Ladrão}%

Você aprimorou suas habilidades na arte do furto de pequenos itens. Gatunos,
bandidos, batedores de carteira e outros criminosos geralmente seguem esse
arquétipo, mas também aqueles ladinos que preferem se ver como caçadores de
tesouro profissionais, exploradores de masmorras e investigadores. Além de
aprimorar sua agilidade e furtividade, você aprende perícias úteis para
desbravar ruínas antigas, ler idiomas incomuns e usar itens mágicos que
normalmente não poderia.

\subsubsection{Mãos Rápidas}%

A partir do 3º nível, você pode usar a sua ação bônus concedida pela Ação
Ardilosa para fazer um teste de Destreza (Prestidigitação), usar suas
ferramentas de ladrão para desarmar uma armadilha ou abrir uma fechadura, ou
realizar a ação de Usar um Objeto.

\subsubsection{Andarilho de Telhados}%

No 3º nível, você adquire a habilidade de escalar mais rápido que o normal.
Escalar agora não possui custo adicional de movimento para você.

Além disso, quando você fizer um salto com corrida, o alcance que pode saltar
aumenta um número de metros igual a 0,3 vezes o seu modificador de Destreza.

\subsubsection{Furtividade Surpresa}%

A partir do 9º nível, você tem vantagem no teste de Destreza (Furtividade) se
você não mover-se mais do que a metade de seu deslocamento em um turno.

\subsubsection{Usar Instrumento Mágico}%

 No 13º nível, você aprende o suficiente sobre como a magia funciona e pode
 improvisar o uso de itens que nem mesmo foram destinados a você. Você ignora
 todos os requisitos de classes, raças e níveis para uso de qualquer item
 mágico.

\subsubsection{Reflexo de Ladrão}%

Quando atinge o 17º nível, você se torna adepto em fazer emboscadas e fugas
rápidas de situações perigosas. Você pode realizar dois turnos durante o
primeiro turno de cada combate. Você realiza seu primeiro turno na sua
iniciativa e o segundo na ordem de sua iniciativa menos 10.

Você não pode usar essa característica quando está surpreso.
\end{multicols}

%%%%%%%%%%%%%%%%%%%%%%%%%%%%%%%%%%%%%%%%%%%%%%%%%%%%%%%%%%%%%%%%%%%%%%%%%%%%%%%%
