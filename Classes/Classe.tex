%%%%%%%%%%%%%%%%%%%%%%%%%%%%%%%%%%%%%%%%%%%%%%%%%%%%%%%%%%%%%%%%%%%%%%%%%%%%%%%%

\documentclass{RPG_Adventure}[2021/10/20]

\input{/home/giatro/.config/user/giatro_packages.tex}

\input{/home/giatro/.config/user/giatro_macros.tex}

\title{\\ \Huge{O Senhor das Sombras}}
\date{\today}
\author{Lucas Paiolla Forastiere}

%%%%%%%%%%%%%%%%%%%%%%%%%%%%%%%%%%%%%%%%%%%%%%%%%%%%%%%%%%%%%%%%%%%%%%%%%%%%%%%%
%%%%%%%%%%%%%%%%%%%%%%%%%%%%%%%%%%%%%%%%%%%%%%%%%%%%%%%%%%%%%%%%%%%%%%%%%%%%%%%%

\begin{document}

\maketitle
\tableofcontents
\newpage

%%%%%%%%%%%%%%%%%%%%%%%%%%%%%%%%%%%%%%%%%%%%%%%%%%%%%%%%%%%%%%%%%%%%%%%%%%%%%%%%

\chapter{Raças}%
\label{cha:racas}

As raças continuam exatamente com as mesmas características que o Livro do
Jogador. Aqui detalharei melhor um pouco sobre como essas raças se dispõem na
\textit{lore} do mundo para que o jogador possa fazer um melhor \textit{role
play} e escolher uma raça que tenha uma história que o agrade mais.

\section*{Humanos\t\t\t\t\t\t\t\t\t\t\t\t\t}%
\label{sec:humanos}

Os humanos, raça mais expansionista de Jǒna, estão por toda parte: desde as
pequenas Cidades Amigas no Sul de Estália até as tribos isoladas em Cafrilar.
Entretanto, o grande ponto de concentração dos humanos é na famosa Península do
Goto, onde ficam as cidades de Calífora, Nize e Argan.

Essas cidades surgiram das guerras e unificações entres antigas tribos (ou
autodenominadas guildas) que habitavam os ricos da região, princialmente o
Kaitû e o Jote. Há muitos anos, uma série de guildas habitavam o rio Kaitû, até
que as águas dele começaram a ficar venenosas e intragáveis. Os Archon e os
Grégios, duas dessas guildas, entraram em guerra culpando uma à outra pelo
ocorrido. Outras guildas como os Taêno migraram da região para o norte, onde
encontraram um novo rio, que hoje é conhecido como Rio Taêno, e lá montaram suas
residências. Já os Saicro habitavam o rio Jote, e forneciam água para os que
habitavam o Kaitû, cobrando caro por isso e impedindo que essas guildas
migrassem para o Jote.

Após vários eventos, dois humanos conhecidos como Aaron e Victor Wilton chegaram
até a região e uniram essas guildas revelando serem magos Saicro que estavam o
tempo todo envenenando a água só para vender água potável para essas guildas.
Então as guildas do Kaitû se juntaram e travaram uma guerra contra os Saicro e
outras guildas do norte, expulsando-os mais e mais para o norte e tomando o Jote
para si.

As guildas humanas no Norte fundaram a cidade de Nize, que fica na costa Norte
da península e tem sua divisão praticamente marcada pelo Rio Nize. Já as guildas
do Sul se uniram e fundaram Calífora, a maior potência economia e militar de
toda Estália até os dias de hoje.

\section*{Draconianos\t\t\t\t\t\t\t\t\t\t\t}%
\label{sec:draconianos}

Os Draconianos são uma das raças mais incomuns em Estália. Eles são oriundos do
longínquo Império de Kenia, conhecido no continente do Sul como o Império dos
Dragões.

Kenia é um império antiquíssimo, que já existia na época de Aaron I, o fundador
de Calífora, e que muitos dizem ter nascido de uma cidade mais antiga ainda
conhecida como Retuc, que hoje é a atual capital do império.

O continente de Cafrilar, onde fica Kenia, é conhecido por ter um clima árido e
desértico, e ser de difícil acesso. Todas as outras raças de Jǒna pouco se
adaptam a esse ambiente, mas para os draconianos, esse não poderia ser um lar
melhor.

Poucos são aqueles de Estália que já foram até Kenia, e a maioria deles são
mercadores que navegam no Entreterras e chegam ao Porto de Das-Galul, talvez um
dos maiores e mais belos portos do mundo. O porto fica na cidade de Galul-Gadev.
Ela fica ao sul de Retuc e é a principal exportadora dos caríssimos produtos
kenianos.

Os povos do Sul adoram e pagam muito caro pelos produtos advindos de Retuc.
Kenia é bastante conhecida por seus palácios majestosos e aqueles realmente
ricos no Sul tentam copiar essa cultura comprando diversos tipos de produtos:
desde a culinária, com especiarias e farinhas das mais diversas; passando pelos
produtos de beleza, como pó de maquiagem, bijuterias de ouro e prata dos mais
puros, vestidos cravejados de gemas preciosas ou até itens mágicos como cajados
formidáveis e tomos de capa de couro de criaturas do deserto; e também materiais
de decoração como enormes tapetes, vasos de cerâmica, castiçais e candelabros,
quadros e esculturas, todos feitos à mão pelos renomados artistas draconianos da
capital de Kenia.

Apesar de Galul-Gadev ser uma cidade riquíssima, devido ao valioso comércio com
os povos de Estália, a capital do Império dos Dragões é Retuc, sem dúvida a
maior cidade de Cafrilar e, segundo alguns, a maior cidade de todo o mundo
(muito mais bonita que Calífora, confirme dizem).

Na cidade de Retuc se encontram os mais ricos e poderoso draconianos de todos.
Ela é uma cidade que fica mais ao norte de Cafrilar, a vários quilômetros de
Galul-Gadev, na beira de um enorme lago no meio do deserto, fazendo dela um
local perfeito para se habitar. Ali são pouquíssimos os não-draconianos e esses
não são tratados com muita empatia.

Da mesma forma que os sulistas adoram fazer negócios com os kenianos, eles
também os temem veementemente. Kenia é um império escravagista e extremamente
militarizado. Toda a vida em Retuc é uma vida militar. Os draconianos homens são
retirados de suas mães logo cedo e enfiados em barracas no meio de campos
militares onde estão destinados a treinar para se tornarem soldados ou falhar
tentando. Os que se tornam soldados ficam conhecidos como Salgum e são o cargo
mais honroso possível. Já os que falham acabam se tornando Gadum, que são cargos
secundários como comerciantes, construtores, etc.

Os Saldum por sua vez possuem uma hierarquia entre si, que começa nos
adolescentes recém formados pelo processo seletivo e chega nos grandes generais
dos exércitos de cada um dos elementos draconianos, dos quais alguns fazem parte
do conselho do império que aconselha o grande imperador, ou Dracon-Gastol como
falam os draconianos.

Já as mulheres, muitas vezes, ficam com um papel secundário no império, mas que
dão frutos a grande parte da cultura de lindos objetos de decoração e
embelezamento. Algumas delas, entretanto, se tornam Senhoras do Destino, ou
Sas-Ladnun. Elas são uma espécie de ordem de draconianas feiticeiras cujas mais
velhas são extremamente poderosas. Elas não vão ao campo de batalha como os
homens, mas exercem seu poder secretamente. Nem sequer o próprio Dracon-Gastol
sabe ao certo como a ordem realmente opera. Este é um segredo apenas das
próprias Sas-Ladnun.

\section*{Anões\t\t\t\t\t\t\t\t\t\t\t\t\t\t\t}%
\label{sec:anoes}

Uma das raças menos sociáveis de Estália, os anões são conhecidos por sua
arquitetura impecável e suas moradias dentro de montanhas, algo que qualquer
outra raça fica incrédula ao ver.

A sociedade anã não é muito grande. Eles se contentam com suas moradias em
montanhas e só saem de lá sejam impelidos por alguma força externa como ocorreu
na cordilheira do Kaitû anos atrás.

Os anões, em geral, contam os anos desde a Diáspora de Malor, como é conhecida a
antiga metrópole anã. Malor era uma cidade gigantesca que, segundo dizem, foi
criada por guildas anãs antes mesmo das guildas humanas se estabelecerem nas
margens do Jote dentro das montanhas da cordilheira de onde nasce o rio Kaitû,
ao sul da Península do Goto. A cidade era enorme e prosperou por muitos anos. Os
anões eram talvez uma das raças mais ricas do mundo, minerando metais e
materiais precisos que eram vendidos principalmente paras os elfos das florestas
ao sul de Estália e para os humanos do norte da Península do Goto, onde surgia a
cidade de Nize.

Os anões jamais deixariam seu querido lar se não fosse por um desastre até hoje
não completamente explicado. Muitas teorias existem, mas o fato é que algo
aconteceu de forma que os Espíritos do Mal começaram a invadir e brotar dentro
de Malor. Isso chegou a um ponto que ficou insustentável: várias outras
monstruosidades começaram a se aproveitar da fraqueza anã para invadir e se
apossar gradualmente dos salões enormes de Malor.

Dessa forma, a riqueza anã começou a ser toda gasta em uma espécie de guerra
contra inimigos poderosos e numerosos demais até um ponto que eles começaram a
minguar e não ter outra escolha a não ser fugir para outros pontos de Estália.
Hoje ainda existem lendas de anões morando em Malor e lutando bravamente contra
criaturas malignas que se apossaram da metrópole, mas os homens de Calífora.

Atualmente, os anões podem ser encontrados em várias das grandes cidades de
Estália, como Calífora, Nize, Argan, Vandelya... Mas ainda existem duas grandes
cidades verdadeiramente anãs que tentam ser o que Malor foi outrora. A primeira
delas é, na verdade, um conjunto de pequenas cidades muito próximas nas
montanhas de Galalio conhecidas como Minas Galalio. Elas são consideradas
território de Nize e os anões pagam tributos (em minérios e metais) para os
Saicro, mas fora isso, são praticamente independentes. Já a segunda cidade anã
fica mais ao sul, na cordilheira que divide a Península do Goto com o resto do
continente de Estália. Essa cidade é conhecida como Nova Malor e se tornou a
nova capital anã realmente independente. Entretanto, ela não é nem um décimo do
que fora outrora.

Nova Malor hoje também sobrevive principalmente exportando minerais precioso e
metais várias civilizações, mas grande parte de sua riqueza surgiu provendo
pedras para a construção de Calífora durante os últimos trezentos anos. Hoje,
entretanto, seu principal comércio é com os elfos de Valoar, na Floresta de
Argan, e também com os gnomos da Gadia, que hoje utilizam muito metal para suas
engenhocas.

\section*{Gnomos\t\t\t\t\t\t\t\t\t\t\t\t\t\t}%
\label{sec:gnomos}

Os gnomos são criaturinhas minúsculas extremamente inquietas de corpo e
espírito. Eles são nativos da ilha da Gadia, no meio do Entreterras e poucos são
vistos fora de lá.

A verdade é que os gnomos são criaturas bastante misteriosas e incompreendidas
pelo resto das civilizações. Segundo eles próprios, os gnomos são numerosos e
habitam uma dimensão chamada de Bandos, de onde vem todo seu conhecimento
milenar sobre eletricidade e mecânica. Contudo, em Jǒna, poucos realmente
acreditam nessas histórias de erva-de-cheiro.

Entretanto, o que realmente todos sabem é que os gnomos são engenheiros e
cientistas natos. Sua inquietude de fato muitas vezes acaba os matando em
experimentos malucos envolvendo explosões ou qualquer outro tipo de coisa
perigosa, mas também acaba revelando tecnologias almejadas ou temidas por todos
os povos da Estália, como os balões e barcos voadores ou então os gigantes de
ferro que transformam miúdos gnomos em seres maiores que dois elfos.

A sorte dos outros povos é que os gnomos estão mais interessados em aventuras,
explorações e descobertas do que em guerra e, portanto, nunca utilizaram essas
tecnologias contra qualquer um dos povos livres.

Além dessas, outra tecnologia que chama bastante atenção dos não-gnomos é a
capacidade deles de criar seres mecânicos completamente conscientes. É o que
eles denominam automatos. Os automatos são praticamente uma raça por si só,
criada pelos gnomos, mas completamente livre de seus criadores. A opinião dos
demais sobre esses seres robóticos é segmentada entre aqueles que condenam e até
destruiriam esses robôs se não fosse pelo medo de uma rechaça gnoma, como os
elfos e halfings; e aqueles que acham fascinante e gostariam muito de saber
criar coisa parecida, como os humanos e anões. E não só com relação aos
automatos, mas os elfos vêm muitas das criações gnômicas como não-naturais e que
violam a ordem natural das coisas, enquanto que os humanos e anões, inventores
em menor escala, gostariam muito de descobrir os segredos dos gnomos.

Na Gadia, os gnomos não possuem realmente uma cidade ou algo parecido. Eles se
organizam em espécies de vilas com casas enormes (para padrões dos gnomos)
subterrâneas, mas, diferente dos anões, saem constantemente à luz do sol para
explorar a natureza ou fazer seus experimentos ao ar livre. Além disso, apesar
de serem parecidos com os haflings, que fazem as casas subterrâneas, todas as
outras construções dos gnomos são construídas para cima, como fazem os humanos e
elfos: mercados, oficinas, comércios, templos...

E por falar em templo, os gnomos são os detentores de um dos maiores templos de
toda Estália, senão o maior, conhecido como o Templo da Gadia, onde muitas
pessoas de outras raças vão em busca de conhecimento e elevação espiritual.
Todos aqueles que seguem a religião dos cinco deuses regentes possuem um apreço
por esse templo, pois seria onde supostamente os cinco deuses derrotaram Berkas,
o deus caído. Esse templo, então, é um território neutro e cuidado por
sacerdotes de várias raças diferentes. Para lá devem ir aqueles que querem se
tornar Goddion, a ordem religiosa que cultua os cinco deuses e espalha sua fé
por todo o continente.

\section*{Automatos\t\t\t\t\t\t\t\t\t\t\t\t}%
\label{sec:automatos}


\section*{Elfos\t\t\t\t\t\t\t\t\t\t\t\t\t\t\t}%
\label{sec:elfos}

\section*{Halfings\t\t\t\t\t\t\t\t\t\t\t\t\t}%
\label{sec:halfings}

%%%%%%%%%%%%%%%%%%%%%%%%%%%%%%%%%%%%%%%%%%%%%%%%%%%%%%%%%%%%%%%%%%%%%%%%%%%%%%%%

\chapter{Role em Geral}%
\label{cha:role_em_geral}

\section{Cidades\t\t\t\t\t\t\t\t\t\t\t\t}%
\label{sec:cidades}

\blindtext

\section{Linguas novas\t\t\t\t\t\t\t\t}%
\label{sec:linguas_novas}




%%%%%%%%%%%%%%%%%%%%%%%%%%%%%%%%%%%%%%%%%%%%%%%%%%%%%%%%%%%%%%%%%%%%%%%%%%%%%%%%

\chapter{Classes}%
\label{cha:classe}

Todas as classe de SdS possui algumas características em comum que, a menos que
se diga o contrário, funcionam todas da mesma forma.

\section*{Conjuração\t\t\t\t\t\t\t\t\t\t\t}%
\label{sec:conjuracao}

Todas as classes de SdS são conjuradoras natas. Algumas sendo mais eficiente que
outras em fazê-lo.

Essas classes utilizam \textbf{mana} para conjurar magias e só podem conjurar as
magias da \textbf{escola} que possuem.

A tabela a seguir define a quantidade de mana por nível de personagem:

\begin{center}
\begin{tabular}{|||c||c|||}
    \hline
    \textbf{Nível} & \textbf{Mana} \\
    \hline
    \hline
    1 & 500 \\
    \hline
    2 & 1000 \\
    \hline
    3 & 2000 \\
    \hline
    4 & 3000 \\
    \hline
    5 & 5000 \\
    \hline
    6 & 6000 \\
    \hline
    7 & 9000 \\
    \hline
    8 &  ??? \\
    \hline
    9 & ??? \\
    \hline
    10 & ??? \\
    \hline
    11 & ??? \\
    \hline
    12 & ??? \\
    \hline
    13 & ??? \\
    \hline
    14 & ??? \\
    \hline
    15 & ??? \\
    \hline
    16 & ??? \\
    \hline
    17 & ??? \\
    \hline
    18 & ??? \\
    \hline
    19 & ??? \\
    \hline
    20 & ??? \\
    \hline
\end{tabular}
\end{center}

\textit{Os valores acima de 7 não estão definidos, pois os testes serão feitos
primeiro nos níveis iniciais.}

As escolas de magia existente são: \textbf{\textit{Elemental}},
\textbf{\textit{Ilusionista}}, \textbf{\textit{Necromancia}},
\textbf{\textit{Invocação}}, \textbf{\textit{Psíquica}},
\textbf{\textit{Musical}}, \textbf{\textit{Espiritual}},
\textbf{\textit{Atrativa}}, \textbf{\textit{Pura}}.

Abaixo há uma descrição curta de cada uma delas:

\subsection*{Ilusionista}%
\label{sub:ilusionista}

Escola exercida por alguns dos povos do sul, ela contém magias capaz de
dissimular os sentidos, como visão, audição e olfato.

\subsection*{Necromancia}%
\label{sub:necromancia}

Escola banida de todas as grandes cidades de Estália e Cafrilar, a necromancia
utiliza os poderes contidos nas coisas mortas. A Necromancia é uma escola tão
diferente que, quando um necromante existe, ele rapidamente torna-se conhecido
pelos seus feitos esquisitos.

\subsection*{Invocação}%
\label{sub:invocacao}

Assim como a Necromancia, a Invocação é tratada como um tipo de magia maléfica,
em que se traz para a vida criaturas, impondo a ela a vontade do conjurador. Ela
é extremamente rara.

\subsection*{Psíquica}%
\label{sub:psiquica}

Igualmente rara, mas essa é uma das escolas mais cobiçadas dentro de Estália.
Acredita-se que escola de magia Psíquica é a única que não precisa transportar
mana do plano astral para o material, pois ela repercute na própria mente de
cada indivíduo.

Os efeitos são os mais diversos e inesperados.

\subsection*{Musical}%
\label{sub:musical}

Uma escola de magia cobiçada pelos bardos do norte e ensinada a poucos por
algumas seletas guildas de músicos do sul que sabem como controlar mana
utilizando instrumentos musicais ou a própria voz.

É uma escola da magia muito diferente de qualquer outra, capaz de encantar e
curar criaturas, ou então fazê-las enlouquecer e sentir sensações horríveis como
uma magia psíquica é capaz de fazer.

\subsection*{Espiritual}%
\label{sub:espiritual}

Um tipo de magia bastante incomum, que envolve de alguma forma espíritos e
deuses. São poucas as magias conhecidas que se encaixam nessa escola.

\subsection*{Atrativa}%
\label{sub:atrativa}

Igualmente poucas as magias chamadas de Atrativas. Elas criam pontos de atração
que sugam tudo para perto deles. São muito poderosas e capaz de grande
devastação.

\subsection*{Pura}%
\label{sub:pura}

Uma escola realmente lendária e teórica. Uma magia é dita Pura quando emite mana
de forma pura e visível no mundo material. Essas magias utilizam uma quantidade
desumana de mana, mas possuem poder além da compreensão.

\subsection*{Elemental}%
\label{sub:elemental}

Essa sim é a mais comum de todas as escolas de magia e é a única que possui
subdivisões internas chamadas de \textbf{Ordens}.

A escola de magia Elemental explora os elementos do mundo material:
\textbf{Água}, \textbf{Fogo}, \textbf{Terra}, \textbf{Ar}, \textbf{Luz},
\textbf{Sombras}, \textbf{Relâmpago}, \textbf{Veneno} e \textbf{Metal}. As
quatro primeiras ordens são de fato as mais comuns. Luz é uma ordem incomum, mas
muitas casas nobres de Calífora ou Nize possuem magos dessa ordem. Já as quatro
últimas são incrivelmente raras, sendo Sombras a ordem mais rara de todas.

\subsubsection{Água}%
\label{ssub:agua}

A Ordem da Água é capaz de utilizar o poder da água para as mais diversas
tarefas. Um mago da água é capaz desde criar um simples copo d'água quando está
com sede até remover toda a água do corpo de uma criatura, fazendo-a morrer
desidratada.

Magos da Água são conhecidos principalmente por seu poder de cura e
fortalecimento de aliados no campo de batalha. Com um simples toque, um Mago da
Água pode remover ferimentos, deixar um aliado mais veloz e certeiro ou um
inimigo mais fraco.

Mas nem só de utilidade é feito um Mago da Água. Eles podem criar jatos de água
tão velozes e finos que são capaz de cortas árvores em duas ou então criar ondas
gigantes capaz de afetar todo o campo de batalha.

\subsubsection*{Fogo}%
\label{ssub:fogo}

Sem dúvida nenhuma, a palavra chave de um Mago do Fogo é \textit{dano}.

Eles são os magos mais destrutíveis dentro de uma batalha, capazes de impor uma
quantidade absurda de dano a um adversário em um piscar de olhos.

Esses magos são capaz de incendiar o próprio corpo, lançar chamas contra seus
inimigos, criar explosões no meio do campo de batalha ou até mesmo conjurar
enormes bolas de fogo.

\subsubsection*{Terra}%
\label{ssub:terra}

Resiliência e controle são as palavras chave que definem a Ordem da Terra.

Esse magos são capaz de transformar partes do seu corpo em pedra, criar tremores
de terra que derrubam todos no campo de batalha ou então lançar pedras
gigantescas contra inimigos ou muralhas.

A Ordem da Terra também possui formas de fortalecer aliados e até algumas curas.

\subsubsection*{Ar}%
\label{ssub:ar}

Os Magos do Ar são extremamente velozes e ágeis, acabando com os inimigos antes
mesmo de eles perceberem sua presença.

Assim como a Ordem do Fogo, são uma Ordem capaz de causa dano descomunal.
Entretanto, a Ordem do Ar possui também várias técnicas de defesa como
\textit{Parede de Vento} e fortalecimentos que deixam um conjurador intangível.

\subsubsection*{Luz}%
\label{ssub:luz}

Uma Ordem com uma das maiores versatilidades. A Ordem da Luz possui também muito
dano, mas ao mesmo possui algumas das melhores curas e fortalecimentos de
aliados que um conjurador encontra dentro das ordens e escolas mais comuns.

Um Mago da Luz consegue emitir uma luz curativa contra seus aliados, emitir
discos de luz capazes de cortas os inimigos ao meio ou então lançar um flash de
luz capaz de cegar todos.

Além disso, esses magos obviamente conseguem transformar um local escuro em
claro com um simples truque mágico, tornando um objeto luminescente.

\subsubsection*{Sombras}%
\label{ssub:sombras}

Uma Ordem que é um legado dos \textit{Kage-Same} de Calífora. Até hoje, apenas
seis magos são conhecidos por possui esse tipo de magia e, sem dúvida, é um
trunfo militar da cidade mais desenvolvida da Península do Goto.

\subsubsection*{Relâmpago}%
\label{ssub:relampago}

A capacidade de controlar os relâmpagos é rara e não transmitida genéticamente
como as anteriores.

Magos do Relâmpago são capaz de criar lâminas de relâmpagos que podem ser
jogadas contra oponentes, fazendo-os explodir com tanta energia concentrada.

Eles são quase tão ágeis quanto os Magos do Ar e quase tão poderosos quanto os
do Fogo, combinando talvez o melhor dos dois mundos.

\subsubsection*{Veneno}%
\label{ssub:veneno}

Uma Ordem pouco conhecida no Norte, mas muito famosa entre as guildas de
assassinos do sul, Magos do Veneno são mortais.

Em alguns casos, um simples arranhão de uma lâmina sobre efeito de uma magia de
Veneno é capaz de assinar gigante em segundos.

Não só isso, mas eles conseguem cuspir veneno ou fazer ele efeitos dos mais
nojentos e questionáveis.

\subsubsection*{Metal}%
\label{ssub:metal}

Uma Ordem bastante cobiçada por ferreiros, mas incrivelmente rara. Um Mago do
Metal é capaz de criar qualquer tipo de metal em seu próprio corpo ou em lugares
que possa ver.

Magos do Metal transformam a expressão ``perceba a lâmina como uma extensão do
seu corpo'' em algo ultrapassado.

\subsection*{Escolas e Ordens Disponíveis}%
\label{sub:escolas_e_ordens_disponiveis}

Independente da classe escolhida pelo jogador, as ordens da Água, Fogo, Terra,
Ar, Luz, Metal e Veneno são aquelas disponíveis dentro da Escola Elemental, além
das escolas de Ilusionismo e Musical.

As outras ordens e escolas podem ser liberadas pelo personagem conforme ele
avança em sua jornada, dependendo da classe que escolheu. Para liberar essas
ordens, o personagem deve fazer alguma missão especial.

\subsection*{Truques}%
\label{sub:truques}

Todas as classes conhecem truques desde o 1º nível. Já os outros ciclos de magia
estão determinados em cada classe a partir de que nível aquela classe o conhece.

\subsection*{Mana e Conjuração}%
\label{sub:mana_e_conjuracao}

A menos que se diga o contrário, se um conjurador possui apenas $20\%$ de sua
mana restante, então ele começa a fazer testes de Constituição sempre que
utiliza mais mana para qualquer coisa. O teste é 10 + (20\% - porcentagem atual
de mana restante).  Cada falha acarreta em um nível de exaustão.

Caso o conjurador possua apenas $5\%$ de mana restante, então uma falha no teste
acarreta em dois níveis de exaustão.

Independe do nível de exaustão, se um personagem chega em $0$ de mana, ele
morre imediatamente.

\subsection*{Conjuração Armada}%
\label{sub:conjuracao_armada}

Para conjurar magias, o conjurador deve ter um foco em canalizar sua mana pelo
corpo e expressa-la no Mundo Material em forma de magias. Portanto, conjurar
magias com as duas mãos ocupadas (seja por duas armas ou seja por uma arma e um
escudo) ou utilizando uma armadura pesada é muito mais difícil.

A menos que se diga o contrário, conjuradores possem certas desvantagens ao
conjurar magias quando estão com as duas mãos ocupadas ou utilizando armada.

Se uma magia possui dados que devem ser rolados (por qualquer motivo), então você
deve rolar o dobro dos dados citados na magia e pegar os menores dados (por
exemplo, se a magia pede para rolar 3d6, então role 6d6 e pegue os 3 menores).

Se a magia diz que um inimigo deve fazer uma salvaguarda, então o inimigo o faz
com vantagem. Se a magia é qualquer tipo de buff ou debuff (por exemplo,
armaduras e escudos mágicos, magias que impõem desvantagem ao inimigo sem que ele
tenha que passar em salvaguardas, magias que adicionam dados a salvaguardas de
aliados...), então você deve passar em um teste CD 12 menos seu modificador de
conjuração para que a magia realmente seja efetiva.

Se o conjurador está utilizando um foco de conjuração (cajado, varinha ou
similares), então essa mão é considerada como mão livre (por exemplo, ele pode
utilizar cajado e adaga e essa regra não se aplica).

\subsection*{Conjuração de Monstros}%
\label{sub:conjuracao_de_monstros}

Os monstros e inimigos em geral sempre possem a regra Conjuração Armada aplicada
a eles. Podem existir exceções a essa regra, mas provavelmente serão inimigos
mais marcantes (por exemplo, inimigos que possuem uma magia característica que
usa sempre ou então inimigos que possuem fichas de personagens jogáveis).


%%%%%%%%%%%%%%%%%%%%%%%%%%%%%%%%%%%%%%%%%%%%%%%%%%%%%%%%%%%%%%%%%%%%%%%%%%%%%%%%

\chapter{Bárbaro}%
\label{cha:barbaro}

%%%%%%%%%%%%%%%%%%%%%%%%%%%%%%%%%%%%%%%%%%%%%%%%%%%%%%%%%%%%%%%%%%%%%%%%%%%%%%%%

\chapter{Clérigo}%
\label{cha:clerigo}

%%%%%%%%%%%%%%%%%%%%%%%%%%%%%%%%%%%%%%%%%%%%%%%%%%%%%%%%%%%%%%%%%%%%%%%%%%%%%%%%

\chapter{Guerreiro}%
\label{cha:guerreiro}

\section*{Características de Classe\t\t}%
\label{sec:caracteristicas_de_classe}

Como um conjurador, você adquire as seguintes características de classe.

\subsubsection{Pontos de Vida}%
\label{ssub:pontos_de_vida}

\textbf{Dado de Vida}: 1d10 por nível de guerreiro \nl
\textbf{Pontos de Vida no 1º Nível:} 10 + seu modificador de Constituição. \nl
\textbf{Pontos de vida nos Níveis Seguintes:} 1d10 (ou 6) + seu modificador de
Constituição por nível de guerreiro após o 1º.

\subsubsection{Proficiências}%
\label{ssub:proficiencias}

\textbf{Armaduras:} Todas as armaduras, escudos \nl
\textbf{Armas:} Armas simples, armas marciais \nl
\textbf{Ferramentas:} Nenhuma \jump
\textbf{Testes de Resistência}: Força, Constituição \nl
\textbf{Pericias:} Escolha duas entre Acrobacia, Adestrar Animais, Atletismo,
História, Intuição, Intimidação, Percepção e Sobrevivência.

\subsubsection{Equipamento}%
\label{ssub:equipamento}

Você começa com o seguinte equipamento, além do equipamento concedido pelo seu
antecedente.
\begin{itemize}
    \item (a) uma cota de malha ou (b) um gibão de peles, um arco longo e 20
        flechas;
    \item (a) uma arma marcial e um escudo ou (b) duas armas marciais;
    \item (a) uma besta leve e 20 virotes ou (b) dois machados de arremesso;
    \item (a) um pacote de aventureiro ou (b) um pacote de explorador.
\end{itemize}

\section*{Conjuração\t\t\t\t\t\t\t\t\t\t\t}%
\label{sec:conjuracao}

A tabela a seguir mostra em quais níveis o mago tem acesso as magias de cada
ciclo:

\begin{center}
\begin{tabular}{|||c||c|||}
    \hline
    \textbf{Nível} & \textbf{Ciclo} \\
    \hline
    2 & 1º \\
    \hline
    6 & 2º \\
    \hline
    12 & 3º \\
    \hline
    18 & 4º \\
    \hline
\end{tabular}
\end{center}

\begin{obs}
\textit{Essa tabela pode mudar no futuro para fazer o guerreiro utilizar magias
de até 5º ciclo}.
\end{obs}

\subsubsection*{Habilidade de Conjuração}%
\label{ssub:habilidade_de_conjuracao}

Inteligência é a sua habilidade para você conjurar suas magias de mago, pois os
magos aprendem novas magias através de estudo e memorização. Você usa sua
Inteligência sempre que alguma magia se referir a sua habilidade de conjurar
magias. Além disso, você usa o seu modificador de Inteligência para definir a CD
dos testes de resistência para as magias de mago que você conjura e quando você
realiza uma jogada de ataque com uma magia.

\begin{center}
\textbf{CD para suas magias} = 8 + bônus de proficiência + seu modificador de
Inteligência. \nl

\textbf{Modificador de ataque de magia} = seu bônus de proficiência + seu
modificador de Inteligência
\end{center}

\section*{Estilo de Luta\t\t\t\t\t\t\t\t\t}%
\label{sec:estilo_de_luta}

Você adota um estilo de combate particular que será sua especialidade. Escolhe
uma das opções a seguir. Você não pode escolher o mesmo Estilo de Combate mais
de uma vez, mesmo se puder escolher novamente.

\subsubsection{Arquearia}%
\label{ssub:arquearia}

Você ganha +2 de bônus nas jogadas de ataque realizadas com uma arma de ataque à
distância.

\subsubsection{Arremesso de Armas}%
\label{ssub:arremesso_de_armas}

Você pode sacar uma arma que possua a propriedade de arremesso como parte de sua
ação de ataque com essa arma.

Adicionalmente, quando você acerta um ataque à distância usando uma arma de
arremesso, você ganha um bônus de +2 na jogada de dano.

\subsubsection{Ataque Desarmado}%
\label{ssub:ataque_desarmado}

Seus ataques desarmados podem provocar dano contundente igual a 1d6 + seu
modificador de Força em caso de acerto. Se você não estiver empunhando nenhuma
arma ou escudo ao realizar a jogada de ataque, esse 1d6 se transforma em 1d8.

No começo de cada um de seus turnos, você pode causar 1d4 de dano contundente a
uma criatura agarrada por você.

\subsubsection{Combate com Armas Grandes}%
\label{ssub:combate_com_armas_grandes}

Quando você rolar um $1$ ou $2$ num dado de dano de um ataque com uma arma
corpo-a-corpo que você esteja empunhando com duas mãos, você pode rolar o dado
novamente e usar a nova rolagem. A arma deve ter a propriedade duas mãos ou
versátil para ganhar esse benefício.

\subsubsection{Combate com Duas Armas}%
\label{ssub:combate_com_duas_armas}

Quando você estiver engajado em uma luta com duas armas, você pode adicionar o
seu modificador de habilidade de dano na jogada de dano de seu segundo ataque.

\subsubsection{Defesa}%
\label{ssub:defesa}

Enquanto estiver usando armadura, você ganha +1 de bônus em sua CA.

\subsubsection{Duelismo}%
\label{ssub:duelismo}

Quando você empunhar uma arma de ataque corpo-a-corpo em uma mão e nenhuma outra
arma, você ganha +2 de bônus nas jogadas de dano com essa arma.

\subsubsection{Interceptador}%
\label{ssub:interceptador}

Quando uma criatura que você possa ver acerta um alvo que não seja você, a até
1,5m de distância de você com um ataque, você pode usar sua reação para reduzir
o dano recebido por esse alvo em 1d10 + seu bônus de proficiência (até um mínimo
de 0 de dano). Você deve estar empunhando um escudo ou uma arma simples ou
marcial para usar essa reação.

\subsubsection{Luta as Cegas}%
\label{ssub:luta_as_cegas}

Você tem percepção às cegas com um alcance de 3m. Dentro desse alcance, você
pode efetivamente ver qualquer coisa que não esteja sob cobertura total, mesmo
se você estiver cego ou na escuridão. Além disso, você pode ver uma criatura
invisível nessa área, a menos que a criatura se esconda de você com sucesso.

\subsubsection{Proteção}%
\label{ssub:protecao}

Quando uma criatura que você possa ver atacar um alvo que esteja a até 1,5 metro
de você, você pode usar sua reação para impor desvantagem nas jogadas de ataque
da criatura. Você deve estar empunhando um escudo.

\subsubsection{Técnica Superior}%
\label{ssub:tecnica_superior}

Você aprende uma manobra a sua escolha dentre aquelas disponíveis para o
arquétipo do Mestre de Batalha. Se uma manobra que você utilizar exigir que o
alvo realize uma jogada de salvaguarda para resistir ao efeito dela, a CD da
salvaguarda será igual a 8 + seu bônus de proficiência + seu modificador de
Força ou Destreza (o que você preferir).

Você ganha um dado de superioridade, que é 1d6 (esse dado é adicionado a
quaisquer dados de superioridade que você tenha de outra fonte). Esse dado é
usado para abastecer suas manobras. Um dado de superioridade é gasto quando você
o utiliza. Você recupera os dados gastos quando você termina um descanso curto
ou longo.

\section*{Retomar o Fôlego\t\t\t\t\t\t\t}%
\label{sec:retomar_o_folego}

Você possui uma reserva de estamina e pode usá-la para proteger a si mesmo
contra danos. No seu turno, você pode usar uma ação bônus para recuperar pontos
de vida igual a 1d10 + seu nível de guerreiro.

Uma vez que você use essa característica, você precisa terminar um descanso
curto ou longo para usá-la de novo.

\section*{Surto de Ação\t\t\t\t\t\t\t\t\t\t}%
\label{sec:surto_de_acao}

A partir do 2º nível, você pode forçar o seu limite para além do normal por um
momento. Durante o seu turno, você pode realizar uma ação adicional juntamente
com sua ação e possível ação bônus.

Uma vez que você use essa característica, você precisa terminar um descanso
curto ou longo para usá-la de novo.

A partir do 17º nível, você pode usá-la duas vezes antes do descanso, porém
somente uma vez por turno.

\section*{Arquétipo Marcial\t\t\t\t\t\t\t}%
\label{sec:arquetipo_marcial}

No 3º nível, você escolhe um arquétipo o qual se esforçará para seguir as
técnicas e estilos de combate dele. Todos os arquétipos estão detalhados no
final da descrição da classe. O arquétipo confere a você características
especial no 3º nível e de novo nos 7º, 10º, 15º e 18º nível.

\section*{Incremento em Habilidade\t\t}%
\label{sec:incremento_em_habilidade}

Quando você atinge o 4º nível e novamente no 8º, 12º, 16º e 19º nível, você pode
ganha dois pontos de habilidades para distribuir entre suas habilidades. Por
padrão, você não pode elevar um valor de habilidade acima de 20 com essa
característica.

Opcionalmente, você pode escolher um talento seguindo as regras de talentos.

\section*{Ataque Extra\t\t\t\t\t\t\t\t\t\t}%
\label{sec:ataque_extra}

A partir do 5º nível, você pode atacar duas vezes, ao invés de uma, quando usar
a ação Atacar durante o seu turno.

O número de ataques aumenta para três quando você alcançar o 11º nível de
guerreiro e para 4 quando alcançar o 20º nível de guerreiro.

\section*{Indomável\t\t\t\t\t\t\t\t\t\t\t\t}%
\label{sec:indomavel}

A partir do 9º nível, você pode jogar de novo um teste de resistência que
falhou. Se o fizer, você deve usar o novo valor e não pode usar essa
característica de novo antes de terminar um descanso longo.

Você pode usar esta característica duas vezes entre descansos longos quando
chegar no 13º nível e três vezes entre descansos longos quando chegar no 17º
nível.

\section*{Arquétipos Marciais\t\t\t\t\t\t}%
\label{sec:arquetipos_marciaistttttt}

Diferentes guerreiros escolhem diferentes caminhos para aperfeiçoar seu poder em
combate. O arquétipo marcial que você escolhe seguir reflete essa escolha.

\subsection*{Campeão}%
\label{sub:campeao}

O arquétipo Campeão foca no desenvolvimento da pura força física acompanhada por
uma perfeição mortal.

Aqueles que trilham o caminho desse arquétipo combinam rigorosos treinamentos
com excelência física para desferir golpes devastadores.

\subsubsection{Crítico Aprimorado}%
\label{ssub:critico_aprimorado}

A partir do 3º nível, seus ataques com armas adquirem uma margem de acerto
crítico de 19 a 20 nas jogadas de ataque.

\subsubsection{Atletismo Extraordinário}%
\label{ssub:atletismo_extraordinario}

A partir do 7º nível, você adiciona metade de seu bônus de proficiência
(arredondado para cima) em qualquer teste de Força, Destreza ou Constituição que
você já não aplique seu bônus de proficiência.

Além disso, quando você fizer um salto longo com corrida, o alcance em metros
que poderá saltar aumenta em 0,3 vezes o seu modificador de Força.

\subsubsection{Estilo de Luta Adicional}%
\label{ssub:estilo_de_luta_adicional}

No 10º nível, você pode escolher um segundo Estilo de Combate da sua
característica de classe.

\subsubsection{Crítico Superior}%
\label{ssub:critico_superior}

A partir do 15º nível, seus ataques com armas adquirem uma margem de acerto
crítico de 18 a 20 nas jogadas de ataque.

\subsubsection{Sobrevivente}%
\label{ssub:sobrevivente}

No 18º nível, você alcança o topo da resiliência em batalha. No começo de cada
um de seus turnos, você recupera uma quantidade de pontos de vida igual a 5 +
seu modificador de Constituição se não estiver com mais que metade de seus
pontos de vida. Você não recebe esse benefício se estiver com 0 pontos de vida.

\subsection*{Mestre de Batalha}%
\label{sub:mestre_de_batalha}

Aqueles que emulam o arquétipo de Mestre de Batalha empregam técnicas marciais
passadas de geração em geração. Para um Mestre de Batalha, o combate é um campo
acadêmico, as vezes, incluindo assuntos além da batalha, como forjaria e
caligrafia. Nem todo guerreiro absorve as lições de história, teoria e arte que
são refletidas no arquétipo de Mestre de Batalha, mas aqueles que conseguem,
tornam-se guerreiros bem-supridos de grande perícia e conhecimento.

\subsubsection{Superioridade em Combate}%
\label{ssub:superioridade_em_combate}

Quando você escolhe esse arquétipo, no 3° nível, você aprende manobras que são
abastecidas com dados especiais chamados dados de superioridade.

\textbf{Manobras.} Você aprende três manobras, à sua escolha, que são detalhadas
em ``Manobras'', a seguir.

Muitas manobras aprimoram um ataque de várias formas.  Você só pode usar uma
manobra por ataque.

Você aprende duas manobras adicionais, à sua escolha, no 7°, 10° e 15° nível. A
cada vez que você aprende uma nova manobra, você pode substituir uma manobra
conhecida por uma diferente.

\textbf{Dados de Superioridade.} Você tem quatro dados de superioridade, que são
d8s. Um dado de superioridade é gasto quando você usa-o. Você recupera todos os
dados de superioridade gastos quando terminar um descanso curto ou longo.

Você adquire outro dado de superioridade no 7° nível e mais um no 15° nível.

\textbf{Teste de Resistência.} Algumas das suas manobras exigem que o alvo
realize um teste de resistência contra o efeito da manobra. A CD do teste de
resistência é calculada a seguir:
\begin{center}
\textbf{CD para suas manobras} = 8 + bônus de proficiência + seu modificador de
Força ou Destreza (à sua escolha).
\end{center}

\subsubsection{Estudioso da Guerra}%
\label{ssub:estudioso_da_guerra}

No 3° nível, você ganha proficiência com um tipo de ferramenta de artesão, à sua
escolha.

\subsubsection{Conheça seu Inimigo}%
\label{ssub:conheca_seu_inimigo}

A partir do 7° nível, se você gastar, pelo menos, 1 minuto observando ou
interagindo com outra criatura fora de combate, você pode aprender certas
informações sobre as capacidades dela comparadas as suas. O Mestre conta a você
se a criatura é igual, superior ou inferior a você a respeito de duas das
seguintes características, à sua escolha:

\begin{itemize}
    \item Valor de Força
    \item Valor de Destreza
    \item Valor de Constituição
    \item Classe de Armadura
    \item Pontos de Vida atuais
    \item Nível total de classe (se possuir)
    \item Níveis da classe guerreiro (se possuir)
\end{itemize}

\subsubsection{Superiorridade em Combate Aprimorada}%
\label{ssub:superiorridade_em_combate_aprimorada}

No 10° nível, seus dados de superioridade se tornam d10s.

No 18° nível, eles se tornam d12s.

\subsubsection{Implacável}%
\label{ssub:implacavel}

No 15º nível, quando você rolar iniciativa e não tiver nenhum dado de
superioridade restante, você recupera um dado de superioridade.

\subsubsection{Manobras}%
\label{ssub:manobras}

As manobras são apresentadas em ordem alfabética.

\paragraph{Aparar.} Quando outra criatura causar dano a você com um ataque
corpo-a-corpo, você pode usar sua reação e gastar um dado de superioridade para
reduzir o dano pelo número rolado no dado de superioridade + seu modificador de
Destreza.

\paragraph{Ataque Ameaçador.} Quando você atingir uma criatura com um ataque com
arma, você pode gastar um dado de superioridade para tentar amedrontar o alvo.
Você adiciona seu dado de superioridade a jogada de dano do ataque e o alvo deve
realizar um teste de resistência de Sabedoria. Se falhar, ele ficará com medo de
você até o final do seu próximo turno.

\paragraph{Ataque de Encontrão.} Quando você atingir uma criatura com um ataque
com arma, você pode gastar um dado de superioridade para tentar empurrar o alvo
para trás. Você adiciona seu dado de superioridade a jogada de dano do ataque e,
se o alvo for Grande ou menor, ele deve realizar um teste de resistência de
Força. Se falhar, você empurra o alvo para até 4,5 metros de você.

\paragraph{Ataque de Finta.} Você pode gastar um dado de superioridade e usar
uma ação bônus, no seu turno, para fintar, escolhendo uma criatura a 1,5 metro
de você como alvo. Você tem vantagem na sua próxima jogada de ataque contra essa
criatura, nesse turno. Se o ataque atingir, você adiciona seu dado de
superioridade ao dano do ataque.

\paragraph{Ataque de Manobra.} Quando você atingir uma criatura com um ataque
com arma, você pode gastar um dado de superioridade para tentar manobrar um de
seus companheiros para uma posição mais vantajosa. Você adiciona seu dado de
superioridade a jogada de dano do ataque e escolhe uma criatura amigável que
possa ver ou ouvir você. Aquela criatura pode usar sua reação para se mover até
metade do seu deslocamento, sem provocar ataques de oportunidade do alvo do seu
ataque.

\paragraph{Ataque de Precisão.} Quando você realizar uma jogada de ataque com
arma contra uma criatura, você pode gastar um dado de superioridade para
adicioná-lo a jogada. Você pode usar essa manobra antes ou depois de realizar a
jogada de ataque, mas deve usá-la antes de qualquer efeito do ataque ser
aplicado.

\paragraph{Ataque Desarmante.} Quando você atingir uma criatura com um ataque
com arma, você pode gastar um dado de superioridade para tentar desarmar o alvo,
forçando-o a derrubar um item, à sua escolha, que ele esteja empunhando. Você
adiciona o dado de superioridade a jogada de dano do ataque e o alvo deve
realizar um teste de resistência de Força. Se fracassar, ele derrubará o objeto
escolhido. O objeto cai aos pés dele.

\paragraph{Ataque Estendido.} Quando você atingir uma criatura com um ataque
corpo-a-corpo com arma, você pode gastar um dado de superioridade para aumentar
o alcance do seu ataque em 1,5 metro. Se você atingir, você adiciona o seu dado
de superioridade ao dano causado pelo ataque.

\paragraph{Ataque Provocante.} Quando você atingir uma criatura com um ataque
com arma, você pode gastar um dado de superioridade para tentar incitar a alvo a
atacar você. Você adiciona seu dado de superioridade a jogada de dano do ataque
e o alvo deve realizar um teste de resistência de Sabedoria. Se falhar, o alvo
terá desvantagem em todas as jogadas de ataque contra alvos diferentes de você,
até o fim do seu próximo turno.

\paragraph{Ataque Trespassante.} Quando você atingir uma criatura com um ataque
corpo-a-corpo com arma, você pode gastar um dado de superioridade para tentar
causar dano a outra criatura com o mesmo ataque. Escolha uma criatura a 1,5
metro do alvo original e que esteja no seu alcance. Se a jogada de ataque
original atingiria a segunda criatura, ela sofre dano igual ao número rolado no
dado de superioridade. O dano é do mesmo tipo que o causado pelo ataque
original.

\paragraph{Avaliação Tática} Quando você fizer um teste de Inteligência
(Investigação), Inteligência (História) ou Sabedoria (Intuição), você pode
gastar um dado de superioridade e adicioná-lo a esse teste.

\paragraph{Contra-Atacar.} Quando uma criatura atacar você com um ataque
corpo-a-corpo e errar, você pode usar sua reação e gastar um dado de
superioridade para realizar um ataque corpo-a-corpo com arma contra essa
criatura.  Se você atingir, você adiciona seu dado de superioridade a jogada de
dano do ataque.

\paragraph{Derrubar.} Quando você atingir uma criatura com um ataque com arma,
você pode gastar um dado de superioridade para tentar derrubar o alvo no chão.
Você adiciona seu dado de superioridade a jogada de dano do ataque e, se o alvo
for Grande ou menor, ele deve realizar um teste de resistência de Força. Se
falhar, o alvo ficará caído no chão.

\paragraph{Emboscada} Quando você realizar um teste de Destreza (Furtividade) ou
uma jogada de iniciativa, você pode gastar um dado de superioridade e adicionar
o valor do dado na rolagem, desde que você não esteja incapacitado.

\paragraph{Engodo} Quando você estiver a até 1,5m de uma criatura em seu turno,
você pode gastar um dado de superioridade e trocar de lugar com essa criatura,
desde que você gaste pelo menos 1,5m de movimento e a criatura seja voluntária e
não esteja incapacitada. Esse movimento não provoca ataques de oportunidade.
Jogue o dado de superioridade. Até o começo do seu próximo turno, você ganha um
bônus na CA igual ao número rolado.

\paragraph{Enganchar} Quando uma criatura que você possa ver se move dentro do
alcance que você possui com uma arma corpo-a-corpo que você está empunhando,
você pode usar sua reação para gastar um dado de superioridade e realizar um
ataque contra essa criatura, usando essa arma. Se o ataque acertar, adicione o
dado de superioridade à jogada de dano da arma.

\paragraph{Golpe Distrativo.} Quando você atingir uma criatura com um ataque com
arma, você pode gastar um dado de superioridade para tentar distrair a criatura,
abrindo uma brecha para um de seus aliados. Você adiciona seu dado de
superioridade a jogada de dano do ataque. A próxima jogada de ataque realizada
contra o alvo por uma criatura diferente de você, tem vantagem, se o ataque for
realizado antes do começo do seu próximo turno.

\paragraph{Golpe do Comandante.} Quando você realiza a ação de Ataque, no seu
turno, você pode desistir de um dos seus ataques e usar uma ação bônus para
direcionar o ataque de um dos seus companheiros. Quando você faz isso, escolha
uma criatura amigável que possa ver ou ouvir você e gaste um dado de
superioridade. Essa criatura pode, imediatamente, usar sua reação para realizar
um ataque com arma, adicionando seu dado de superioridade a jogada de dano do
ataque.

\paragraph{Inspirar.} No seu turno, você pode usar uma ação bônus e gastar um
dado de superioridade para reforçar a determinação dos seus companheiros. Quando
o fizer, escolha uma criatura amigável que possa ver ou ouvir você. Essa
criatura ganha uma quantidade de pontos de vida temporários igual a sua rolagem
de dado de superioridade + seu modificador de Carisma.

\paragraph{Passo Evasivo.} Quando você se mover, você pode gastar um dado de
superioridade, role o dado e adicione o número rolado a sua CA até você terminar
seu deslocamento.

\paragraph{Presença Dominante} Quando você realizar um teste de Carisma
(Intimidação, Performance ou Persuasão), você pode gastar um dado de
superioridade e adicionar o resultado dele a esse teste.

\paragraph{Golpe Imobilizador} Imediatamente após acertar uma criatura com um
ataque corpo-a-corpo em seu turno, você pode gastar um dado de superioridade e
então tentar agarrar o alvo como uma ação bônus (veja o Livro do Jogador para as
regras sobre Agarrar). Adicione o dado de superioridade ao seu teste de Força
(Atletismo).

\paragraph{Lançamento Rápido} Como uma ação bônus, você pode gastar um dado de
superioridade e realizar um ataque com uma arma que tenha a propriedade de
arremesso. Você pode sacar a arma como parte dessa ação de ataque. Se você
acertar, adicione o dado de superioridade na jogada de dano da arma.

%%%%%%%%%%%%%%%%%%%%%%%%%%%%%%%%%%%%%%%%%%%%%%%%%%%%%%%%%%%%%%%%%%%%%%%%%%%%%%%%

\chapter{Ladino}%
\label{cha:ladino}

\section*{Características de Classe\t\t}%
\label{sec:caracteristicas_de_classe}

Como um mago, você adquire as seguintes características de classe.

\subsubsection{Pontos de Vida}%
\label{ssub:pontos_de_vida}

\textbf{Dado de Vida}: 1d8 por nível de ladino \nl
\textbf{Pontos de Vida no 1º Nível:} 8 + seu modificador de Constituição. \nl
\textbf{Pontos de vida nos Níveis Seguintes:} 1d8 (ou 5) + seu modificador de
Constituição por nível de ladino após o 1º.

\subsubsection{Proficiências}%
\label{ssub:proficiencias}

\textbf{Armaduras:} Armaduras leves \nl
\textbf{Armas:} Armas simples, bestas de mão, espadas longas, rapieiras, espadas
curtas\nl
\textbf{Ferramentas:} Ferramentas de ladrão \jump
\textbf{Testes de Resistência}: Destreza, Inteligência \nl
\textbf{Pericias:} Escolha quatro entre Acrobacia, Atletismo, Atuação,
Enganação, Furtividade, Intimidação, Intuição, Investigação, Percepção,
Persuasão e Prestidigitação.

\subsubsection{Equipamento}%
\label{ssub:equipamento}

Você começa com o seguinte equipamento, além do equipamento concedido pelo seu
antecedente.
\begin{itemize}
    \item (a) uma rapieira ou (b) uma espada longa;
    \item (a) um arco curto e uma aljava com 20 flechas ou (b) uma espada curta;
    \item (a) um pacote de assaltante ou (b) um pacote de aventureiro ou (c) um
        pacote de explorador;
    \item Armadura de couro, duas adagas e ferramentas de ladrão.
\end{itemize}

\section*{Conjuração\t\t\t\t\t\t\t\t\t\t\t}%
\label{sec:conjuracao}

A tabela a seguir mostra em quais níveis o mago tem acesso as magias de cada
ciclo:

\begin{center}
\begin{tabular}{|||c||c|||}
    \hline
    \textbf{Nível} & \textbf{Ciclo} \\
    \hline
    2 & 1º \\
    \hline
    6 & 2º \\
    \hline
    12 & 3º \\
    \hline
    18 & 4º \\
    \hline
\end{tabular}
\end{center}

\begin{obs}
\textit{Essa tabela pode mudar no futuro para fazer o ladino utilizar magias
de até 5º ciclo}.
\end{obs}

\subsubsection*{Habilidade de Conjuração}%
\label{ssub:habilidade_de_conjuracao}

Inteligência é a sua habilidade para você conjurar suas magias de mago, pois os
magos aprendem novas magias através de estudo e memorização. Você usa sua
Inteligência sempre que alguma magia se referir a sua habilidade de conjurar
magias. Além disso, você usa o seu modificador de Inteligência para definir a CD
dos testes de resistência para as magias de mago que você conjura e quando você
realiza uma jogada de ataque com uma magia.

\begin{center}
\textbf{CD para suas magias} = 8 + bônus de proficiência + seu modificador de
Inteligência. \nl

\textbf{Modificador de ataque de magia} = seu bônus de proficiência + seu
modificador de Inteligência
\end{center}

\section*{Especialização\t\t\t\t\t\t\t\t\t}%
\label{sec:especializacaottttttttt}

No 1º nível, você escolhe duas de suas perícias que seja proficiente, ou uma
perícia que seja proficiente e ferramentas de ladrão. Seu bônus de proficiência
é dobrado em qualquer teste de habilidade que fizer com elas.

No 6º nível, você pode escolher outras duas de suas proficiências (em perícias
ou ferramentas de ladrão) para ganhar esse benefício.

\section*{Ataque Furtivo\t\t\t\t\t\t\t\t\t}%
\label{sec:ataque_furtivo}

A partir do 1º nível, você sabe como atacar sutilmente e explorar a distração de
seus inimigos. Uma vez por turno, você pode adicionar 1d6 nas jogadas de dano
contra qualquer criatura que acertar, desde que tenha vantagem nas jogadas de
ataque. O ataque deve ser com uma arma de acuidade ou à distância.

Você não precisa ter vantagem nas jogadas de ataque se outro inimigo do seu alvo
estiver a 1,5 metro de distância dele, desde que este inimigo não esteja
incapacitado e você não tenha desvantagem nas jogadas de ataque.

A quantidade de dano extra aumenta conforme você ganha níveis nessa classe, como
mostrado na coluna Ataque Furtivo da tabela O Ladino.

\section*{Gíria de Ladrão\t\t\t\t\t\t\t\t}%
\label{sec:giria_de_ladrao}

Durante seu treinamento você aprendeu as gírias de ladrão, um misto de dialeto,
jargão e códigos secretos que permitem você passar mensagens secretas durante
uma conversa aparentemente normal. Somente outra criatura que conheça essas
gírias de ladrão entende as mensagens.

Leva-se quatro vezes mais tempo para transmitir essa mensagem do que falar a
mesma ideia claramente.

Além disso, você entende um conjunto de sinais secretos e símbolos usados para
transmitir mensagens curtas e simples, como saber se uma área é perigosa ou se é
território de uma guilda de ladrões, se o saque está próximo, se as pessoas na
área são alvos fáceis ou até mesmo indicar lugares seguros para ladinos se
esconderem.

\section*{Ação Ardilosa\t\t\t\t\t\t\t\t\t\t}%
\label{sec:acao_ardilosa}

A partir do 2º nível, seu pensamento rápido e agilidade faz você se mover e agir
rapidamente. Você pode usar uma ação bônus durante cada um de seus turnos em
combate. Esta ação pode ser usada somente para Disparada, Desengajar ou
Esconder.

\section*{Arquétipo de Ladino\t\t\t\t\t\t}%
\label{sec:arquetipo_de_ladino}

No 3º nível, você escolhe um arquétipo que se esforçará para se equiparar
através de exercícios de suas habilidades de ladino. Sua escolha garante a você
características no 3º nível e de novo no 9º, 13º e 17º nível. As possíveis
escolhas estão detalhadas no final da descrição da classe.

\section*{Incremento em Habilidade\t\t}%
\label{sec:incremento_em_habilidade}

Quando você atinge o 4º nível e novamente no 8º, 12º, 16º e 19º nível, você pode
ganha dois pontos de habilidades para distribuir entre suas habilidades. Por
padrão, você não pode elevar um valor de habilidade acima de 20 com essa
característica.

Opcionalmente, você pode escolher um talento seguindo as regras de talentos.

\section*{Esquiva Sobrenatural\t\t\t\t\t}%
\label{sec:esquiva_sobrenatural}

A partir do 5º nível, quando um inimigo que você possa ver o acerta com um
ataque, você pode usar sua reação para reduzir pela metade o dano sofrido.

\section*{Evasão\t\t\t\t\t\t\t\t\t\t\t\t\t\t}%
\label{sec:evasao}

A partir do 7º nível, você pode esquivar-se agilmente de certos efeitos em área,
como o sopro flamejante de um dragão vermelho ou uma magia tempestade de gelo.

Quando você for alvo de um efeito que exige um teste de resistência de Destreza
para sofrer metade do dano, você não sofre dano algum se passar, e somente
metade do dano se falhar.

\section*{Talento Confiável\t\t\t\t\t\t\t}%
\label{sec:talento_confiavel}

No 11º nível, você refinou suas perícias beirando à perfeição. Toda vez que você
fizer um teste de habilidade no qual possa adicionar seu bônus de proficiência,
você trata um resultado no d20 de 9 ou menor como um 10.

\section*{Sentido Cego\t\t\t\t\t\t\t\t\t\t}%
\label{sec:sentido_cego}

No 14º nível, se você for capaz de ouvir, você está ciente da localização de
qualquer criatura escondida ou invisível a até 3 metros de você

\section*{Mente Escorregadia\t\t\t\t\t\t}%
\label{sec:mente_escorregadia}

No 15º nível, você adquire uma grande força de vontade, adquirindo proficiência
nos testes de resistência de Sabedoria.

\section*{Elusivo\t\t\t\t\t\t\t\t\t\t\t\t\t\t}%
\label{sec:elusivo}

A partir do 18º nível, você se torna tão sagaz que raramente alguém encosta a
mão em você. Nenhuma jogada de ataque tem vantagem contra você, desde que você
não esteja incapacitado.

\section*{Golpe de Sorte\t\t\t\t\t\t\t\t\t}%
\label{sec:golpe_de_sorte}

No 20º nível, você adquire um dom incrível para ter sucesso nos momentos em que
mais precisa. Se um ataque seu falhar contra um alvo ao seu alcance, você pode
transformar essa falha em um acerto. Ou se falhar em um teste qualquer, você
pode tratar a jogada desse mesmo teste como 20 natural.

Uma vez que você use essa característica, você não pode fazê-lo de novo até
terminar um descanso curto ou longo.

\section*{Arquétipos de Ladino\t\t\t\t\t}%
\label{sec:arquetipos_de_ladino}

Ladinos possuem muitas características em comum, incluindo a ênfase no
aperfeiçoamento de suas perícias, na precisão e aproximação mortal em combate, e
nos seus reflexos cada vez mais rápidos. Mas, diferentes ladinos orientam seus
talentos em direções variadas, personificadas pelos vários arquétipos de ladino.
Seu arquétipo escolhido reflete o seu foco – não necessariamente a indicação de
sua profissão, mas a descrição de suas técnicas preferidas.

\subsection*{Assassino}%
\label{sub:assassino}

Você focou seu treinamento na macabra arte da morte.

Aqueles que devotam-se a esse arquétipo são diversos: assassinos de aluguel,
espiões, caçadores de recompensa e, até mesmo, padres especialmente treinados em
exterminar os inimigos das suas divindades. Subterfúgio, veneno e disfarces
ajudam você a eliminar seus oponentes com eficiência mortífera.

\subsubsection{Proficiência Adicional}%
\label{ssub:proficiencia_adicional}

Quando você escolhe esse arquétipo, no 3º nível, você ganha proficiência com kit
de disfarce e kit de venenos.

\subsubsection{Assassinar}%
\label{ssub:assassinar}

A partir do 3° nível, você fica mais mortal quando pega seus oponentes
desprevenidos. Você tem vantagem nas jogadas de ataque contra qualquer criatura
que ainda não tenha chegado ao turno dela no combate. Além disso, qualquer
ataque que você fizer contra essa criatura que está surpresa, será um ataque
crítico.

\subsubsection{Especialização em Infiltração}%
\label{ssub:especializacao_em_infiltracao}

A partir do 9° nível, você pode infalivelmente, criar identidades falsas para si
mesmo. Você deve gastar sete dias e 25 po para estabelecer o histórico,
profissão e filiações para uma identidade. Você não pode estabelecer uma
identidade que pertença a alguém. Por exemplo, você deveria adquirir roupas
apropriadas, cartas de introdução e um certificado, aparentemente oficial, para
estabelecer-se como um membro da casa de comércio de uma cidade remota, assim,
você poderia introduzir-se na companhia de outros comerciantes abastados.

Posteriormente, se você adotar a nova identidade como disfarce, outras criaturas
acreditarão que você é aquela pessoa, até terem algum motivo obvio para pensarem
o contrário.

\subsubsection{Impostor}%
\label{ssub:impostor}

No 13° nível, você adquire a habilidade de imitar a fala, escrita e
comportamento de outra pessoa, infalivelmente.

Você deve gastar pelo menos três horas estudando esses três componentes do
comportamento de uma pessoa, ouvindo sua articulação, examinando sua escrita e
observando seus maneirismos.

Seu ardil é imperceptível para um observador casual. Se uma criatura desconfiada
suspeitar que algo está errado, você tem vantagem em qualquer teste de Carisma
(Enganação) que você fizer para evitar ser detectado.

\subsubsection{Golpe Letal}%
\label{ssub:golpe_letal}

No 17° nível, você se torna um mestre da morte instantânea. Quando você atacar e
atingir uma criatura que esteja surpresa, ela deve realizar um teste de
resistência de Constituição (CD 8 + seu modificador de Destreza + seu bônus de
proficiência). Se ela falhar, dobre o dano do seu ataque contra a criatura.

\subsection*{Ladrão}%
\label{sub:ladrao}

Você aprimorou suas habilidades na arte do furto de pequenos itens. Gatunos,
bandidos, batedores de carteira e outros criminosos geralmente seguem esse
arquétipo, mas também aqueles ladinos que preferem se ver como caçadores de
tesouro profissionais, exploradores de masmorras e investigadores. Além de
aprimorar sua agilidade e furtividade, você aprende perícias úteis para
desbravar ruínas antigas, ler idiomas incomuns e usar itens mágicos que
normalmente não poderia.

\subsubsection{Mãos Rápidas}%
\label{ssub:maos_rapidas}

A partir do 3º nível, você pode usar a sua ação bônus concedida pela Ação
Ardilosa para fazer um teste de Destreza (Prestidigitação), usar suas
ferramentas de ladrão para desarmar uma armadilha ou abrir uma fechadura, ou
realizar a ação de Usar um Objeto.

\subsubsection{Andarilho de Telhados}%
\label{ssub:andarilho_de_telhados}

No 3º nível, você adquire a habilidade de escalar mais rápido que o normal.
Escalar agora não possui custo adicional de movimento para você.

Além disso, quando você fizer um salto com corrida, o alcance que pode saltar
aumenta um número de metros igual a 0,3 vezes o seu modificador de Destreza.

\subsubsection{Furtividade Surpresa}%
\label{ssub:furtividade_surpresa}

A partir do 9º nível, você tem vantagem no teste de Destreza (Furtividade) se
você não mover-se mais do que a metade de seu deslocamento em um turno.

\subsubsection{Usar Instrumento Mágico}%
\label{ssub:usar_instrumento_magico}

 No 13º nível, você aprende o suficiente sobre como a magia funciona e pode
 improvisar o uso de itens que nem mesmo foram destinados a você. Você ignora
 todos os requisitos de classes, raças e níveis para uso de qualquer item
 mágico.

\subsubsection{Reflexo de Ladrão}%
\label{ssub:reflexo_de_ladrao}

Quando atinge o 17º nível, você se torna adepto em fazer emboscadas e fugas
rápidas de situações perigosas. Você pode realizar dois turnos durante o
primeiro turno de cada combate. Você realiza seu primeiro turno na sua
iniciativa e o segundo na ordem de sua iniciativa menos 10.

Você não pode usar essa característica quando está surpreso.

%%%%%%%%%%%%%%%%%%%%%%%%%%%%%%%%%%%%%%%%%%%%%%%%%%%%%%%%%%%%%%%%%%%%%%%%%%%%%%%%

\chapter{Mago}%
\label{cha:mago}

\section*{Características de Classe\t\t}%
\label{sec:caracteristicas_de_classe}

Como um mago, você adquire as seguintes características de classe.

\subsubsection{Pontos de Vida}%
\label{ssub:pontos_de_vida}

\textbf{Dado de Vida}: 1d6 por nível de mago \nl
\textbf{Pontos de Vida no 1º Nível:} 6 + seu modificador de Constituição. \nl
\textbf{Pontos de vida nos Níveis Seguintes:} 1d6 (ou 4) + seu modificador de
Constituição por nível de mago após o 1º.

\subsubsection{Proficiências}%
\label{ssub:proficiencias}

\textbf{Armaduras:} Nenhuma \nl
\textbf{Armas:} Armas simples \nl
\textbf{Ferramentas:} Nenhuma \jump
\textbf{Testes de Resistência}: Inteligência, Sabedoria \nl
\textbf{Pericias:} Escolha duas entre Arcanismo, Atuação, História, Intimidação,
Intuição, Investigação, Lidar com Animais, Medicina, Persuasão e Religião.

\subsubsection{Equipamento}%
\label{ssub:equipamento}

Você começa com o seguinte equipamento, além do equipamento concedido pelo seu
antecedente.
\begin{itemize}
    \item (a) um bordão ou (b) uma adaga ou (c) um arco curto;
    \item (a) uma bolsa de componentes ou (b) um foco arcano;
    \item (a) um pacote de estudioso ou (b) um pacote de explorador;
\end{itemize}

\section*{Conjuração\t\t\t\t\t\t\t\t\t\t\t}%
\label{sec:conjuracao}

A tabela a seguir mostra em quais níveis o mago tem acesso as magias de cada
ciclo:

\begin{center}
\begin{tabular}{|||c||c|||}
    \hline
    \textbf{Nível} & \textbf{Ciclo} \\
    \hline
    1 & 1º \\
    \hline
    3 & 2º \\
    \hline
    5 & 3º \\
    \hline
    7 & 4º \\
    \hline
    9 & 5º \\
    \hline
    11 & 6º \\
    \hline
    13 & 7º \\
    \hline
    15 & 8º \\
    \hline
    17 & 9º \\
    \hline
\end{tabular}
\end{center}

\subsubsection*{Habilidade de Conjuração}%
\label{ssub:habilidade_de_conjuracao}

Inteligência é a sua habilidade para você conjurar suas magias de mago, pois os
magos aprendem novas magias através de estudo e memorização. Você usa sua
Inteligência sempre que alguma magia se referir a sua habilidade de conjurar
magias. Além disso, você usa o seu modificador de Inteligência para definir a CD
dos testes de resistência para as magias de mago que você conjura e quando você
realiza uma jogada de ataque com uma magia.

\begin{center}
\textbf{CD para suas magias} = 8 + bônus de proficiência + seu modificador de
Inteligência. \nl

\textbf{Modificador de ataque de magia} = seu bônus de proficiência + seu
modificador de Inteligência
\end{center}

\section*{Plano Astral\t\t\t\t\t\t\t\t\t\t}%
\label{sec:plano_astral}

Com muito tempo de estudo e dedicação, os magos conseguem expandir seu plano
astral, tornando-os especialistas em conjuração de magias que outras classes
nunca conseguiram conjurar.

\subsection*{Plano expandido}%
\label{sub:plano_expandido}

Os estudos e prática de um mago fazem com que seu plano se fortaça e expanda
como um músculo. Você possui o dobro de mana armazenado em seu plano astral do
que qualquer outra classe conseguiria.

Além disso, ao invés de começar a ficar exausto com 20\% da mana, um mago
começa a ficar exausto com 15\%. E portanto, o teste é 10 + (15\% - porcentagem
atual de mana restante).


\subsection*{Consciência de seu Plano}%
\label{sub:consciencia_de_seu_plano}

Diferente de outras classes que não utilizam magia de forma tão natural quanto o
mago, você possui treinamento especial para estar sempre consciente de seu
plano astral.
Isso faz com que ele não precisa utilizar componentes materiais (que não possuem
custo atrelado) se ele estiver com seu cajado ou item que usa para conjurar
magias.

\subsection*{Pedras de Poder}%
\label{sub:pedras_de_poder}

A partir do 6º nível, o mago pode usar Pedras de Poder em seus equipamentos para
aumentar a mana total que ele possui e melhorar suas magias. Cada pedra de poder
possui uma característica única.

\section*{Recuperação Arcana\t\t\t\t\t}%
\label{sec:recuperacao_arcana}

Você aprendeu como recuperar um pouco de sua mana em seus estudos como mago. Uma
vez por dia, quando você terminar um descanso curto, você pode recuperar $10\%$
de sua mana total.

\section*{Metamágica\t\t\t\t\t\t\t\t\t\t\t}%
\label{sec:metamagica}

No 2º nível, você adquire a habilidade de distorcer suas magias para se
adequarem às suas necessidades. Você ganha duas das seguintes opções de
Metamágica, à sua escolha. Você adquire outra no 10º e 17º nível.

Você pode usar apenas uma opção de Metamágica em uma magia quando a conjura, a
não ser que esteja descrito o contrário.

\subsubsection*{Magia Acelerada}%
\label{ssub:magia_acelerada}

Quando você conjurar uma magia que tenha um tempo de conjuração de 1 ação, você
gasta $50\%$ a mais da mana descrita na magia para conjurá-la como uma ação
bônus.

\subsubsection*{Magia Aumentada}%
\label{ssub:magia_aumentada}

Quando você conjura uma magia que obriga uma criatura a realizar um teste de
resistência contra o seu efeito, você pode gastar $75\%$ a mais da mana descrita
na magia para dar desvantagem a um alvo da magia no primeiro teste de
resistência feito contra ela.

\subsubsection*{Magia Cuidadosa}%
\label{ssub:magia_cuidadosa}

Quando você conjurar uma magia que obriga outras criaturas a realizarem um teste
de resistência, você pode proteger algumas dessas criaturas da força total da
magia. Para tanto, você gasta $25\%$ a mais da mana descrita na magia e escolhe
um número dessas criaturas até o seu modificador de Carisma (mínimo de uma
criatura). Uma criatura escolhida passa automaticamente no teste de resistência
contra a magia.

\subsubsection*{Magia Distante}%
\label{ssub:magia_distante}

Quando você conjurar uma magia que tenha distância de $1,5$ metro ou maior, você
pode gastar $25\%$ a mais da mana descrita na magia para dobrar o alcance dela.

Quando você conjurar uma magia com alcance de toque, você pode gastar $75\%$ da
mana descrita na magia para mudar o alcance para $4,5$ metros.*

\subsubsection*{Magia Duplicada}%
\label{ssub:magia_duplicada}

Quando você conjurar uma magia que seja incapaz de ter mais de uma criatura como
alvo no nível atual dela e não possua alcance pessoal, você pode gastar uma
quantidade de mana igual a $25\%$ da mana descrita na magia vezes o nível da
magia para ter uma segunda criatura, no alcance da magia, como alvo (se a magia
for um truque, então o gasto é de $25\%$ da mana descrita na magia).

\subsubsection{Magia Estendida}%
\label{ssub:magia_estendida}

Quando você conjurar uma magia que tenha duração de $1$ minuto ou maior, você
pode gastar $50\%$ da mana usada para conjurar a magia para reduzir pela metade o
custo adicional de mana por tempo descrito na magia e dobrar o tempo total de
duração da magia (máximo de $24$ horas).

\subsubsection*{Magia Perseguidora}%
\label{ssub:magia_perseguidora}

Se você realizar uma jogada de ataque para uma magia e errar, você pode gastar
$50\%$ da mana usada para conjurar a magia e jogar o d20 novamente, ficando com
o novo resultado.

Você pode usar a Magia Perseguidora mesmo se já tiver usado outra opção
Metamágica durante a conjuração da magia.

\subsubsection*{Magia Sutil}%
\label{ssub:magia_sutil}

Quando você conjurar uma magia, você pode gastar $25\%$ do custo descrito na
magia para fazê-lo sem qualquer componente somático ou verbal.

\section*{Foco na batalha\t\t\t\t\t\t\t\t}%
\label{sec:foco_na_batalha}

A partir do 3º nível, o mago possui uma experiência em combate grande o
suficiente para conjurar magias sem se distrair com os barulhos e luzes da
batalha ao seu redor.

A partir de agora, sempre que uma magia exigir uma rolagem de dados, você joga o
dobro da quantidade de dados indica e pega a metade de dados maiores. Por
exemplo, se uma magia possui 4d6 de dano, então você deve rolar 8d6 e pegar os
4 maiores valores.

Para fazer isso, entretanto, o mago precisa estar utilizando seu item de
conjuração e ter pelo menos uma mão livre para fazer meio \textit{símbolo de
poder}.

\section*{Incremento em Habilidade\t\t}%
\label{sec:incremento_em_habilidade}

Quando você atinge o 4º nível e novamente no 8º, 12º, 16º e 19º nível, você pode
ganha dois pontos de habilidades para distribuir entre suas habilidades. Por
padrão, você não pode elevar um valor de habilidade acima de 20 com essa
característica.

Opcionalmente, você pode escolher um talento seguindo as regras de talentos.

\section*{Conjuração Extra\t\t\t\t\t\t\t}%
\label{sec:conjuraçao_extra}

A partir do 5º nível, você pode conjurar duas vezes, ao invés de uma, quando
usar a ação Conjuração durante o seu turno.

O número de ataques aumenta para três quando você alcançar o 11º nível de
guerreiro e para 4 quando alcançar o 20º nível de guerreiro.




%%%%%%%%%%%%%%%%%%%%%%%%%%%%%%%%%%%%%%%%%%%%%%%%%%%%%%%%%%%%%%%%%%%%%%%%%%%%%%%%

\chapter{Monge}%
\label{cha:monge}

%%%%%%%%%%%%%%%%%%%%%%%%%%%%%%%%%%%%%%%%%%%%%%%%%%%%%%%%%%%%%%%%%%%%%%%%%%%%%%%%

\chapter{Xamã}%
\label{cha:xama}


\end{document}
