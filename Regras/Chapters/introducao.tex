%!TeX root=../Rules.tex
\chapter{Introdução} % (fold)
\label{cha:Introdução}

\section{Bem-Vindo} % (fold)
\label{sec:Bem-Vindo}

Bem-vindo ao livro de regras de \textit{O Senhor das Sombras}, uma adaptação do
RPG de Mesa D\&D 5e. Esse sistema surgiu de uma vontade minha de usar o D\&D,
mas ao mesmo tempo criar histórias no meu mundo de fantasia, o mundo de
\textit{Jǒna} (pronuncia-se \textit{Jon a}).

Esse livro de regras é uma adaptação do D\&D 5e, e portanto, não é um sistema
completamente novo. Em geral, é desejável que você leia primeiro o Livro do
Jogador (LDJ) da quinta edição de D\&D e tenha alguma familiaridade com esse
sistema. Esse arquivo contém apenas as regras que foram alteradas ou
adicionadas.

\subsection{Quais as principais diferenças?} % (fold)
\label{sub:Quais as principais diferenças?}

A diferença mais importante entre D\&D e SdS é o sistema de magia. Em
\textit{Jǒna}, a magia é comum e acessível. Toda pessoa sabe pelo menos um pouco
de magia, apesar de que aqueles que são realmente bons com ela continuam sendo
uma raridade. Em SdS, a magia é dividida em Escolas, como a Escola Elemental, a
Escola de Ilusão e a Escola Musical. Cada magia pode possuir a uma ou mais
escolas e um personagem também pode pertencer a uma ou mais escolas. Dessa
forma, um personagem sabe todas as magias de sua escola dentro dos níveis que
ele possui. Dessa forma, a flexibilidade da magia dentro do universo de
\textit{Jǒna} é traduzida para o sistema.

Outra diferença importante é que um mago não possui \textit{slots} de
conjuração (por exemplo, 5 slots de nível 1 e 1 slot de nível 2). Em vez disso,
ele possui \textit{mana}, um recurso que existe também dentro do universo do
jogo (algo que não acontecia com os slots). A \textit{mana}, em termos de
sistema, um valor numérico que deve ser gasto para conjurar magias. Cada magia
possui um custo em \textit{mana} e um mago pode conjurar qualquer magia que
conheça, desde que tenha \textit{mana} suficiente para isso. Dessa forma,
portanto, um mago não fica restrito a apenas poucas magias e ganha mais
flexibilidade em quais magias ele vai usar.

Por fim, a última principal diferença é em relação às classes. As classes em SdS
são uma adaptação das existentes em SdS. Algumas são bastante parecidas, como o
\textit{ladino} e o \textit{guerreiro}, enquanto outras foram completamente
mudadas, como o \textit{mago}, e existem ainda algumas que sequer existem em
D\&D, como o \textit{arqueiro} e o \textit{xamã}.

Para mais informações exatas sobre cada uma das mudanças, continue lendo esse
livro de regras.

