%!TeX root=../Rules.tex
\chapter{Lore em Geral}%
\label{cha:lore_em_geral}
\begin{multicols}{2}

\section{Cidades}%

Vamos cobrir bem resumidamente cada cidade. Ao ler cada raça, o jogador pode
entender um pouco mais sobre cada cidade.

\subsection{Península do Goto}%

A península ao norte do continente de Estália é talvez o local mais povoado do
mundo, com principalmente humanos, anões e elfos. Os humanos vivem em cidades
enormes com milhares de estruturas de pedras, geralmente cercadas por muralhas.
Os anões vivem em suas gigantes cordilheiras, escavando a rocha e construindo
salões maciços onde fazem suas festas e reuniões. E os elfos moram nas
florestas, utilizando as próprias árvores como estrutura para suas casas e
vivendo em sintonia com a bela natureza ao seu redor.

A começar pelos humanos, existem três principais cidades. Elas, por sua vez,
possuem diversas vilas e vilarejos subordinados (e por vezes até outras
cidades). São elas Calífora, Nize e Argan.

Calífora é a maior das três, cobrindo a porção sul e central da península, ao
norte das cordilheiras de Malor, que dividem a península do resto do continente,
ao Sul. Ela é também a mais desenvolvida militarmente, com dezenas de guildas
de magos e com uma academia de magia para cada uma das quatro Ordens básicas no
situadas no centro da cidade, ao redor da Grande Arena, que é palco dos maiores
torneios de magia do mundo.

Em Calífora, quem governa é o Kage-Same, que seria Senhor das Sombras em
tradução literal. A família dos Wilton são os únicos conhecidos por manipular as
Sombras e, por isso, governam a cidade desde sua fundação por Aaron Wilton I.

Além da grande metrópole, Calífora possui várias aldeias, vilas e vilarejos sob
seu domínio, fornecendo minérios, alimentos, entre outras mercadorias.

A principal rival de Calífora é Nize, fundada por Ednel Saicro no final da
Guerra dos Dois Irmãos. As guildas de Nize residiam no rio Jote, que hoje é
parte de Calífora, e foram expulsos para o Norte, onde hoje se encontra a cidade
de Nize.

Nize também possui um sistema de guildas de magia, mas diferente de Calífora,
que possui apenas quatro academias de magia para cada uma das ordens básicas,
Nize tem diversas academias espalhadas pela cidade (algumas até fora da
metrópole) e para várias outras ordens, como Veneno, Metal e Ilusionista.

Ela tem um porto que dá para o Norte, de onde recebe mercadorias de Kenia, Argan
e dos povos bárbaros do sul da península do Goto. Além disso, ela também mantém
comércio via terra com Argan e Calífora, mas o principal comércio terrestre é
feito dentro dos limites de Nize com suas cidades e aldeias subordinadas (como
as Minas Galalio, dos anões).

Os governantes de Nize são os Saicro, algumas vezes chamados em Calífora pelo
nome de Grande-Sama ou apenas Sama, muitas vezes pejorativamente. Em Nize, assim
como em Calífora (e também Argan), além das Guildas, que servem como instâncias
militares, existem as corporações de ofício, que são organizações que
representam toda uma categoria de trabalhadores comuns, como ferreiros,
mineiros, agricultores, mercadores, sapateiros e assim por diante. Praticamente
toda categoria de trabalho possui uma corporação de ofício, e isso vale para as
três cidades da península. Entretanto, as corporações não são exatamente iguais.
Em Calífora, existe a corporação de ofício dos fazendeiros, enquanto que em Nize
existem as corporações de ofício dos agricultores e a dos pecuaristas. Tudo
depende de se naquela cidade um ofício é menos ou mais desenvolvido ao ponto de
merecer uma corporação só para si. Essas corporações, por sua vez, respondem a
um órgão maior que, por sua vez, responde ao governante da cidade.

Argan é uma cidade muito parecida com as outras duas mencionadas anteriormente,
mas é a menos militarizada entre elas. Argan não se envolveu na guerra que
acabou gerando Calífora e Nize e, portanto, não tem rivalidade com nenhuma das
duas cidades. Não só isso, mas ela também mantém boas relações com os povos do
sul (geralmente chamados de bárbaros, apesar de alguns arganianos estarem
deixando de utilizar esse nome).

Dentro as três cidades, ela é a que mais possui raças não humanas. Ela fica na
costa oeste da península, ao sul. Próxima dela existe a Floresta de Valoar
(muitas vezes também chamada de Floresta de Argan), onde vivem elfos (chamdos de
Elfos de Valoar). Também existe a Cortilheira Divisora, onde fica atualmente a
cidade anã de Nova Malor. Essas três raças convivem pacificamente entre si e
fazem trocas muito importantes para todos. Além disso, por ficar na costa oeste,
Argan utiliza o Mar Entreterras demasiadamente para fazer trocar mais rápidas
com todas as cidades conhecidas do mundo. Em especial, Argan é a cidade que mais
comercializa com o império dos dragões, Kenia, que fica no continente ao Norte,
Cafrilar. Por conta disso, são encontradas todas essas raças em Argan.

Seus governantes são os Entuart. Há algum tempo houve uma guerra civil em Argan

\section{Linguas novas}%

Além das linguas comumente faladas em D&D, em Jôna exite (pelo menos por
enquanto) uma nova lingua, a Taêno. O nome dessa lingua vem da guilda Taêno, uma
guilda fundada daqueles que se refugiaram da Guerra dos Dois Irmãos, que deu
origem à cidade de Calífora. Entretanto, essa linguagem já existia possivelmente
muito antes dos Taêno e hoje é considerada uma espécie de lingua morta, falada
apenas pelos mais cultos magos e sacerdotes.

Em geral, se utiliza o Taêno para escrita em grimórios e livros religiosos, além
de outros documentos cultos no geral, como livros e pergaminhos que relatam a
história e geografia do mundo conhecido. Essa linguagem é aprendida pelos magos
nos anos mais avançados da academia. Contudo, mesmo assim, o Taêno é uma lingua
que praticamente ninguém sabe realmente falar. O mais comum é ver as raças
misturando-a com suas próprias linguas em um amalgama de palavras bastante
confuso para quem pertence a outra raça.

O fato é que muitas vezes os grimórios e livros de um mago acabam só sendo
inteligíveis para eles mesmos e, quando um outro mago pega um desses manuscritos
para ler, deve levar um bom tempo descifrando o que o autor realmente quis
dizer.

\end{multicols}
