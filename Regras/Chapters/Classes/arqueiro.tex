%!TeX root=../../rules.tex
\chapter{Arqueiro}%
\label{cha:arqueiro}
\begin{multicols}{2}

\section*{Características de Classe}%

Como um arqueiro, você adquire as seguintes características de classe.

\subsubsection{Pontos de Vida}%

\noindent\textbf{Dado de Vida}: 1d6 por nível de arqueiro \nl
\textbf{Pontos de Vida no 1º Nível:} 6 + seu modificador de Constituição. \nl
\textbf{Pontos de Vida nos Níveis Seguintes:} 1d6 (ou 4) + seu modificador de
Constituição por nível de arqueiro após o 1º.

\subsubsection{Proficiências}%

\noindent\textbf{Armaduras:} Armaduras leves \nl
\textbf{Armas:} Armas simples, armas marciais à distância, cimitarra, espada
curta \nl
\textbf{Ferramentas:} Nenhuma \jump
\textbf{Testes de Resistência}: Força, Destreza \nl
\textbf{Pericias:} Escolha três dentre Acrobacia, Adestrar Animais, Atletismo,
Furtividade, Investigação, Natureza, Percepção, Prestidigitação, Sobrevivência

\subsubsection{Equipamento}%

Você começa com o seguinte equipamento, além do equipamento concedido pelo seu
antecedente.
\begin{itemize}
    \item (a) um arco curto e 20 flechas ou (b) uma besta leve e 20 virotes
    \item (a) um arco longo e 20 flechas ou (b) uma besta de mão e 20 virotes
    \item (a) um pacote de aventureiro ou (b) um pacote de explorador
    \item uma armadura de couro e uma adaga
\end{itemize}

\section*{Conjuração}%

A tabela a seguir mostra em quais níveis o arqueiro tem acesso às magias de cada
ciclo:

\begin{center}
\begin{tabular}{|||c||c|||}
    \hline
    \textbf{Nível} & \textbf{Ciclo} \\
    \hline
    2 & 1º \\
    \hline
    5 & 2º \\
    \hline
    8 & 3º \\
    \hline
    11 & 4º \\
    \hline
    14 & 5º \\
    \hline
    17 & 6º \\
    \hline
    20 & 7º \\
    \hline
\end{tabular}
\end{center}

\subsubsection*{Habilidade de Conjuração}%

Destreza é a sua habilidade para você conjurar suas magias de arqueiro, pois os
arqueiros usam sua agilidade e precisão no meio do combate.  Você usa seu
Destreza sempre que alguma magia se referir a sua habilidade de conjurar magias.
Além disso, você usa o seu modificador de Destreza para definir a CD dos testes
de resistência para as magias de arqueiro que você conjura e quando você realiza
uma jogada de ataque com uma magia.

\begin{center}
\textbf{CD para suas magias} = 8 + bônus de proficiência + seu modificador de
Destreza. \nl

\textbf{Modificador de ataque de magia} = seu bônus de proficiência + seu
modificador de Destreza
\end{center}

\end{multicols}
\begin{center}
\begin{tabular}{
        | b{8mm}<{\centering}
        b{23mm}<{\centering}
        b{18mm}<{\centering}
        p{70mm}<{\raggedright\arraybackslash} |
}
    \hline

    \multicolumn{4}{|l|}{\textbf{\Large O Arqueiro}} \\

    % Header
    \textbf{Nível} & \textbf{Bônus de Proficiência} & \textbf{Inimigo Favorito}
    & \textbf{Características} \\
    \hline \hline

    \textbf{1º} & +2 & 1d6 & Arquearia, Inimigo Favorito \\
    \hline
    \textbf{2º} & +2 & 1d6 & Pontaria \\
    \hline
    \textbf{3º} & +2 & 1d6 & Varar \\
    \hline
    \textbf{4º} & +2 & 1d6 & Incremento em Habilidade \\
    \hline
    \textbf{5º} & +3 & 1d8 & Ataque Extra \\
    \hline
    \textbf{6º} & +3 & 1d8 & \\
    \hline
    \textbf{7º} & +3 & 1d8 & Precisão, Pontaria (Manobra, Dado) \\
    \hline
    \textbf{8º} & +3 & 1d8 & Incremento em Habilidade \\
    \hline
    \textbf{9º} & +4 & 1d8 & \\
    \hline
    \textbf{10º} & +4 & 1d10 & Pontaria (Manobra) \\
    \hline
    \textbf{11º} & +4 & 1d10 & Ataque Extra \\
    \hline
    \textbf{12º} & +4 & 1d10 & Incremento em Habilidade \\
    \hline
    \textbf{13º} & +5 & 1d10 & Saraivada \\
    \hline
    \textbf{14º} & +5 & 1d10 & Pontaria (Manobra, Dado) \\
    \hline
    \textbf{15º} & +5 & 1d10 & Precisão \\
    \hline
    \textbf{16º} & +5 & 2d6 & Incremento em Habilidade \\
    \hline
    \textbf{17º} & +6 & 2d6 & \\
    \hline
    \textbf{18º} & +6 & 2d6 & \\
    \hline
    \textbf{19º} & +6 & 2d8 & Incremento em Habilidade \\
    \hline
    \textbf{20º} & +6 & 2d8 & Ataque Extra, Precisão \\
    \hline
\end{tabular}
\end{center}
\begin{multicols}{2}

\section*{Arquearia}%

A partir do 1º nível, você ganha alguns benefícios:
\begin{itemize}
    \item Você ganha +2 de bônus nas jogadas de ataque realizadas com uma arma
        de ataque à distância;
    \item Qualquer arma de ataque à distância não arremessável pode ser um foco
        de conjuração para você, como arcos e bestas;
    \item Sempre que o arqueiro é listado em uma magia que se refere a ataques
        corpo-a-corpo, ele pode fazer o ataque à distância usando seu foco de
        conjuração.
\end{itemize}

\section*{Inimigo Favorito}%

A partir do 1º nível, você pode escolher um inimigo favorito. Esse inimigo deve
ser uma criatura que você consiga ver a até 18 metros de você. Enquanto você
conseguir ver esse inimigo e enquanto ele estiver dentro do alcance, sempre que
você o atingir com uma arma de ataque à distância, você causa 1d6 a mais de
dano perfurante.

Esse dano aumenta conforme você ganha níveis de arqueiro. Ele se torna 1d8 no
5º nível, 1d10 no 10º nível, 2d6 no 16º nível e 2d8 no 19º nível.

Quando um inimigo favorito for reduzido para 0 pontos de vida, você pode usar
sua ação bônus para escolher outro inimigo favorito.

Possuir um inimigo favorito utiliza a sua concentração até o 5º nível de
arqueiro. Do 6º em diante, você pode ter um inimigo favorito e utilizar sua
concentração para outras coisas.

\section*{Pontaria}%

A partir do 2º nível, você aprende manobras que são abastecidas com dados
especiais chamados dados de pontaria.

\textbf{Manobras.} Você aprende três manobras, à sua escolha, que são detalhadas
em ``Manobras'' no final do capítulo.

Muitas manobras aprimoram um ataque de várias formas. Você só pode usar uma
manobra por ataque.

Você aprende uma manobra adicional, à sua escolha, no 7°, 10° e 14° nível. A
cada vez que você aprende uma nova manobra, você pode substituir uma manobra
conhecida por uma diferente.

\textbf{Dados de Pontaria.} Você tem quatro dados de pontaria, que são d8s. Um
dado de pontaria é gasto quando você usa-o. Você recupera todos os dados de
pontaria gastos quando terminar um descanso curto ou longo.

Você adquire outro dado de pontaria no 7° nível e mais um no 14° nível.

\textbf{Teste de Resistência.} Algumas das suas manobras exigem que o alvo
realize um teste de resistência contra o efeito da manobra. A CD do teste de
resistência é calculada a seguir:
\begin{center}
\textbf{CD para suas manobras} = 8 + bônus de proficiência + seu modificador de
Destreza.
\end{center}

\section*{Varar}%

A partir do 3º nível, sempre que você acertar um crítico ou matar um alvo, seu
tiro continua em linha reta até atingir um segundo alvo, bater em uma parede ou
acabar o alcance de sua arma (para armas com duas distâncias, considere alcance
como a distância normal daquela arma).

No caso de um crítico, o segundo alvo toma metade do dano que o primeiro levou.
Em caso de você matar o primeiro alvo, então o dano que o segundo recebe é o
dano total menos a vida que o primeiro alvo tinha antes de morrer. Por exemplo,
se você deu 20 de dano e o primeiro alvo tinha 8 de vida, você o mata e o alvo
de trás recebe 12 pontos de dano.

O segundo alvo toma o dano automaticamente, independentemente de sua CA.

\section*{Incremento em Habilidade}%

Quando você atinge o 4° nível e novamente no 8°, 12°, 16° e 19° nível, você pode
aumentar um valor de habilidade, à sua escolha, em 2 ou você pode aumentar dois
valores de habilidade, à sua escolha, em 1. Como padrão, você não pode elevar um
valor de habilidade acima de 20 com essa característica.

\section*{Ataque Extra}%

A partir do 5º nível, você pode atacar duas vezes, ao invés de uma, quando usar
a ação Atacar durante o seu turno.

O número de ataques aumentar para três quando você alcançar o 11º nível de
arqueiro e 4 quando você alcançar o 20º nível de arqueiro.


\section*{Precisão}%

A partir do 7º nível, você reduz em 1 a quantidade que você precisa tirar para
acertar um crítico. Em particular, se você utilizar uma arma que diz que o
crítico é em 19 ou 20, então agora um 18 no dado também se torna um crítico.

No 15º nível, você reduz essa quantidade em 2 e no 20º nível, em 3.


\section*{Saraivada}%

A partir do 13º nível, você pode usar sua ação para realizar um ataque à
distância contra qualquer número de criaturas a até 3 metros de um ponto que
você possa ver, no alcance de sua arma. Você deve ter munição para cada alvo,
como normalmente, e você realiza uma jogada de ataque separada para cada alvo.

Após isso, você não pode usar qualquer tipo de ações extras para fazer jogadas
de ataque em seu turno.

\section*{Manobras}%

As manobras são apresentadas em ordem alfabética.

\paragraph{Contra-Atacar.}%

Quando uma criatura atacar você com um ataque à distância e errar, você pode
usar sua reação e gastar um dado de pontaria para realizar um ataque à distância
com arma contra essa criatura. Se você atingir, você adiciona seu dado de
pontaria a jogada de dano do ataque.

\paragraph{Derrubar.}%

Quando você atingir uma criatura com um ataque à distância, você pode gastar um
dado de pontaria para tentar derrubar o alvo. Você adiciona seu dado de pontaria
à jogada de dano do ataque e, se o alvo for Grande ou menor, ele deve realizar
um teste de resistência de Força. Se falhar, o alvo ficará caído no chão.

\paragraph{Emboscada.}%

Quando você realizar um teste de Destreza (Furtividade) ou uma jogada de
iniciativa, você pode gastar um dado de pontaria e adicionar o dado na rolagem,
desde que você não esteja incapacitado.

\paragraph{Evasão.}%

Quando você for alvo de um efeito que exige um teste de resistência de Destreza,
você pode gastar um dado de pontaria e adicionar o dado na rolagem.

\paragraph{Golpe Distrativo.}%

Quando você atingir uma criatura com um ataque à distância com arma, você pode
gastar um dado de pontaria para tentar distrair a criatura, abrindo uma brecha
para um de seus aliados. Você adiciona seu dado de pontaria à jogada de dano do
ataque. A próxima jogada de ataque realizada contra o alvo por uma criatura
diferente de você tem vantagem se o ataque for realizado antes do começo do seu
próximo turno.

\paragraph{Golpe Preciso.}%

Quando você realizar uma jogada de ataque à distância com arma contra uma
criatura com cobertura, gaste um dado de pontaria para tentar acertá-la. Você
ignora o fato dela estar com cobertura e adiciona o valor rolado à jogada de
dano em caso de acerto.

\paragraph{Lançamento Rápido.}%

Como uma ação bônus, você pode gastar um dado de pontaria e realizar um ataque
com uma arma que tenha a propriedade de arremesso. Você pode sacar a arma como
parte dessa ação de ataque. Se você acertar, adicione o dado de pontaria à
jogada de dano da arma.

\paragraph{Passo Evasivo.}%

Quando você se mover, você pode gastar um dado de pontaria, role o dado e
adicione o número rolado à sua CA até você terminar seu deslocamento.

\paragraph{Sentido Apurado}%

Quando você precisar fazer um teste de Percepção ou Investigação, você pode
gastar um dado de pontaria e adicionar o dado na rolagem.

\end{multicols}

%%%%%%%%%%%%%%%%%%%%%%%%%%%%%%%%%%%%%%%%%%%%%%%%%%%%%%%%%%%%%%%%%%%%%%%%%%%%%%%%


