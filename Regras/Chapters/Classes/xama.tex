%!TeX root=../../rules.tex
\chapter{Xamã}%
\label{cha:xama}
\begin{multicols}{2}

\section*{Características de Classe}%

Como um xamã, você adquire as seguintes características de classe.

\subsubsection{Pontos de Vida}%

\noindent\textbf{Dado de Vida}: 1d6 por nível de xamã \nl
\textbf{Pontos de Vida no 1º Nível:} 6 + seu modificador de Constituição. \nl
\textbf{Pontos de Vida nos Níveis Seguintes:} 1d6 (ou 4) + seu modificador de
Constituição por nível de xamã após o 1º.

\subsubsection{Proficiências}%

\noindent\textbf{Armaduras:} Nenhuma \nl
\textbf{Armas:} Armas simples \nl
\textbf{Ferramentas:} Nenhuma \jump
\textbf{Testes de Resistência}: Inteligência, Sabedoria \nl
\textbf{Pericias:} Escolha duas entre Arcanismo, Atuação, História, Intuição,
Investigação, Lidar com Animais, Medicina, Persuasão e Religião.

\subsubsection{Equipamento}%

Você começa com o seguinte equipamento, além do equipamento concedido pelo seu
antecedente.
\begin{itemize}
    \item (a) um bordão ou (b) uma adaga ou (c) um arco curto;
    \item (a) uma bolsa de componentes ou (b) um foco arcano;
    \item (a) um pacote de estudioso ou (b) um pacote de explorador;
\end{itemize}

\section*{Conjuração}%

A tabela a seguir mostra em quais níveis o xamã tem acesso às magias de cada
ciclo:

\begin{center}
\begin{tabular}{|||c||c|||}
    \hline
    \textbf{Nível} & \textbf{Ciclo} \\
    \hline
    1 & 1º \\
    \hline
    3 & 2º \\
    \hline
    5 & 3º \\
    \hline
    7 & 4º \\
    \hline
    9 & 5º \\
    \hline
    11 & 6º \\
    \hline
    13 & 7º \\
    \hline
    15 & 8º \\
    \hline
    17 & 9º \\
    \hline
\end{tabular}
\end{center}

\subsubsection*{Habilidade de Conjuração}%

Sabedoria é a sua habilidade para você conjurar suas magias de xamã, pois os
xamã aprendem novas magias através da experiência e intuição. Você usa sua
Sabedoria sempre que alguma magia se referir a sua habilidade de conjurar
magias. Além disso, você usa o seu modificador de Sabedoria para definir a CD
dos testes de resistência para as magias de xamã que você conjura e quando você
realiza uma jogada de ataque com uma magia.

\begin{center}
\textbf{CD para suas magias} = 8 + bônus de proficiência + seu modificador de
Sabedoria. \nl

\textbf{Modificador de ataque de magia} = seu bônus de proficiência + seu
modificador de Sabedoria
\end{center}

\end{multicols}
\begin{center}
\begin{tabular}{
        | b{8mm}<{\centering}
        b{23mm}<{\centering}
        b{18mm}<{\centering}
        p{70mm}<{\raggedright\arraybackslash} |
}
    \hline

    \multicolumn{4}{|l|}{\textbf{\Large O Xamã}} \\

    % Header
    \textbf{Nível} & \textbf{Bônus de Proficiência} & \textbf{Receitas} &
    \textbf{Características} \\
    \hline \hline

    \textbf{1º} & +2 & - & Plano Astral \\
    \hline
    \textbf{2º} & +2 & 2 & Receitas \\
    \hline
    \textbf{3º} & +2 & 3 & Trilha Xamã \\
    \hline
    \textbf{4º} & +2 & 3 & Incremento em Habilidade \\
    \hline
    \textbf{5º} & +3 & 3 & Conjuração Extra \\
    \hline
    \textbf{6º} & +3 & 4 & Vidente, Trilha Xamã \\
    \hline
    \textbf{7º} & +3 & 4 & Conjurador Nato \\
    \hline
    \textbf{8º} & +3 & 4 & Incremento em Habilidade \\
    \hline
    \textbf{9º} & +4 & 5 & \\
    \hline
    \textbf{10º} & +4 & 5 & \\
    \hline
    \textbf{11º} & +4 & 5 & Trilha Xamã \\
    \hline
    \textbf{12º} & +4 & 6 & Incremento em Habilidade \\
    \hline
    \textbf{13º} & +5 & 6 & Conjuração Extra \\
    \hline
    \textbf{14º} & +5 & 6 & Vidente, Invocador Nato \\
    \hline
    \textbf{15º} & +5 & 7 & \\
    \hline
    \textbf{16º} & +5 & 7 & Incremento em Habilidade, Despertar \\
    \hline
    \textbf{17º} & +6 & 8 & Trilha Xamã \\
    \hline
    \textbf{18º} & +6 & 8 & \\
    \hline
    \textbf{19º} & +6 & 9 & Incremento em Habilidade \\
    \hline
    \textbf{20º} & +6 & 10 & Conjuração Extra, Senhor do Destino \\
    \hline
\end{tabular}
\end{center}
\begin{multicols}{2}

\section*{Plano Astral}%

Todos os conjuradores possuem um plano astral. Nele, o conjurador guarda sua
mana e pode canalizá-la para conjurar maravilhas no Mundo Natural.

\subsection*{Plano expandido}%

Com muito tempo de estudo e dedicação, os magos conseguem expandir seu plano
astral, tornando-os especialistas em conjuração de magias que outras classes
nunca conseguiram conjurar.

Os estudos e prática de um mago fazem com que seu plano se fortaleça e expanda
como um músculo. Você possui o dobro de mana armazenado em seu plano astral do
que qualquer outra classe conseguiria.

Além disso, ao invés de começar a ficar exausto com 20\% da mana, um mago
começa a ficar exausto com 15\%. E portanto, o teste é 10 + (15\% - porcentagem
atual de mana restante).

\subsection*{Ajuda Anciã}%

O conhecimento e intuição de deuses e espíritos ancestrais fluem pelo seu sangue
e mana. Todas magias marcadas como \textit{buff} ou \textit{cura} são conjuradas
por você utilizando $20\%$ a menos de mana.

\subsection*{Espíritos Antigos}%

Você ouve sussurros e pensamentos difusos de criaturas que não consegue
compreender. Entretanto, você sabe como convocar temporariamente criaturas de
reinos e planos distantes.

Além de uma habitual ordem elemental, o xamã sabe conjurar magias de invocação.

\subsection*{Proficiências Bônus}%

Seu plano astral evoluído permite que você misteriosamente saiba falar, ler e
escrever em Taêno. Além disso, você ganha proficiência com ferramentas de
artesão.

\section*{Receitas}%

A partir do segundo nível, um xamã aprende a criar receitas, que são um tipo
especial de magia salva em um pergaminho de papel. Xamãs ao redor do mundo
armazenam suas magias nesses papeis e vendem a aventureiros, guardas ou até
mesmo pedreiros e arquitetos por quantidades grandes de dinheiro.

Para criar uma receita, você precisa:
\begin{itemize}
    \item Saber conjurar a magia normalmente;
    \item Não pode ser uma magia de concentração e nem de reação;
    \item Possuir um pergaminho vazio;
    \item Possuir uma tinta especial para a ordem/escola de magia que você
            quer registrar. Para as ordens elementais comuns, se encontra essa tinta
        facilmente em lojas de outros xamãs;
    \item Não ter dez ou mais receitas em seu inventário;
    \item Gastar o dobro da mana que a magia necessita;
\end{itemize}

Cumprindo os requisitos, você cria um item que vai imediatamente para seu
inventário chamado \textit{Receita de X} (onde X é o nome da magia). Esse item
pode ser utilizado com uma ação de conjuração sem qualquer gasto de mana e o
efeito é a conjuração da magia como se ela fosse sua (independente de você ter
ou não aquela magia e/ou aquela ordem/escola).

O limite de receitas que um xamã pode carregar é determinado pelo seu nível de
acordo com a tabela O Xamã. E o limite de outras classes é fixo em 3. Receitas
são instáveis e não se deve ter muitas juntas, mesmo em lugares seguros como
casas e armazéns. Mesmo xamãs que possuem lojas não armazenam muitas receitas,
possuindo apenas receitas raras que eles mesmos não conseguem criar.

\section*{Trilha Xamã}

Quando você alcança o 3º nível, você começa a trilhar um caminho próprio
diferente da maioria dos xamãs. Você escolhe entre: Trilha do Rúnico, Trilha do
Curandeiro. Sua trilha concede a você características no 3º nível e novamente no
6º, 11º e 17º nível. As trilhas estão detalhadas no final do capítulo.

\section*{Incremento em Habilidade}

Quando você atinge o 4° nível e novamente no 8°, 12°, 16° e 19° nível, você pode
aumentar um valor de habilidade, à sua escolha, em 2 ou você pode aumentar dois
valores de habilidade, à sua escolha, em 1. Como padrão, você não pode elevar um
valor de habilidade acima de 20 com essa característica.

\section*{Conjuração Extra}%

A partir do 5º nível, você pode conjurar duas vezes, ao invés de uma, quando
usar a ação Conjuração durante o seu turno.

Esse número aumenta para 3 quando você alcançar o 13º nível de mago e para 4
quando alcançar o 20º nível de mago.

\section*{Vidente}%

A partir do 6º nível, você começa os espíritos anciões começam a revelar
visões pequenas para você. Quando você termina um descanso longo, role dois d20s
e anote os números rolados. Você pode substituir qualquer jogada de ataque,
teste de resistência ou teste de habilidade feito por você ou por outra criatura
que você possa ver por uma das rolagens que você anotou.

Você deve escolher fazer isso antes da rolagem e você pode substituir uma
rolagem dessa forma apenas uma vez por rodada.

A partir do 14º nível, as visões em seus sonhos se intensificam e você rola três
d20s ao invés de dois.

\section*{Conjurador Nato}%

A partir do 11º nível, o xamã mostra ser um conjurador mais eficiente que
qualquer outro tipo, executando as mesmas magias que outros conjuradores, mas de
forma muito mais eficiente.

A partir de agora, sempre que uma magia exigir uma rolagem de dados, você joga o
dobro da quantidade de dados indicada e pega a metade de dados maiores. Por
exemplo, se uma magia possui 4d6 de dano, então você deve rolar 8d6 e pegar os
4 maiores valores.

Para fazer isso, entretanto, o xamã precisa estar utilizando seu item de
conjuração e ter pelo menos uma mão livre para fazer meio \textit{símbolo de
poder}.

\section*{Invocador Nato}%

A partir do 14º nível, suas invocações ficam mais poderosas. Qualquer criatura
que você invoca terá 30 pontos de vida temporários.

\section*{Despertar}%

A partir do 16º nível, você as vozes em sua cabeça despertam sobre você, agindo
como se fossem uma outra consciência. A partir de agora, você pode se concentrar
em duas magias ao mesmo tempo.

\section*{Senhor do Destino}%

A partir do 20º nível, você suas premonições do futuro se intensificam de tal
forma que você consegue escolher ver o futuro momentaneamente. Você agora possui
as seguintes características:
\begin{enumerate}
    \item \textbf{Vidente.} Você consegue saber o resultado das próximas 5
        rolagens de qualquer criaturas controladas pelo destino que você
        encontrar. Isso significa que o mestre deve rolar cinco d20s e
        anotá-los, revelando-os a você. Das próximas cinco vezes que um
        Personagem do Mestre precisar realizar uma rolagem de um d20, a rolagem
        será aquela anotada pelo mestre;
    \item \textbf{Costureiro da Malha do Tempo.} Você pode forçar qualquer
        criatura que você possa ver (incluindo você) a falhar ou passar em um
        teste de resistência ou teste de habilidade. Se você escolhe que ela
        falhará, então é como se ela tivesse rolado 1 no d20. Se você escolhe
        que ela passará, então é como se ela tivesse rolado 20 no d20;
    \item \textbf{Voltar no Tempo.} Quando uma criatura atinge você e realiza
        uma rolagem de dano, você pode forçar a criatura a rolar novamente o
        dano, mesmo depois de o mestre já ter rolado os dados. O novo dano deve
        ser utilizado.
\end{enumerate}

Cada uma dessas características só pode ser utilizada uma vez por descanso
longo.

\section*{Trilhas do Xamã}%

Cada xamã deve trilhar um caminho próprio. Essas trilhas geralmente remetem a
outros xamãs que já cursaram aquele caminho e servem de inspiração para novos
xamãs.

\subsection*{Trilha do Rúnico}%

Os xamãs rúnicos aprimoram sua proeza mágica utilizando o poder sobrenatural de
runas ensinadas pelos ancestrais e perdidas durante a Era Antiga. Hoje poucos
são aqueles que conseguem desbloquear seu verdadeiro potencial.

\subsubsection*{Entalhador de Runas}%

Você pode usar runas mágicas para aprimorar seu equipamento ou suas receitas. Ao
ganhar essa habilidade você aprende a inscrever duas runas à sua escolha dentre
as runas descritas abaixo.

Quando você alcançar certos níveis nessa classe, você aprende runas adicionais,
conforme mostrado na tabela de Runas Conhecidas. Além disso, toda vez que você
aprender uma runa adicional, você pode trocar uma das runas que você conhecia
por outra.

\begin{tabular}{c c}
\multicolumn{2}{l}{\textbf{\Large Runas Conhecidas}} \\
\textbf{Nível de Xamã} & \textbf{Número de Runas} \\
3º & 2 \\
7º & 3 \\
10º & 4 \\
15º & 5 \\
\end{tabular}

Sempre que terminar um descanso longo você pode tocar uma quantidade de objetos
igual ao número de runas que você conhece e gravar uma runa diferente em cada um
deles. Para que isso seja possível, este objeto deve ser uma arma, armadura,
escudo, uma joia, alguma outra coisa que você possa vestir ou carregar em uma
mão. Sua runa permanece em um objeto até você terminar outro descanso longo, e
um objeto pode conter apenas uma runa por vez.

As seguintes runas estão disponíveis quando você aprender uma runa. Se uma runa
possuir algum requisito de nível, você deve estar pelo menos no nível mínimo
exigido para aprendê-la. Se uma runa pedir por uma salvaguarda, a CD de magia da
runa será 8 + bônus de proficiência + modificador de Sabedoria.

\paragraph{Runa da Nuvem.}%

Enquanto estiver usando ou carregando um item com essa runa gravada, você tem
vantagem nos testes de Destreza (Prestidigitação) e Carisma (Enganação).

Além disso, quando você ou uma criatura a até 9 metros for acertada por um
ataque, você pode usar sua reação para invocar a runa e escolher outra criatura
a até 9m de você, que não seja o atacante. A criatura escolhida se torna o alvo
do ataque, utilizando a mesma jogada. Ao invocar esta runa você não poderá fazer
isso novamente até terminar um descanso curto ou longo.

\paragraph{Runa do Fogo.}%

Enquanto estiver usando ou carregando um item com uma runa dessas gravada seu
bônus de proficiência é dobrado para qualquer teste que utilizar ferramentas.

Além disso, ao acertar um ataque com arma em uma criatura, você pode invocar a
runa para conjurar grilhões incandescentes: o alvo recebe 2d6 de dano de fogo e
deve ser bem-sucedido em uma salvaguarda de Força ou ficará impedido por 1
minuto. Enquanto estiver impedido pelas correntes, o alvo sofre 2d6 de dano de
fogo no começo de cada um dos turnos dele. O alvo pode repetir a salvaguarda no
final de cada um dos turnos dele, e os grilhões desaparecem caso ele seja
bem-sucedido. Ao invocar esta runa você não poderá fazer isso novamente até
terminar um descanso curto ou longo.

\paragraph{Runa do Gelo.}%

Enquanto estiver usando ou carregando um item com uma runa dessas gravada você
tem vantagem nos testes de Sabedoria (Lidar com Animais) e Carisma
(Intimidação).

Além disso, você pode invocar a runa como uma ação bônus para aumentar sua
resistência. Por 10 minutos, você ganha um bônus de +2 para todos os testes de
habilidade e salvaguarda que utilizem Força ou Constituição. Ao invocar esta
runa você não poderá fazer isso novamente até terminar um descanso curto ou
longo.

\paragraph{Runa da Pedra.}%

Enquanto estiver usando ou carregando um item com uma dessas runas gravada você
tem vantagem nos testes de Sabedoria (Intuição) e adquire visão no escuro com
alcance de 36m.

Além disso, quando uma criatura que você possa ver termina o turno dela a até 9
metros de você, você pode usar sua reação para invocar a runa e forçar a
criatura a fazer uma salvaguarda de Sabedoria. A menos que seja bem sucedida, a
criatura fica enfeitiçada por você por 1 minuto. Enquanto estiver enfeitiça
desta forma, a criatura tem deslocamento 0 e fica incapacitada, caindo em um
coma estupor cheio de sonhos. A criatura repete a salvaguarda ao final de cada
um dos turnos dela, encerrando o efeito em caso de sucesso. Ao invocar esta runa
você não poderá fazer isso novamente até terminar um descanso curto ou longo.

\paragraph{Runa da Colina. (7º nível ou superior)}%

Enquanto estiver usando ou carregando um item com uma dessas runas gravada você
tem vantagem em salvaguardas contra ser envenenado e tem resistência contra dano
venenoso.

Além disso, você pode invocar a runa com uma ação bônus, ganhando resistência a
dano cortante, perfurante e contundente por 1 minuto. Ao invocar esta runa você
não poderá fazer isso novamente até terminar um descanso curto ou longo.

\paragraph{Runa da Tempestade. (7º nível ou superior)}%

Enquanto estiver usando ou carregando um item com uma dessas runas gravada você
tem vantagem nos testes de Inteligência (Arcanismo) e não pode ser surpreendido,
desde que não esteja incapacitado.

Além disso, você pode invocar a runa com uma ação bônus para entrar em um estado
profético por 1 minuto ou até ser incapacitado. Enquanto estiver neste estado,
quando você ou outra criatura que você possa ver a até 18m da sua posição fizer
uma jogada de ataque, salvaguarda ou teste de habilidade, você pode usar sua
reação para dar vantagem ou desvantagem nesta rolagem. Ao invocar esta runa você
não poderá fazer isso novamente até terminar um descanso curto ou longo.

\subsubsection*{Runa Gnômica}%

A partir do 6º nível, você é capaz de conjurar runas gnômicas em suas receitas,
tornando-as automatizáveis. Você pode escrever uma runa gnômica em uma de suas
receitas e descrever um gatilho especial que vai conjurar a magia
automaticamente quando acontecer esse gatilho age como se fosse uma ação de
Preparar.

Você pode colar o pergaminho da receita em uma parede ou deixá-lo no chão e
ainda assim, se o gatilho acontecer perto da receita, ela se ativará e você
ficará sabendo devido sua ligação com a runa. Ao invocar a runa, o pergaminho se
queima, deixando apenas um leve desenho da runa sobre a superfície, você pode
escolher entre deixar essa runa completamente visível ou camuflada. Uma runa
camuflada requer um teste de Inteligência (Investigação) para ser vista contra o
CD de seu modificador de magias.

Você só pode ter uma receita por vez com uma runa gnômica e você só pode ter uma
runa gnômica ativa por vez.

\subsection*{Runa Gnômica Aprimorada}%

A partir do 11º nível, você aprimora suas runas gnômicas, adquirindo mais
conhecimento sobre elas. A partir de agora, você é capaz de manter duas runas
gnômicas inscritas em receitas e você não precisa mais de um gatilho especial
para conjurar a magia. Utilizando uma ação, você pode ativa uma das runas que
você escolher desde que a receita e você estejam no mesmo plano.

\subsubsection*{Mestre das Runas}%

A partir do 17º nível, você pode invocar cada runa que conhece duas vezes ao
invés de uma. Além disso, você possui agora três runas gnômicas ao invés de
duas.

\subsection*{Trilha do Curandeiro}%

Curandeiros são uma fonte de benevolência dentro de seus grupos. Eles se tornam
grandes referências para cidadãos, soldados e até líderes locais, que buscam
esses xamãs cheios de conhecimento sobre o mundo natural.

\subsubsection*{Alquimista}%

A partir do 3º nível, você adquire a habilidade de criar poções de vários
tipos. Você sabe criar três poções e o número aumenta conforme você ganha níveis
de xamã conforme a tabela Alquimista abaixo.

tabela

Sempre que você realizar um descanso curto ou longo, se você tiver os
ingredientes necessários para construção de uma poção que você saiba como fazer,
você pode gastar uma hora fazendo a poção. Sempre que alguém tomar uma poção, o
você pode pegar o frasco e guardá-lo.

As seguintes poções estão disponíveis quando você aprender a fazer uma poção. Se
uma poção possuir algum requisito de nível, você deve estar pelo menos no nível
mínimo exigido para aprendê-la.

\paragraph{Poção de Cura.} %

Requer \textit{um frasco de poção normal}, \textit{água destilada},
\textit{ervas medicinais}, 2 pp.

É também possível criar poções de cura em níveis maiores: \nl
Maior no 6º nível. (por 10 pp) \nl
Superior no 11º nível. (por 100 pp) \nl
Suprema no 15º nível. (por 1000 pp) \nl

\paragraph{Poção de Mana.}%

Requer \textit{um frasco de poção normal}, \textit{água destilada},
\textit{ervas medicinais}, 2 pp.

É também possível criar poções de cura em níveis maiores: \nl
Maior no 6º nível. (por 10 pp) \nl
Superior no 11º nível. (por 100 pp) \nl
Suprema no 15º nível. (por 1000 pp) \nl

\paragraph{Poção de Aprimoramento.}

Requer \textit{um frasco de poção normal}, \textit{água destilada}, 500 de mana.

Poções de aprimoramento aumentam em +1 o status de quem tomar por um dia. Elas
podem aprimorar qualquer status, sendo este definido na hora que a poção é
criada.

É também possível criar poção de aprimoramento em níveis maiores: \nl
Maior (+2) no 6º nível. (por 2000 de mana) \nl
Superior (+3) no 11º nível. (por 10000 de mana) \nl

\paragraph{Poção de Invisibilidade.}

Requer \textit{um frasco de poção normal}, \textit{água destilada}, \textit{uma
gota de sangue humanoide}, \textit{um olho de algum animal morto}, 500 de mana.

O recipiente desta poção aparenta estar vazio, mas parece conter um líquida.
Quando o bebe, você fica invisível por 1 hora. Tudo que você estiver vestindo ou
carregando fica invisível com você. O efeito termina prematuramente se você
atacar ou conjurar uma magia.

\paragraph{Poção de Resistência.}

Requer \textit{um frasco de poção normal}, \textit{água destilada}, \textit{uma
gota de sangue humanoide}, \textit{um pedaço de pele de algum animal}, 500 de
mana.

Quando bebe esta poção, você ganha resistência a um tipo de dano por 1 hora. O
dano é determinado no momento de construção da poção.

\paragraph{Poção de Respirar na Água.}

Requer \textit{um frasco de poção normal}, \textit{água salgada}, \textit{ser
mago da Água}, 500 de mana.

Você pode respirar embaixo d'água por 1 hora após beber esta poção. Seu fluido
verde nebuloso cheira a maresia e possui uma bolha similar a uma água-viva
flutuando nele.

\paragraph{Poção de Sopro de Fogo.}%

Requer \textit{um frasco de poção normal}, \textit{cinzas}, \textit{ser mago do
Fogo}, 500 de mana.

Quando bebe esta poção, você pode usar uma ação bônus para exalar fogo em um
alvo a até 9 metros de você. O alvo deve realizar um teste de resistência de
Destreza CD 13, sofrendo 4d6 de dano de fogo se falhar na resistência, ou metade
desse dano caso obtenha sucesso. O efeito termina após você expelir fogo três
vezes, ou após 1 hora se passar. O líquido laranja desta poção lampeja e fumaça
preenche o topo do recipiente e vaza toda vez que ele é aberto.

\paragraph{Poção de Velocidade.}%

Requer \textit{um frasco de poção normal}, \textit{vento}, \textit{ser mago do
Ar}, 500 de mana.

Quando bebe esta poção, você ganha +3m de deslocamento por 1 minuto (não requer
concentração). O fluido amarelo da poção é matizado de preto e gira em torno de
si.

\paragraph{Poção de Vitalidade.}%

Requer \textit{um frasco de poção normal}, \textit{raízes}, \textit{ser mago do
Terra}, 500 de mana.

Beba quando for utilizar Dados de Vida. Você recupera o dobro do valor rolado.
O líquido carmesim da poção pulsa regularmente com luz pálida, lembrando as
batidas de um coração.

\paragraph{Poção de Ler Mentes. (13º nível ou superior)}

Requer \textit{um frasco de poção ornamentado}, \textit{o cérebro e sangue de um
psiônico}, 5000 de mana.

Quando bebe esta poção, você ganha os efeitos da magia detectar pensamentos (CD
de resistência 13). O líquido roxo denso dessa poção tem uma nuvem ovoide rosa
flutuando nele.

\paragraph{Poção de Invulnerabilidade. (6º nível ou superior)}

Requer \textit{um frasco de poção grande}, \textit{300ml de sangue humanoide},
\textit{pó de ferro}, 1000 de mana.

Por 1 minuto após beber esta poção, você adquire resistência a todos os danos. O
líquido xaroposo dessa poção parece ferro derretido.

\paragraph{Poção de Heroísmo. (6º nível ou superior)}

Requer \textit{um frasco de poção grande}, \textit{300ml de sangue humanoide},
\textit{pó de ouro equivalente a 5po}, 1000 de mana.

Por 1 hora, após bebe-la, você ganha 10 pontos de vida temporários. Pela mesma
duração, você está sob efeito da magia benção (não requer concentração). Esta
poção azul borbulha e fumega como se estivesse fervendo.

\paragraph{Poção do Amor. (6º nível ou superior)}

Requer \textit{um frasco de poção ornamentado}, \textit{água salgada},
\textit{um fio de cabelo da pessoa que se deseja enfeitiçar}, \textit{uma gota
de sangue da pessoa pela qual a outra ficará enfeitiçada}, \textit{pó de ouro
equivalente a 5po}, \textit{uma forma de aquecer o líquido}, 1000 de mana.

Ao criar essa poção, você precisa de um fio de cabelo da pessoa que deve beber
essa poção e de uma gota de sangue da pessoa pela qual o bebedor ficará
enfeitiçado. Após beber, a criatura ficará enfeitiçada pela outra por 1 hora
começando a contar quando as criaturas se verem pela primeira vez depois da
poção ser bebida. Se a criatura for da espécie ou gênero que o bebedor
geralmente é atraído, então ele a reconhece como seu amor verdadeiro enquanto
enfeitiçado. O líquido efervescente de coloração rosa desta poção contém uma
bolha em formato de coração difícil de perceber.

\subsubsection*{Curador das Mazelas}%

A partir do 6º nível, todas suas curas são melhoradas. Sempre que você realizar
uma cura, role o dobro de dados e pegue a metade maior.

\subsubsection*{Aliado Fiel}%

A partir do 11º nível, você se torna um aliado que todos querem ter ao lado.
Toda magia com a marca \textit{buff} pode ser realizada como uma ação bônus.
Além disso, você pode se concentrar em duas magias ao mesmo tempo desde que uma
delas tenha a marca \textit{buff}.

\subsubsection*{Curador das Massas}%

A partir do 17º nível, sempre que uma magia falar que cura uma criatura, você
pode escolher mais duas criaturas como alvo.

\end{multicols}
