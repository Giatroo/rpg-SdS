%!TeX root=../Rules.tex
\chapter{Classes}%
\label{cha:classe}
\begin{multicols}{2}

Todas as classe de SdS possui algumas características em comum que, a menos que
se diga o contrário, funcionam todas da mesma forma.

\section{Conjuração}%

Todas as classes de SdS são conjuradoras natas. Algumas sendo mais eficiente que
outras em fazê-lo.

Essas classes utilizam \textbf{mana} para conjurar magias e só podem conjurar as
magias da \textbf{escola} que possuem.

A tabela a seguir define a quantidade de mana por nível de personagem:

\begin{center}
\begin{tabular}{|||c||c|||}
    \hline
    \textbf{Nível} & \textbf{Mana} \\
    \hline
    \hline
    1 & 500 \\
    \hline
    2 & 800 \\
    \hline
    3 & 1600 \\
    \hline
    4 & 2000 \\
    \hline
    5 & 4000 \\
    \hline
    6 & 4800 \\
    \hline
    7 & 9000 \\
    \hline
    8 & 10000 \\
    \hline
    9 & 20000 \\
    \hline
    10 & 24000 \\
    \hline
    11 & 45000 \\
    \hline
    12 & 50000 \\
    \hline
    13 & 100000 \\
    \hline
    14 & 120000 \\
    \hline
    15 & 240000 \\
    \hline
    16 & 280000 \\
    \hline
    17 & 550000 \\
    \hline
    18 & 750000 \\
    \hline
    19 & 1400000 \\
    \hline
    20 & 2000000 \\
    \hline
\end{tabular}
\end{center}

\textit{Os valores acima de 5 não estão definidos, pois os testes serão feitos
primeiro nos níveis iniciais (truques e magias de 1º e 2º ciclos).}

As escolas de magia existente são: \textbf{\textit{Elemental}},
\textbf{\textit{Ilusionista}}, \textbf{\textit{Necromancia}},
\textbf{\textit{Invocação}}, \textbf{\textit{Psíquica}},
\textbf{\textit{Musical}}, \textbf{\textit{Espiritual}},
\textbf{\textit{Atrativa}}, \textbf{\textit{Pura}}.

Abaixo há uma descrição curta de cada uma delas:

\subsection*{Ilusionista}%

Escola exercida por alguns dos povos do sul, ela contém magias capaz de
dissimular os sentidos, como visão, audição e olfato.

\subsection*{Necromancia}%

Escola banida de todas as grandes cidades de Estália e Cafrilar, a necromancia
utiliza os poderes contidos nas coisas mortas. A Necromancia é uma escola tão
diferente que, quando um necromante existe, ele rapidamente torna-se conhecido
pelos seus feitos esquisitos.

\subsection*{Invocação}%

Assim como a Necromancia, a Invocação é tratada como um tipo de magia maléfica,
em que se traz para a vida criaturas, impondo a elas a vontade do conjurador.
Ela é extremamente rara.

\subsection*{Psíquica}%

Igualmente rara, mas essa é uma das escolas mais cobiçadas dentro de Estália.
Acredita-se que escola de magia Psíquica é a única que não precisa transportar
mana do plano astral para o material, pois ela repercute na própria mente de
cada indivíduo.

Os efeitos são os mais diversos e inesperados.

\subsection*{Musical}%

Uma escola de magia cobiçada pelos bardos do norte e ensinada a poucos por
algumas seletas guildas de músicos do sul que sabem como controlar mana
utilizando instrumentos musicais ou a própria voz.

É uma escola da magia muito diferente de qualquer outra, capaz de encantar e
curar criaturas, ou então fazê-las enlouquecer e sentir sensações horríveis como
uma magia psíquica é capaz de fazer.

\subsection*{Espiritual}%

Um tipo de magia bastante incomum, que envolve de alguma forma espíritos e
deuses. São poucas as magias conhecidas que se encaixam nessa escola.

\subsection*{Pura}%

Uma escola realmente lendária e teórica. Uma magia é dita Pura quando emite mana
de forma pura e visível no mundo material. Essas magias utilizam uma quantidade
desumana de mana, mas possuem poder além da compreensão.

\subsection*{Elemental}%

Essa sim é a mais comum de todas as escolas de magia e é a única que possui
subdivisões internas chamadas de \textit{\textbf{Ordens}}.

A escola de magia Elemental explora os elementos do mundo material:
\textbf{Água}, \textbf{Fogo}, \textbf{Terra}, \textbf{Ar}, \textbf{Luz},
\textbf{Sombras}, \textbf{Relâmpago}, \textbf{Veneno} e \textbf{Metal}. As
quatro primeiras ordens são de fato as mais comuns. Luz é uma ordem incomum, mas
muitas casas nobres de Calífora ou Nize possuem conjuradores dessa ordem. Já
Relâmpago e Sombras são realmente raras, sendo Sombra a mais rara de todas.

\subsubsection{Água}%

A Ordem da Água é capaz de utilizar o poder da água para as mais diversas
tarefas. Um conjurador da água é capaz desde criar um simples copo d'água quando
está com sede até remover toda a água do corpo de uma criatura, fazendo-a morrer
desidratada.

Conjuradores da Água são conhecidos principalmente por seu poder de cura e
fortalecimento de aliados no campo de batalha. Com um simples toque, um
conjurador da Água pode remover ferimentos, deixar um aliado mais veloz e
certeiro ou um inimigo mais fraco.

Mas nem só de utilidade é feito um conjurador da Água. Eles podem criar jatos de
água tão velozes e finos que são capaz de cortas árvores em duas ou então criar
ondas gigantes capaz de afetar todo o campo de batalha.

\subsubsection*{Fogo}%

Sem dúvida nenhuma, a palavra chave de um conjurador do Fogo é \textit{dano}.

Eles são os conjuradores mais destrutíveis dentro de uma batalha, capazes de
impor uma quantidade absurda de dano a um adversário em um piscar de olhos.

Esses conjuradores são capaz de incendiar o próprio corpo, lançar chamas contra
seus inimigos, criar explosões no meio do campo de batalha ou até mesmo conjurar
enormes bolas de fogo.

\subsubsection*{Terra}%

Resiliência e controle são as palavras chave que definem a Ordem da Terra.

Esses conjuradores são capaz de transformar partes do seu corpo em pedra, criar
tremores de terra que derrubam todos no campo de batalha ou então lançar pedras
gigantescas contra inimigos ou muralhas.

A Ordem da Terra também possui formas de fortalecer aliados e até algumas curas.

\subsubsection*{Ar}%

Os conjuradores do Ar são extremamente velozes e ágeis, acabando com os inimigos
antes mesmo de eles perceberem sua presença.

Assim como a Ordem do Fogo, é uma Ordem capaz de causa dano descomunal.
Entretanto, a Ordem do Ar possui também várias técnicas de defesa como
\textit{Parede de Vento} e fortalecimentos que deixam um conjurador intangível.

\subsubsection*{Luz}%

Uma Ordem com uma das maiores versatilidades. A Ordem da Luz possui também muito
dano, mas ao mesmo possui algumas das melhores curas e fortalecimentos de
aliados que um conjurador encontra dentro das ordens e escolas mais comuns.

Um conjurador da Luz consegue emitir uma luz curativa contra seus aliados,
emitir discos de luz capazes de cortas os inimigos ao meio ou então lançar um
flash de luz capaz de cegar todos.

Além disso, esses conjuradores obviamente conseguem transformar um local escuro
em claro com um simples truque mágico, tornando um objeto luminescente.

\subsubsection*{Sombras}%

Uma Ordem que é um legado dos \textit{Kage-Same} de Calífora. Até hoje, apenas
seis conjuradores são conhecidos por possuir esse tipo de magia e, sem dúvida, é
um trunfo militar da cidade mais desenvolvida da Península do Goto.

\subsubsection*{Relâmpago}%

A capacidade de controlar os relâmpagos é rara e não transmitida genéticamente
como as anteriores.

Conjuradores do Relâmpago são capazes de criar lâminas de relâmpagos que podem
ser jogadas contra oponentes, fazendo-os explodir de tanta energia concentrada.

Eles são quase tão ágeis quanto os conjuradores do Ar e quase tão poderosos
quanto os do Fogo, combinando talvez o melhor dos dois mundos.

\subsubsection*{Veneno}%

Uma Ordem menos conhecida no Norte, mas muito famosa entre as guildas de
assassinos do sul, conjuradores do Veneno são mortais.

Em alguns casos, um simples arranhão de uma lâmina sobre efeito de uma magia de
Veneno é capaz de assinar gigante em segundos.

Não só isso, mas eles conseguem cuspir veneno ou fazer ele efeitos dos mais
nojentos e questionáveis.

\subsubsection*{Metal}%

Uma Ordem bastante cobiçada por ferreiros e por aqueles que gostam da linha de
frente do campo de batalha. Um conjurador do Metal é capaz de criar qualquer
tipo de metal em seu próprio corpo ou em lugares que possa ver.

Conjuradores do Metal transformam a expressão ``perceba a lâmina como uma
extensão do seu corpo'' em algo ultrapassado.

\subsection*{Escolas e Ordens Disponíveis}%

Independente da classe escolhida pelo jogador, as ordens da Água, Fogo, Terra,
Ar, Luz, Metal e Veneno são aquelas disponíveis dentro da Escola Elemental, além
das escolas de Ilusionismo e Musical.

As outras ordens e escolas podem ser liberadas pelo personagem conforme ele
avança em sua jornada, dependendo da classe que escolheu. Para liberar essas
ordens, o personagem deve fazer alguma missão especial determinada de campanha
para campanha.

\subsection*{Truques}%

Todas as classes conhecem truques desde o 1º nível. Já os outros ciclos de magia
estão determinados em cada classe a partir de que nível aquela classe o conhece.

\subsection*{Energia Vital}%

A menos que se diga o contrário, se um conjurador possui apenas $20\%$ de sua
mana restante, então ele começa a fazer testes de Constituição sempre que
utiliza mais mana para qualquer coisa. O teste é 10 + (20\% - porcentagem atual
de mana restante).  Cada falha acarreta em um nível de exaustão.

Caso o conjurador possua apenas $5\%$ de mana restante, então uma falha no teste
acarreta em dois níveis de exaustão.

Independe do nível de exaustão, se um personagem chega em $0$ de mana, ele
morre imediatamente.

\subsection*{Recuperar Mana}%
\label{sub:recuperar_mana}

Existem várias formas de se recuperar mana, algumas mais naturais que outras.
A mais natural é tendo uma boa noite de descanso. Um conjurador recupera 100\%
dos seus pontos de mana em um Descanso Longo bem descansado e apenas 80\% caso o
descanso não tenha sido bem descansado.

Geralmente, um descanso longo não é bem descansado se os personagens acabam não
dormindo em um lugar ou posição confortável, como presos em uma cela, ou em um
abrigo improvisado.

Em um Descanso Curto, também é possível recuperar mana. Caso o conjurador esteja
abaixo de $15\%$, ele recupera mana até ficar exatamente com $15\%$.

Além disso, é possível recuperar mana com poções e alguns itens mágicos e
magias.

\subsection*{Conjuração Armada}%

Para conjurar magias, o conjurador deve ter um foco em canalizar sua mana pelo
corpo e expressá-la no Mundo Material em forma de magias. Portanto, conjurar
magias com as duas mãos ocupadas (seja por duas armas ou seja por uma arma e um
escudo) ou utilizando uma armadura pesada é muito mais difícil.

A menos que se diga o contrário, conjuradores possem certas desvantagens ao
conjurar magias quando estão com as duas mãos ocupadas ou utilizando armadura.

Se uma magia possui dados que devem ser rolados (por qualquer motivo), então você
deve rolar o dobro dos dados citados na magia e pegar os menores dados (por
exemplo, se a magia pede para rolar 3d6, então role 6d6 e pegue os 3 menores).

Se a magia diz que um inimigo deve fazer uma salvaguarda, então o inimigo o faz
com vantagem. Se a magia é qualquer tipo de buff ou debuff (por exemplo,
armaduras e escudos mágicos, magias que impõem desvantagem ao inimigo sem que ele
tenha que passar em salvaguardas, magias que adicionam dados a salvaguardas de
aliados...), então você deve passar em um teste CD 12 menos seu modificador de
conjuração para que a magia realmente seja efetiva.

Se o conjurador está utilizando um foco de conjuração (cajado, varinha ou
similares), então essa mão é considerada como mão livre (por exemplo, ele pode
utilizar cajado e adaga e essa regra não se aplica).

\subsection*{Conjuração de Monstros}%

A menos que se diga o contrário, os monstros e inimigos em geral sempre pousem a
regra Conjuração Armada aplicada a eles.

\section{Pedras de Poder}%

Equipamentos mágicos podem receber Pedras de Poder, que são cristais energizados
e cheios de mana capazes de proporcionar efeitos especiais. Cada equipamento
pode comportar um certo número de Pedras de Poder existem pedras mais poderosas
que podem ocupar mais de um espaço.
Abaixo segue uma descrição dos equipamentos e quantos espaços cada um tem e,
depois, de todas as pedras, seus efeitos e em que equipamentos elas podem ser
colocadas. A quantidade de espaços de pedras que um item contém pode,
entretanto, ser diferente da mencionada abaixo.
Em geral, Pedras de Poder podem ser encontradas como tesouro em catacumbas,
masmorras e vilarejos abandonados, mas também há a opção de se comprar ou ir
atrás dos santuários de onde essas pedras são extraídas. Em ambos os casos,
colocar uma pedra de poder em um equipamento é uma tarefa que apenas um Xamã
muito poderoso consegue fazer e, em geral, ele cobra bastante caro por isso.

\end{multicols}

\subsection*{Equipamentos}%

\begin{center}
\begin{tabular}{lr}
    \textbf{Nome} & \textbf{Espaços} \\
    \hline
    \multicolumn{2}{l}{\t\textit{Armas}} \\
    \t\t Armas Simples Corpo-a-Corpo & 1 espaço \\
    \t\t Armas Simples à Distância & 1 espaço \\
    \t\t Armas Marciais Corpo-a-Corpo & 1 espaço p/ cada nível mágico \\
    \t\t Armas Marciais à Distância & 1 espaço p/ cada nível mágico \\
    \t\t Focos arcanos & 1 espaço p/ cada nível mágico \\
    \t\t Instrumentos musicais & 1 espaço p/ cada nível mágico \\
    \hline
    \multicolumn{2}{l}{\t\textit{Armaduras}} \\
    \t\t Armaduras Leves & 1 espaço p/ cada nível mágico \\
    \t\t Armaduras Médias & 1 mais 1 espaço p/ cada nível mágico \\
    \t\t Armaduras Pesadas & 2 mais 1 espaço p/ cada nível mágico \\
    \t\t Escudos & 1 espaço p/ cada nível mágico \\
    \t\t Robes & 1 mais 1 espaço p/ cada nível mágico \\
    \hline
    \multicolumn{2}{l}{\t\textit{Equipamentos Gerais}} \\
    \t\t Alvajas & 1 espaço \\
    \t\t Canetas tinteiro & 1 espaço \\
    \t\t Lâmpadas, lanternas e postes & 1 espaço \\
    \t\t Marretas, martelos, pás e picaretas & 1 espaço \\
    \hline
    \multicolumn{2}{l}{\t\textit{Outros Itens Mágicos}} \\
    \t\t Anéis & 1 pedra \\
    \t\t Botas, cintos, capas e luvas & 1 espaço p/ cada nível mágico \\
    \t\t Amuletos, broches, braceletes... & 1 pedra \\

\end{tabular}
\end{center}

\subsection*{Pedras}%

\begin{center}
\begin{longtable}{ m{6cm} m{2cm} m{8cm} }
    \textbf{Nome/Equipamento} & \textbf{Espaços} & \textbf{Descrição} \\

    \hline \hline \hline \hline

    %% Safira
    \multicolumn{2}{l}{\t\textit{Safira Azul}} \\
    \t\t Todos exceto Gerais & 1 espaço & 25\% da mana total como mana adicional
    \\
    \hline \hline

    \multicolumn{2}{l}{\t\textit{Safira Azul-Escura}} \\
    \t\t Todos exceto Gerais & 2 espaços & todos efeitos da Safira Azul \\
    && curas tem seu valor final dobrado \\
    && pontos de vida temporário são dobrados \\
    && rolagens de buffs são feitos com vantagem \\
    \hline \hline

    \multicolumn{2}{l}{\t\textit{Safira Rosa}} \\
    \t\t Todos exceto Gerais & 3 espaços & todos efeitos da Safira Azul-Escura \\
    && em um descanso curto, pode recuperar 20\% da sua mana total e
    distribuir isso entre criaturas voluntárias. Essa recuperação pode
    ocorrer apenas uma vez a cada descanso longo. \\

    \hline \hline \hline \hline

    %% Rubi
    \multicolumn{2}{l}{\t\textit{Rubi Vermelho}} \\
    \t\t Armas & 1 espaço & transformam-se em focos de conjuração (ver
    Conjuração Armada) \\ \hline
    \t\t Luvas & 1 espaço & +1 de força \\ \hline
    \t\t Botas & 1 espaço & +1 de destreza \\ \hline
    \t\t Capas & 1 espaço & +1 de inteligência \\
    \hline \hline

    \multicolumn{2}{l}{\t\textit{Rubi Rosa}} \\
    \t\t Armas & 2 espaços & transforma uma arma +1 em uma arma +2 e assim por
    diante\\ \hline
    \t\t Luvas & 2 espaços & +2 de força \\ \hline
    \t\t Botas & 2 espaços & +2 de destreza \\ \hline
    \t\t Capas & 2 espaços & +2 de inteligência \\ \hline
    \t\t Anéis, amuletos, broches... && escolha duas magias de quaisquer
    escolas de um nível de magia que você domina e agora você sabe
    conjurá-las \\
    \hline \hline

    \multicolumn{2}{l}{\t\textit{Rubi Negro}} \\
    \t\t Armas & 3 espaços & transforma uma arma +1 em uma arma +2 e assim por
    diante\\&& diminui um no valor para ser um acerto crítico (20 para 19, ...)
    \\&&em um crítico, role três vezes o número de dados ao invés de duas\\
    \hline
    \t\t Luvas & 3 espaços & +3 de força \\ \hline
    \t\t Botas & 3 espaços & +3 de destreza \\ \hline
    \t\t Capas & 3 espaços & +3 de inteligência \\ \hline
    \t\t Anéis, amuletos, broches... && escolha cinco magias de quaisquer
    escolas de um nível de magia que você domina e agora você sabe
    conjurá-las \\

    \hline \hline \hline \hline

    %% Topázio
    \multicolumn{2}{l}{\t\textit{Topázio incolor}} \\
    \t\t Armaduras e escudo & 1 espaço & +1 CA \\ \hline
    \t\t Robes & 1 espaço & +1 de constituição \\
    \hline \hline

    \multicolumn{2}{l}{\t\textit{Topázio laranja}} \\
    \t\t Armaduras e escudo & 2 espaços & +2 CA \\ \hline
    \t\t Robes & 2 espaços & +2 de constituição \\ \hline
    \t\t Botas & 2 espaços & imune a efeitos que reduzam sua movimentação \\ \hline
    \t\t Cintos & 2 espaços & imune a envenenamentos \\ \hline
    \t\t Capas & 2 espaços & imune a encantamentos \\ \hline
    \t\t Anéis, amuletos, broches... && +25\% da vida total como vida temporária
    a cada descanso longo \\
    \hline \hline

    \hline \hline \hline \hline

    %% Ametista
    \multicolumn{2}{l}{\t\textit{Ametista Violeta}} \\
    \t\t Luvas & 1 espaço & você ganha movimentação de escalada igual
    metade do seu movimento \\ \hline
    \t\t Botas & 1 espaço & você pode usar Disparada como uma ação bônus \\
    \hline
    \t\t Anéis, amuletos, broches... && uma vez por descanso longo,
    você pode usar uma magia sem gastar mana \\
    \hline \hline

    \multicolumn{2}{l}{\t\textit{Ametista Púrpura}} \\
    \t\t Armas & 1 espaço & remove as características \textit{pesada} e
    \textit{duas mãos} se a arma tive-las \\ && se a arma não tiver,
    transforma-se em \textit{leve} \\ && se a arma já for \textit{leve}, seu
    peso diminui pela metade \\ \hline
    \t\t Armaduras e escudo & 1 espaço & diminui seu peso pela metade \\ \hline
    \t\t Luvas & 1 espaço & você pode segurar objetos incandescentes \\ \hline
    \t\t Botas & 1 espaço & você pode andar sobre a água \\ \hline
    \t\t Caneta tinteiro & 1 espaço & permite usar \textit{Escrita Ilusória} sem
    qualquer gasto \\ \hline
    \t\t Anéis, amuletos, broches... && +1 em qualquer perícia \\
    \hline \hline

    \multicolumn{2}{l}{\t\textit{Ametista Quartzo}} \\
    \t\t Botas & 1 espaço & terreno difícil custa movimento extra para você \\
    \hline
    \t\t Lâmpadas, lanternas... & 1 espaço & alcance da luz dobrado \\
    \hline
    \t\t Anéis, amuletos, broches... &&  +1 em qualquer atributo \\
    \hline \hline

\end{longtable}
\end{center}

\subsubsection{Esmeralda}%

Controle de grupo

\subsubsection{Topázio}%

Defesa

\subsubsection{Diamante}%

\begin{multicols}{2}
\end{multicols}

